\documentclass[11pt,a4paper,oneside]{article}
\linespread{1.5}
\usepackage{amsmath,amsthm,amssymb,amsfonts,adjustbox,bm}
\usepackage{geometry}
\usepackage{threeparttable}
%\usepackage[dvipsnames]{xcolor}
\usepackage{mathtools}
\usepackage{refstyle}
\usepackage{pdflscape}
\usepackage{longtable}
\usepackage{eurosym}
\usepackage{enumerate}
\usepackage{booktabs}
\usepackage{siunitx}
\usepackage{rotating}
\usepackage{graphicx}
\usepackage{bbm}
\usepackage{subcaption}
\usepackage{caption}
\captionsetup{width=.8\textwidth}
\usepackage[section]{placeins}
\usepackage{xcolor}
\usepackage{geometry}
\usepackage{changepage}
\usepackage{float}
\usepackage[natbibapa]{apacite}
\usepackage{threeparttable}

\makeatletter
\NAT@longnamesfalse
\makeatother


\newcommand{\footremember}[2]{%
   \footnote{#2}
    \newcounter{#1}
    \setcounter{#1}{\value{footnote}}%
}
\newcommand{\footrecall}[1]{%
    \footnotemark[\value{#1}]%
} 
\usepackage{lineno}
\makeatletter
\def\makeLineNumberLeft{%
  \linenumberfont\llap{\hb@xt@\linenumberwidth{\LineNumber\hss}\hskip\linenumbersep}% left line number
  \hskip\columnwidth% skip over column of text
  \rlap{\hskip\linenumbersep\hb@xt@\linenumberwidth{\hss\LineNumber}}\hss}% right line number
\leftlinenumbers% Re-issue [left] option
\makeatother

\usepackage{blindtext}
\usepackage[colorlinks=true, citecolor=blue, linkcolor=blue, urlcolor=blue,breaklinks]{hyperref}

% fast compile
% \pdfcompresslevel=0
% \pdfobjcompresslevel=0


\title{Class differences in social networks: Evidence from a referral experiment}
% \\ \large Experimental Design \\
% \title{Skills, Inequality, and Referrals}
\author{Manuel Munoz\footremember{liser}{Luxembourg Institute of Socio-Economic Research}, Ernesto Reuben\footnote{Division of Social Science, New York University Abu Dhabi} \footrecall{liser}, Reha Tuncer\footnote{University of Luxembourg}}  
% \footnote{Center for Behavioral Institutional Design at New York University Abu Dhabi} 
\begin{document}
\linenumbers 
% HERE FOR LINES
\maketitle



\section*{Abstract}
Economic connectivity, defined as the share of high-SES connections in one's network, is a strong correlate of labor market income. Yet, low-SES individuals are typically at a disadvantage when it comes to knowing the right people. Referral hiring leverages networks and make explicit the role of economic connectivity where taste-based biases could further exacerbate low-SES outcomes. We conduct a field experiment with 734 university students to study the network compositions of different SES groups. We leverage enrollment networks to identify all potential referral candidates and conduct an incentivized referral exercise to reveal SES biases within these choice sets. We find that the university enrollment networks are highly segregated, with low-SES and high-SES individuals having a higher share of same-SES connections in their networks due to program selection (12\% and 31\% respectively). When considering ex post actualized choice sets for the observed referrals, the segregation becomes worse: Low-SES individuals connect with other low-SES individuals at rates 30\% higher than the university share, while high-SES individuals connect with other high-SES individuals at rates 55\% higher than the university share. Yet, we find no bias against low-SES individuals once we account for network structures. We randomly assign half of the participants to a condition where their referral candidate receives a fixed bonus on top of pay-for-performance referral incentives. We find that additional incentives for the referral candidate do not change social proximity with the referral nor the referral quality. Our findings suggest that systematic segregation patterns in networks that alter choice sets matter more than taste-based SES biases in referrals, and highlight the potential for institutional action in promoting SES diversity. \medskip \\
\textbf{JEL Classification:} C93, J71, D85, Z13 \\
\textbf{Keywords:} social capital, social networks, referral hiring, socioeconomic status, field experiment

\pagebreak
\section{Introduction}
Equally qualified individuals in terms of productivity face different labor market outcomes based on their socioeconomic status \citep{stansbury2024class}. This persistent inequality undermines meritocratic ideals and represents a substantial barrier to economic mobility. A key driver of SES-based inequality in the labor market stems from differences in social capital.\footnote{See for example \citet{loury_dynamic_1977,bourdieu_forms_1986} for pioneering work on the relationship between social position and human capital acquisition.} Economic connectivity, defined as the share of high-SES connections among low-SES individuals, is the most important facet of social capital because it correlates strongly with labor market income \citep{chetty2022na}. In this sense, a lack of social capital means lack of access to individuals with influential (higher paid) jobs and job opportunities. It implies having worse outcomes when using one's network to find jobs conditional on the capacity to leverage one's social network.\footnote{See for example \citet{lin1981asr,mouw2003asr} for differential outcomes while using contacts in job search, and \citet{smith2005ajs,Pedulla2019RaceAN} specifically for the effects of race conditional on network use.} 

Referral hiring—the formal or informal process where firms ask workers to recommend qualified candidates for job opportunities—is a common labor market practice that makes differences in social capital evident.\footnote{Referrals solve some frictions in the search and matching process and benefit both job-seekers and employers. As a consequence, referral candidates get hired more often, have lower turnover, and earn higher wages \citep{brown2016informal,dustmann2016referral,friebel2023employee}.} Since referrals originate from the networks of referrers, the composition of referrer networks becomes a crucial channel that propagates inequality. Similar individuals across socio-demographic characteristics form connections at higher rates \citep{mcpherson2001birds}, making across-SES (low-to-high) connections less likely than same-SES connections \citep{chetty2022na}. Referrals will thus reflect similarities in socio-demographic characteristics present in networks even in the absence of biases in the referral procedure—that is, even when referring randomly from one's network according to some productivity criteria.

Yet, experimental evidence shows referrals can be biased even under substantial pay-for-performance incentives beyond what is attributable to differences in network compositions, at least in the case of gender \citep{beaman_job_2018,hederos2025gender}. A similar bias against low-SES individuals may further exacerbate their outcomes. If job information is in the hands of a select few high-SES individuals to whom low-SES individuals already have limited network access due to their lack of economic connectivity, and high-SES referrers are biased against low-SES individuals—referring other high-SES individuals at higher rates than their network composition would suggest—we should expect referral hiring to further disadvantage low-SES individuals. 

The empirical question we answer in this paper is whether referrers exhibit bias against low-SES peers after accounting for differences in network SES composition. We also evaluate the causal impact of two different incentive structures on referral behavior.

In this study, we examine inequalities related to SES by curating a university-wide network dataset comprising over 4,500 students for whom classroom interactions are recorded along with individual attributes. We focus on the role of SES in referrals by experimentally investigating whether individuals who are asked to refer a peer tend to refer a same-SES candidate. We also explore potential mechanisms behind referral patterns by randomizing participants into two different incentive structures. To this end, we conducted a lab-in-the-field experiment with 734 students at a Colombian university. We instructed participants to refer a qualified student for tasks similar to the math and reading parts of the national university entry exam (equivalent to the SAT in the US system). To incentivize participants to refer qualified candidates during the experiment, we set earnings to depend on referred candidates' actual university entry exam scores.

Referral hiring in the labor market can range from firm-level formal referral programs asking employees to bring candidates to simply passing on job opportunities between network members \citep{topa2019social}. Since our participants are students at the university and refer based on exam scores, we abstract away from formal referral programs with defined job openings. Our setting instead resembles situations where contacts share opportunities with each other without requiring the referred candidate to take any action and without revealing the referrer's identity. This eliminates reputational concerns since there is no hiring employer. It also establishes a lower bound on the expected reciprocity for the referrer when combined with pay-for-performance incentives \citep{witte2021workers,bandiera2009social}. At the same time, referring based on university entry exam scores is still an objective, widely accepted measure of ability. We show evidence that referrers in our setting not only possess accurate information about these signals but can also screen more productive individuals from their university network.

In a university setting, class attendance provides essential opportunities for face-to-face interaction between students. This is a powerful force that reduces network segregation by providing ample opportunities to meet across SES groups, because of exposure to an equal or higher level of high-SES individuals compared to the general population \citep{chetty2022n}.\footnote{In a different sample from the same university population, \citet{diaz_peer_2025} show this holds true for the highest-SES individuals at this institution, accounting for about 6\% of their sample but less than 5\% of Colombian high-school graduates \citep{fergusson2021desigualdad}.} The very high level of income inequality in Colombia makes SES differences extremely visible in access to tertiary education, where rich and poor typically select into different institutions \citep{jaramillo-echeverri2023}. However, in the particular institutional setting we have chosen for this study, different SES groups mix at this university, allowing us to focus on SES diversity within the institution. At the same time, as students take more classes together, their similarities across all observable characteristics tend to increase \citep{kossinets_origins_2009}. This is an opposite force that drives high- and low-SES networks to segregate. We observe the net effect of these two opposing forces using administrative data and construct class attendance (enrollment) networks for 734 participants based on the number of common courses they have taken together with other students. This allows us to directly identify aggregate characterizations of different SES groups' network compositions as a function of courses taken (e.g., in same-SES share), as well as the individual characteristics of network members who receive referrals among all possible candidates. 

We find strong evidence that networks of high- and low-SES participants exhibit same-SES bias. On average, both groups connect with their own SES group at higher rates than would occur randomly given actual group shares at the university (12\% for low-SES and 31\% for high-SES). As students take more courses together within the same program, their networks dwindle in size and become even more homogeneous in SES shares. At 12 courses together (the median number of courses taken together among referrals), the same-SES share increases to 30\% above the university share for low-SES students and 55\% above for high-SES students. We identify selection into academic programs as a key mechanism explaining this phenomenon: The private university where our study took place implements exogenous cost-based program pricing and does not offer SES-based price reductions. This results in programs with very large cost differences within the same university, with some programs costing up to six times the cheapest one. We find that the average yearly fee paid per student increases with SES, and the high-SES share in the most expensive program at the university—medicine—drives a large part of the network segregation across SES groups. 

Do segregated networks account for the differences in SES referral rates across SES groups? Same-SES referrals are 17\% more common than referrer networks suggest. Controlling for differences in network compositions, we find that the entirety of the bias is driven by low-SES referrers. We find no bias against low-SES peers beyond what is attributable to differences in network composition. Regardless of SES, participants refer productive individuals, and referred candidates are characterized by a very high number of courses taken together. The latter underlies the impact of program selection on the intensity of social interaction, where participants activate smaller and more homogeneous parts of their networks for making referrals. Our treatment randomized participants across two different incentive schemes by adding a substantial monetary bonus (\$25) for the referred candidate on top of the pay-for-performance incentives. We provide evidence that treatment incentives did not change referral behavior across the same-SES referral rate, the number of courses taken together with the referral candidate, and the candidate's exam scores. We interpret the lack of differences in the number of courses taken together as further evidence that referrals go to strong social ties across both treatments regardless of the incentive structure.\footnote{This follows directly from earlier evidence showing that referrals tend to go to strong ties, i.e., close friends and/or family members \citep{gee2017social,kramarz2014strong,wang2013marriage}.}
  
Our main empirical contribution to the experimental referral literature is our observation of the entire network that characterizes the referral choice set. Earlier research compares referrals made across different incentive structures and makes inferences about the counterfactual. For example, \citet{beaman_who_2012} compared referrers paid based on their referred candidate's productivity instead of receiving a fixed finder's fee, and \citet{beaman_job_2018} compared referrers who were restricted to refer either a male or female candidate instead of choosing freely. While \citet{pallais_why_2016} recruited a random sample of non-referred workers for comparison with referred ones, none of the previous studies could identify the entire referral choice set and provide a direct comparison to those who were referred by the participants. Observing the entire network allows us to identify biases in referrals in a more natural way, without imposing restrictions on the choice sets. A similar approach to ours is \citet{hederos2025gender}, who elicited friendship networks by asking referrers to name 5 close friends. Their findings suggest only half of those who were referred were from the elicited friendship network, and thus represent an incomplete observation of the entire referral choice set. We take our analysis one step further by requesting referrals from the enrollment network, where we have complete information on every single connection that may or may not receive a referral. This allows us to neatly separate the effect of network composition from any potential biases stemming from the referral procedure itself.  

Second, we build upon the earlier work on inequalities in referrals and the role of SES differences. The reliance of labor markets on referrals, coupled with homophily in social networks, can lead to persistent inequalities in wages and employment \citep{montgomery_social_1991,calvo2004effects,bolte_role_2021}. The premise of these models is that referrals exhibit homophily, so that employees are more likely to refer workers of their own race, gender, SES, etc. Supporting evidence shows that low-SES individuals have networks with lower shares of high-SES individuals, which partly explains why they have worse labor market outcomes \citep{stansbury2024class,chetty2022na}. We contribute by separately identifying the role of network homophily (the tendency to connect with similar others) and referral homophily (the tendency to refer similar others). Our results suggest that network homophily, rather than referral homophily, drives SES inequality in our setting. 

To our knowledge, \citet{diaz_peer_2025} are the first to study SES biases in referrals, and our study is conceptually the closest to theirs. Drawing from a similar sample at the same institution, \citet{diaz_peer_2025} focus on referrals from first-year students made within mixed-program classrooms and find no evidence for an aggregate bias against low-SES individuals. We also find no aggregate bias against low-SES individuals in referrals beyond what is attributable to differences in network structure. Our setup differs as we sample from students who completed their first year and impose no limits on referring from a classroom. This has several implications: We find that referrals in our setup go to individuals within the same program, and that programs have different SES shares which become even more accentuated as students take more courses together. While networks drive inequality in referral outcomes because of the institutional environment in our sample, we have no reason to believe first-year student networks in \citet{diaz_peer_2025} have similar levels of segregation to begin with. Our findings suggest that implementing more mixed-program courses that allow for across-SES mixing should be a clear policy goal to reduce segregation \citep{alan2023troeasa,rohrer2021po}.

The remainder of the paper is organized as follows. Section \ref{s2} begins with the background and setting in Colombia. In Section \ref{s3} we present the design of the experiment. In Section \ref{s4} we describe the data and procedures. Section \ref{s5} discusses the results of the experiment and Section \ref{s6} introduces robustness checks. Section \ref{s7} concludes. The Appendix presents additional tables and figures as well as the experiment instructions.

\section{Background and Setting} \label{s2}

Our experiment took place in Colombia, a country that consistently ranks highly in terms of economic inequality. The richest decile of Colombians earn 50 times more than the poorest decile \citep{barretoherrera2024regional,comisioneconomicaparaamericalatinayel2023}. This economic disparity creates profound differences in outcomes across SES groups in terms of education, geographic residence, language, manners, and social networks \citep{garciavillegas2021dimension, angulo2012movilidad,garcia2015loteria}. While these patterns are not atypical and exist elsewhere, Colombia's pronounced inequality makes economic, educational, and cultural differences across SES particularly visible and thus provides an ideal setting to study SES biases in referral selection.

We rely on Colombia's established estrato classification system to measure SES in our study. In 1994, Colombia introduced a nationwide system that divides the population into six strata based on  ``similar social and economic characteristics'' \citep[p. 102]{hudson_colombia_2010}. Designed for utility subsidies from higher strata to support lower strata, the system aligns with and reinforces existing social class divisions \citep{uribe2008estratificacion, guevara2019spatializing}. It is widely used by policymakers and in official statistics \citep{fergusson2021desigualdad}. Using the estrato system, we categorize students in strata 1-2 as low-SES, strata 3-4 as middle-SES, and strata 5-6 as high-SES.

Colombia's educational segregation typically prevents meaningful interaction between socioeconomic groups, as wealthy families attend exclusive private schools while poorer families access lower-quality public or ``non-elite'' private institutions (see Figure \ref{school_quality_income}). Our study takes place in a non-elite private university which attracts students across the socioeconomic spectrum: The university's student body comprises 35\% low-SES, 50\% middle-SES, and 15\% high-SES students.\footnote{Government statistics reveal less than 5\% of the population is high-SES \citep[p. 103]{hudson_colombia_2010}.} This diversity provides opportunities for different SES groups to meet and interact within the same institutional framework.

The partner university creates conditions for contact on equal status. All students pay the same fees based on their program choices, and less than 5\% of students receive scholarships. The student body is mostly urban and has comparable university entry exam scores due to the entrance exam \textcolor{red}{ADD NUMBERS}. These additional factors make our setting appropriate to study the effects of contact on intergroup discrimination.  

% These factors typically facilitate reducing bias between interacting groups \citep{allport1954nature,rao2019familiarity,mousa2020building,lowe2021types}

% For our experiment, we invited via email all 4,417 undergraduate students who had completed their first year at the university at the time of recruitment. These students vary in terms of their academic programs \textcolor{red}{ADD NUMBERS}, SES \textcolor{red}{ADD NUMBERS}, and progress in their studies \textcolor{red}{ADD NUMBERS}. 

% Undergraduate students take between 5 to 7 courses per semester, with programs lasting anywhere between 4 to 12 semesters (2 to 6 years). Medicine, the largest program by size at the university, lasts for 12 semesters, followed by engineering programs at 10 semesters. Most remaining programs last for about 8 to 10 semesters, with specialized programs for immediate entry into the workforce lasting only 4 semesters.

% 837 students joined the experiment (19\%). 

\begin{figure}[H]
  \centering
  \caption{Income, performance, and university choice in Colombia}\label{school_quality_income}
  \includegraphics[width=0.6\linewidth]{../figures/school_quality_income.png}
  \begin{tablenotes}
    \footnotesize
    \item[] \textit{Note:} This figure shows the average score national university entry exam by monthly family income and type of higher education institution. With average student score in the 65-70 band, the private university where we conducted this experiment caters to low- and high-SES students. Figure reproduced from \citet{fergusson2021distincion}.
  \end{tablenotes}
\end{figure}



\section{Design} \label{s3}
We designed an online experiment to assess peer referral selection from an SES perspective and to evaluate the causal effect of providing a bonus to referral candidates. The experimental design consisted of two incentivized tasks administered in the following sequence to all participants: First, participants completed belief elicitation tasks about their own performance on the national university entry exam. Second, they completed the main referral task, nominating peers based on exam performance in two academic areas. Finally, participants reported beliefs about their referrals' performance and provided demographic information. This structure allowed us to measure both the accuracy of participants' beliefs and their referral behavior under controlled incentive conditions. Figure \ref{timeline} shows the experimental timeline, and detailed instructions are provided in Appendix \ref{instructions}.

\begin{figure}[H]
\centering
\caption{Experimental Timeline}\label{timeline}
\includegraphics[width=\linewidth]{../figures/timeline.jpg}
\begin{tablenotes}
\footnotesize
\item[] \textit{Note:} Participants first reported beliefs about their own university entry exam performance, then made referrals for each academic area, and finally reported beliefs about their referrals' performance and provided demographics.
\end{tablenotes}
\end{figure}

\subsection{Performance measures}
To establish an objective basis for referral performance, we use national university entry exam scores (SABER 11). All Colombian high school students take the SABER 11 exam at the end of their final year as a requirement for university admission. These scores provide pre-existing, comparable measures of performance. By using existing administrative data, we also ensure that all eligible students have comparable performance measures.

The SABER 11 exam consists of five areas (critical reading, mathematics, natural sciences, social sciences, and English). We focus on critical reading and mathematics as these represent fundamental skills most relevant across academic programs and future employment. Critical reading evaluates competencies necessary to understand, interpret, and evaluate texts found in everyday life and broad academic fields (e.g., history), while mathematics assesses students' competency in using undergraduate level mathematical tools (e.g., reasoning in proportions, financial literacy). These together capture performance in comprehending and critically evaluating written material as well as reasoning and problem-solving abilities.

For each area, we calculate percentile rankings based on the distribution of scores among all currently enrolled students, providing a standardized measure of relative performance within the university population.

\subsection{Referral task}
The main experimental task involves making referrals among peers. For both exam areas (critical reading and mathematics), participants refer one peer they believe excels in that area. We provide an example question from the relevant exam area to clarify the skills that are being assessed. Participants type the name of their preferred candidate to make a referral. To avoid issues with recall, the interface provides autocomplete name and program suggestions from the administrative database (see Figure \ref{interface}). 

\begin{figure}[H]
  \centering
  \caption{Referral task interface}\label{interface}
  \includegraphics[width=0.6\linewidth]{/Users/reha.tuncer/Documents/GitHub/icfes-referrals/figures/interface.png}
  \begin{tablenotes} 
    \footnotesize
    \item[] \textit{Note:} This illustration shows how the system provides suggestions from enrolled students with their program and year of study from the administrative database.
  \end{tablenotes}
\end{figure}

Participants can only refer students with whom they have taken at least one class during their university studies. This requirement ensures that referrals are based on actual peer interactions. We randomize the order in which participants make referrals across the two exam areas.

We incentivize referrals using a piece rate payment structure. Referrers earn increasing payments as the percentile ranking of their recommendation increases (see Figure \ref{piece_rate_by_score}). We multiply the piece rate coefficient associated with the percentile rank by the actual exam scores of the recommendation to calculate earnings. This payment structure provides strong incentives to refer highly ranked peers with potential earnings going up to \$60 per referral. \footnote{Note that due to the selection into the university, the actual exam score distribution has limited variance. Below a certain threshold students cannot qualify for the institution and choose a lower ranked university, and above a certain threshold they have better options to choose from.}

\begin{figure}[H]
  \centering
  \caption{Referral incentives}\label{piece_rate_by_score}
  \includegraphics[width=0.6\linewidth]{/Users/reha.tuncer/Documents/GitHub/icfes-referrals/figures/piece_rate_by_score.png}
  \begin{tablenotes}
    \footnotesize
    \item[] \textit{Note:} This figure shows how the piece rate coefficient increases as a function of the referral ranking in the university, providing incrementally higher rewards for higher ranked peers.
  \end{tablenotes}
\end{figure}

\subsection{Bonus Treatment}
To examine how different incentive structures affect referral selection, we randomly assign a fixed bonus payment for students who get a referral. In the \textbf{Baseline} treatment, only the referrer can earn money based on their referral's performance. The \textbf{Bonus} treatment adds an additional fixed payment of \$25 to the peer who gets the referral. This payment is independent of the referred candidate's actual performance (see Table \ref{tab:treatments}).

\begin{table}[htbp]
  \centering
  \begin{threeparttable}
  \caption{Incentive structure by treatment}
  \begin{tabular*}{\textwidth}{@{\extracolsep{\fill}} l c c @{}}
  \toprule
  & \multicolumn{1}{c}{\textbf{Baseline}} & \multicolumn{1}{c}{\textbf{Bonus}} \\
  \midrule
  Referrer (sender) & Performance-based & Performance-based \\
  Referral (receiver) & No payment  & Fixed reward  \\
  \bottomrule
  \end{tabular*}
  \label{tab:treatments}
  \end{threeparttable}
\end{table}

We use a between-subjects design and randomly assign half our participants to the \textbf{Bonus} treatment. This allows us to causally identify the effect of the bonus on referral selection. Participants learn whether their referral gets the fixed bonus before making referral decisions.  

\subsection{Belief elicitation}
We collect two sets of incentivized beliefs to assess the accuracy of participants' knowledge about exam performance. Participants first report beliefs about their own percentile ranking in the university for each exam area. After making referrals, participants report their beliefs about their referrals' percentile ranking in the university. For both belief elicitation tasks, participants earn \$5 per correct belief if their guess is within 7 percentiles of the true value. This margin of error is designed to balance precision with the difficulty of the task.

% participants report the closeness of their relationship with each recommended peer on a 0-10 scale. This allows us to examine whether social proximity influences referral decisions and accuracy. . Finally, participants estimate the distribution of university students across these SES groups, providing insight into their perceptions of the university-wide SES distribution.

\section{Sample, Incentives, and Procedure} \label{s4}
We invited all 4,417 undergraduate students who had completed their first year at the university at the time of recruitment to participate in our experiment. A total of 837 students participated in the data collection (19\% response rate). Our final sample consists of 734 individuals who referred peers with whom they had taken at least one class together, resulting in an 88\% success rate for the sample. We randomly allocated participants to either \textbf{Baseline} or \textbf{Bonus} treatments.

Table \ref{tab:balance} presents key demographic characteristics and academic performance indicators across treatments (see Appendix Table \ref{tab:selection} for selection). The sample is well-balanced between the \textbf{Baseline} and \textbf{Bonus} conditions and we observe no statistically significant differences in any of the reported variables (all \textit{p} values $> 0.1$). Our sample is characterized by a majority of middle-SES students with about one-tenth of the sample being high-SES students. The test scores and GPA distributions are balanced. On average, participants had taken 3.8 courses together with members of their network, and the average network consisted of 175 peers.

\begin{table}[H]
  \centering
  \begin{threeparttable}
  \caption{Balance between treatments}
  \begin{tabular*}{\textwidth}{@{\extracolsep{\fill}}lcc c@{}}
  \toprule
   & \multicolumn{1}{c}{\textbf{Baseline}} & \multicolumn{1}{c}{\textbf{Bonus}} & \multicolumn{1}{c}{\textit{\textbf{p}}} \\ 
  \midrule
  Reading score & 64.712 & 65.693 & \multicolumn{1}{c}{0.134} \\
  Math score & 67.366 & 67.597 & \multicolumn{1}{c}{0.780} \\
  GPA & 4.003 & 4.021 & \multicolumn{1}{c}{0.445} \\
  Connections & 173.40 & 176.88 & \multicolumn{1}{c}{0.574} \\
  Courses taken & 3.939 & 3.719 & \multicolumn{1}{c}{0.443} \\
  Low-SES & 0.419 & 0.401 & \multicolumn{1}{c}{0.615} \\
  Middle-SES & 0.492 & 0.506 & \multicolumn{1}{c}{0.714} \\
  High-SES & 0.089 & 0.094 & \multicolumn{1}{c}{0.824} \\
  \midrule
  Observations & 382 & 352 & 734 \\
  \bottomrule
  \end{tabular*}
  \begin{tablenotes}[flushleft]
  \footnotesize
  \item[] \textit{Note:} This table presents balance tests between \textbf{Baseline} and \textbf{Bonus} conditions. $p$-values for binary outcomes are from two-sample tests of proportions; for continuous variables, from two-sample $t$-tests with unequal variances. All reported \textit{p}-values are two-tailed. Reading and math scores are in original scale units out of 100. GPA is grade point average out of 5. Connections refers to the average number of network members. Low-SES, Med-SES, and High-SES indicate SES categories based on strata.
  \end{tablenotes}
  \label{tab:balance}
  \end{threeparttable}
\end{table}

The experiment was conducted online through Qualtrics, with participants recruited from active students. To ensure data quality while managing costs, we randomly selected one in ten participants for payment. Selected participants received a fixed payment of \$17 for completion. They also received potential earnings from one randomly selected belief question (up to \$5) and one randomly selected referral question (up to \$60). This structure resulted in maximum total earnings of \$82. The average time to complete the survey was 30 minutes, with an average compensation of \$80 for the one in ten participants randomly selected for payment. Payment processing occurred through bank transfer within 15 business days of participation.

\section{Results} \label{s5}

\subsection{Network characteristics}
We begin by describing the key features of the enrollment network for all participants. This network connects every participant in our sample with another university student if they have taken at least one course together at the time of data collection. By doing so, we construct the entire referral choice set for participants. We include in this dataset both the participant's and their potential candidate's individual characteristics, as well as the number of common courses they have taken together. Figure \ref{connections_class} describes the evolution of the enrollment network across the average number of network connections and the number of common courses taken with network members as participants progress through semesters.

\begin{figure}[H]
  \centering
  \caption{Network size and courses taken together by time spent at the university}\label{connections_class}
  \includegraphics[width=0.6\linewidth]{../figures/connections.png}
  \begin{tablenotes}
    \footnotesize
    \item[] \textit{Note:} This figure displays the average number of connections in blue and the average number of classes they have taken together with their connections in grey across semesters spent. The data shows an increase in the number of classes taken together as students progress in their programs, with the connections peaking around 7 semesters and dropping as certain students finish their bachelor's.
  \end{tablenotes}
\end{figure}

Having established the overall network structure, we now examine differences across SES groups. Are enrollment networks different across SES groups? We look at how the number of connections (network size) and number of courses taken together (tie strength) change across SES groups in Figure \ref{connections_by_ses}. Low- and middle-SES students have larger networks but take fewer courses together with network members, while high-SES students have smaller, denser networks. Specifically, both low- and middle-SES students have significantly larger networks than high-SES students ($t = 3.03, p = .003$ and $t = 2.49, p = .013$, respectively), but high-SES students take significantly more courses with their network members than both low- ($t = -3.70, p < .001$) and middle-SES ($t = -4.20, p < .001$).

\begin{figure}[H]
  \centering
  \caption{Network size and courses taken together by SES}\label{connections_by_ses}
  \includegraphics[width=0.6\linewidth]{../figures/connections_by_ses.png}
  \begin{tablenotes}
    \footnotesize
    \item[] \textit{Note:} This figure displays the average number of connections and the average number of classes taken together across SES groups. The data shows a decrease in the number of connections with SES, and an associated increase in the number of classes taken together.
  \end{tablenotes}
\end{figure}

\subsection{SES diversity in networks}\label{s:ses-diversity}
What are the diversity-related consequences of SES-driven differences across networks? In terms of network compositions, SES groups may connect with other SES groups at different rates than would occur randomly (Figure \ref{availability_lses}).\footnote{Because we estimate the share of SES groups in every individual network, we get very precise estimates of the actual means. However, it is important to note that these are not independent observations for each network. Estimates are precise because each network is a draw with replacement from the same pool of university population, from which we calculate the proportion of SES groups per individual network, and take the average over an SES group. Pooling over SES groups who are connected with similar others systematically reduces variance (similar to resampling in bootstrapping). For this reason we choose not reporting test results in certain sections including this one and focus on describing the relationships between SES groups.} Our results reveal modest deviations from university-wide SES composition across groups. Low-SES students have networks with 38.4\% low-SES peers compared to the university average of 34.3\%, middle-SES students connect with 52.9\% middle-SES peers versus the university average of 50.5\%, and high-SES students show 20.4\% high-SES connections compared to the university average of 15.3\%.

\begin{figure}[H]
  \centering
  \caption{Network shares of SES groups}\label{availability_lses}
  \includegraphics[width=0.6\linewidth]{../figures/availability_lses.png}
  \begin{tablenotes}
    \footnotesize
    \item[] \textit{Note:} This figure compares the network shares of SES groups in the networks of low-, middle-, and high-SES participants. Horizontal lines plot the university-wide shares of each SES group (Low: 35\%, Mid: 51\%, High: 14\%). While the share of low-SES peers in the network decreases as the SES of the group increases, the share of high-SES peers in the network increases.
  \end{tablenotes}
\end{figure}

At the same time, we observe much larger differences between SES groups in their connection patterns with other groups. Low-SES students connect with other low-SES students at higher rates than middle-SES students (38.4\% vs 31.4\%) and high-SES students (38.4\% vs 25.1\%). Conversely, high-SES students connect more with other high-SES students than both low-SES students (20.4\% vs 12.6\%) and middle-SES students (20.4\% vs 15.8\%). Middle-SES students are in between the two extreme patterns, connecting with middle-SES peers at higher rates than low-SES students (52.9\% vs 49.0\%) but lower rates than high-SES students (52.9\% vs 54.5\%). These findings indicate SES-based network segregation, with same-SES homophily patterns across groups.

Having examined basic network composition, we now turn to connection intensity. So far we have looked at the entire network without considering the intensity of connections between students. In our network dataset, this variable amounts to the number of classes taken together with peers. As we will see in the next section, referrals go to peers with whom participants have taken an average of 14 courses, implying the intensity of the connection matters. We begin by dissecting what the intensity means in our context. As students take more courses together, the proportion of peers from the same academic program quickly goes beyond 95\% (see Figure \ref{share_same_program_by_tie}). Similarly, the average network size drops very quickly from above 210 to below 50 (see Figure \ref{conns_by_tie}). Both results indicate that actual referral considerations originate from a much smaller pool of individuals from the same academic program.

\begin{figure}[H]
  \centering
  \caption{Network characteristics and courses taken together}
  \begin{subfigure}[t]{0.49\textwidth}
    \centering
      \includegraphics[width=\linewidth]{../figures/share_same_program_by_tie.png}
      \caption{Same-program share}  
      \label{share_same_program_by_tie}
  \end{subfigure}
  \begin{subfigure}[t]{0.49\textwidth}
    \centering
      \includegraphics[width=\linewidth]{../figures/conns_by_tie.png}
      \caption{Network size}
      \label{conns_by_tie}
  \end{subfigure}
  \caption*{\footnotesize\textit{Note:} The left panel illustrates the share of connections within the same program as a function of the number of courses taken together. The right panel shows the average network size as a function of the number of courses taken together. Taking more than 5 courses together with a network member means on average 90\% chance to be in the same program. Similarly, past 5 courses together, the average network size dwindles by 80\%, from more than 210 individuals to below 50. 
  }
\end{figure}

This raises an important question: What are the diversity implications of increased connection intensity between students? As students take more courses together with peers, the share of same-SES peers in the networks of low- and high-SES increases while the share of middle-SES declines (see Figure \ref{share_same_ses_by_tie}). Both increases are substantial, amounting to 50\% for high-, and 30\% for low-SES. Combining these with the earlier result that beyond 5 courses taken together network members are almost entirely within the same program, these suggest program selection may have strong consequences for SES diversity in our setting.

\begin{figure}[H]
  \centering
  \caption{Network size and courses taken together by courses taken}\label{share_same_ses_by_tie}
  \includegraphics[width=0.6\linewidth]{../figures/share_same_ses_by_tie.png}
  \begin{tablenotes}
    \footnotesize
    \item[] \textit{Note:} This figure illustrates the shares of same-SES connections for low-, middle-, and high-SES as a function of the average number of courses taken together with network members. Low- and high-SES networks both become more homogenous as the average number of courses taken together with their connections increase.
  \end{tablenotes}
\end{figure}

\subsection{Program selection and SES diversity}
To understand the mechanisms driving these patterns, we examine program selection. Academic programs at this university use cost-based pricing, and typically less than 5\% of students receive any kind of scholarship \citep{diaz_peer_2025}. Based on this, we first calculate how much every program at the university is expected to cost students per year (see Figure \ref{fees_with_boxplot}). Considering that net minimum monthly wage stands at \$200 and the average Colombian salary around \$350, the cost differences between programs are large enough to make an impact on program selection. Is it the case that SES groups select into programs with financial considerations?

\begin{figure}[H]
  \centering
  \caption{Programs sorted by fee}\label{fees_with_boxplot}
  \includegraphics[width=0.6\linewidth]{../figures/fees_with_boxplot.png}
  \begin{tablenotes}
    \footnotesize
    \item[] \textit{Note:} This figure illustrates the distribution of programs at the university by their average yearly fee. The average yearly fee stands at \$3000, and medicine is an outlier at \$6000.
  \end{tablenotes}
\end{figure}

We examine how SES groups are distributed across programs to identify evidence of SES-based selection (see Figure \ref{ses_distribution_by_fees}). Indeed, low-SES students select into more affordable programs, followed by middle-SES students. High-SES students sort almost exclusively into above-average costing programs, with a third selecting into medicine and creating a very skewed distribution. The distributions are significantly different across all pairwise comparisons: low-SES vs. middle-SES (Kolmogorov-Smirnov test $D=33.89$, $p<0.001$), low-SES vs. high-SES ($D=31.31$, $p<0.001$), and middle-SES vs. high-SES ($D=31.31$, $p<0.001$). With this finding, program selection could be the reason why low- and high-SES networks tend to segregate as the number of courses taken increases. The next section characterizes the referrals, and we will return to the diversity implications of program selection once we propose an understanding of how referrals were made.

\begin{figure}[H]
  \centering
  \caption{Programs sorted by fee}\label{ses_distribution_by_fees}
  \includegraphics[width=0.6\linewidth]{../figures/ses_distribution_by_fees.png}
  \begin{tablenotes}
    \footnotesize
    \item[] \textit{Note:} This figure illustrates the distribution of each SES group across programs sorted by fee. the majority of low-SES select into programs with below average cost, while high-SES select into programs with above average cost. Medicine accounts for a third of all high-SES students at this university.
  \end{tablenotes}
\end{figure}

\subsection{Characterizing referrals}

We observe 1,342 referrals from our 734 participants in our final dataset. More than 90\% of these consist of participants referring for both areas of the national entry exam (see Appendix Table \ref{tab:referral_distribution}). While participants made one referral for Math and Reading parts of the exam, about 70\% of these referrals went to two separate individuals. We compare the outcomes across areas for unique referrals in Appendix Table \ref{tab:referral_by_area} and all referrals in Appendix Table \ref{tab:referral_area}. In both cases, we find no meaningful differences between referrals made for Math or Reading areas of the entry exam. As referrals in both exam areas come from the same referrer network, we pool referrals per participant and report their averages in our main analysis to avoid inflating statistical power in our comparisons.

What are the characteristics of the individuals who receive referrals, and how do they compare to others in the enrollment network? Because we have an entire pool of potential candidates with one referral chosen from it, we compare the distributions for our variables of interest between the referred and non-referred students.

First, referrals go to peers with whom the referrer has taken around 14 courses with on average, compared to almost 4 on average with others in their network (see Figure \ref{tie_hist}). This difference of 10.1 courses is significant ($t = 34.98$, $p < 0.001$), indicating that referrers choose individuals with whom they have stronger ties. While the median referral recipient has taken 12 courses together with the referrer, the median network member has shared only 2.8 courses. The interquartile range for referrals spans from 7.5 to 19.5 courses, compared to just 2.1 to 4.0 courses for the broader network, highlighting the concentration of referrals among peers with high social proximity and within same program (93\%).

\begin{figure}[H]
  \centering
  \caption{Courses taken together with network members and referrals}\label{tie_hist}
  \includegraphics[width=0.6\linewidth]{../figures/tie_hist.png}
  \begin{tablenotes}
    \footnotesize
    \item[] \textit{Note:} This figure compares the distributions of the number of courses taken together between referrers and their network members (orange) versus referrers and their chosen referral recipients (dark blue) for all 734 participants. 75\% of referral recipients having taken more than 7.5 courses together with the referrer, compared to only 25\% of network members. The distributions are significantly different (Kolmogorov-Smirnov test $D = 33.37$, $p < 0.001$).
  \end{tablenotes}
\end{figure}

Second, we examine entry exam score differences between referred students and the broader network. Referrals go to peers with an average score of 69.5 points, compared to 64.5 points for other network members (see Figure \ref{other_score_hist}). This difference of 5 points is significant ($t = 18.97$, $p < 0.001$), indicating that referrers choose higher-performing peers. While the median referral recipient scores 71 points, the median network member scores 65.1 points. The interquartile range for referrals spans from 65.5 to 75 points, compared to 63.5 to 66.9 points for the broader network, highlighting the clear concentration of referrals among higher performing peers.

\begin{figure}[H]
  \centering
  \caption{Entry exam scores of network members and referrals}\label{other_score_hist}
  \includegraphics[width=0.6\linewidth]{../figures/other_score_hist.png}
  \begin{tablenotes}
    \footnotesize
    \item[] \textit{Note:} This figure compares the distributions of entry exam scores (Math and Reading average) between referrers' network members (orange) versus their chosen referral recipients (dark blue) for all 734 participants. 75\% of referral recipients score above 65.5 points compared to only 25\% of network members scoring above 66.9 points. The distributions are significantly different (Kolmogorov-Smirnov test $D = 71.16$, $p < 0.001$). 
  \end{tablenotes}
\end{figure}

\subsection{Effect of the Bonus treatment}

Do referred individuals have different outcomes across treatments? We compare the performance, number of courses taken together, and SES shares of referred individuals between the \textbf{Baseline} and \textbf{Bonus} treatments in Table \ref{tab:referral_by_treatment}. While performance of referrals across Reading, Math, and GPA are similar across treatments, middle- and high-SES shares have significant differences. We find that referrals under the \textbf{Bonus} condition referred a higher proportion of high-SES individuals (13.5\% vs 8.8\%, $p=0.041$) and a lower proportion of middle-SES individuals on average (47.0\% vs 53.7\%, $p=0.072$). However, these differences do not appear to stem from systematic behavioral changes by any particular SES group of referrers, and the overall patterns remain largely consistent across treatments. The similarities in academic performance and number of courses taken together suggest that the core selection criteria—academic merit and social proximity—remain unchanged between conditions. For this reason, in the remainder of the paper, we report pooled results combining the averages of referral outcomes across treatments.

\begin{table}[H]
  \centering
  \begin{threeparttable}
  \caption{Characteristics of referrals by treatment condition}
  \begin{tabular*}{\textwidth}{@{\extracolsep{\fill}}lccc}
  \toprule
  & \multicolumn{1}{c}{\textbf{Baseline Referred}} & \multicolumn{1}{c}{\textbf{Bonus Referred}} & \multicolumn{1}{c}{\textbf{\textit{p}}} \\
  \midrule
  Reading score & 67.806 & 67.210 & 0.308 \\
  Math score & 70.784 & 70.155 & 0.406 \\
  GPA & 4.155 & 4.149 & 0.799 \\
  Courses taken & 13.840 & 14.065 & 0.723 \\
  Low-SES & 0.376 & 0.395 & 0.593 \\
  Middle-SES & 0.537 & 0.470 & 0.072 \\
  High-SES & 0.088 & 0.135 & 0.041 \\
  \midrule
  Observations & 382 & 352 \\
  \bottomrule
  \end{tabular*}
  \begin{tablenotes}[flushleft]
  \footnotesize
  \item[] \textit{Note:} This table compares the characteristics of network members who were referred under baseline vs bonus treatment conditions. $p$-values for binary outcomes are from two-sample tests of proportions; for continuous variables, from two-sample $t$-tests with unequal variances. All reported \textit{p}-values are two-tailed. Reading and math scores are raw test scores out of 100. GPA is grade point average out of 5. Courses taken is the number of courses participant has taken  with their referral. Low-SES, Med-SES, and High-SES are binary variables indicating the share of participants in estrato 1-2, 3-4, or 5-6, respectively. Both columns include only network members who were actually nominated for referral in each treatment condition.
  \end{tablenotes}
  \label{tab:referral_by_treatment}
  \end{threeparttable}
\end{table}

\subsection{Referral SES composition}
We first examine the overall SES compositions in referral selection. Referrals to low-SES peers constitute 37.9\% of all referrals, compared to 33.7\% low-SES representation in individual networks (see Figure \ref{fig:all_referral_rates}). This represents a modest over-representation of 4.3 percentage points. For middle-SES students, referrals constitute 51.0\% versus 51.4\% network representation, showing virtually no difference (-0.5 pp.). High-SES referrals account for 11.1\% compared to 14.9\% network share, an under-representation of 3.8 percentage points. While these patterns suggest some deviation from proportional representation—with slight over-referral to low-SES peers and under-referral to high-SES peers—the magnitudes are relatively modest. Overall, referral compositions are largely balanced and closely mirror the underlying network structure, with the largest deviation being less than 5 percentage points for any SES group.

\begin{figure}[H]
  \centering
  \caption{Referral patterns compared to network composition}
  \label{fig:all_referral_rates}
  \includegraphics[width=0.6\linewidth]{../figures/all_referral_rates.png}
  \begin{tablenotes}
    \footnotesize
    \item[] \textit{Note:} This figure compares the average SES composition of referrers' networks (dark gray) to the SES composition of referrals (light gray). Error bars represent 95\% confidence intervals.
  \end{tablenotes}
\end{figure}

Then, we examine referral patterns by referrer SES to identify potential SES biases across groups. Figure \ref{fig:all_ses_referral_rates} reveals mixed patterns of deviation from network composition that vary by referrer SES. Most patterns show modest deviations from network composition, with differences typically ranging from 1-6 percentage points. However, at the very extremes—low-SES to high-SES connections and vice versa—we observe the largest discrepancies between network share (which were already biased toward same-SES connections to begin with) and referral rates. Low-SES referrers show the strongest same-SES preference, referring 12.9 percentage points more to low-SES students than their network composition would suggest, while under-referring to high-SES recipients by 6.3 percentage points. Conversely, high-SES referrers under-refer to low-SES students by 10.9 percentage points compared to their network composition. Middle-SES referrers show the most balanced patterns, with deviations generally under 3 percentage points across all recipient groups. Cross-SES referral patterns, particularly between the most socioeconomically distant groups, show the largest departures from network availability. These results suggest that referral behavior diverges most from underlying network structure when SES differences are most pronounced.

\begin{figure}[H]
  \centering
  \caption{Referral patterns by referrer SES compared to network composition}
  \label{fig:all_ses_referral_rates}
  \includegraphics[width=0.6\linewidth]{../figures/all_ses_referral_rates.png}
  \begin{tablenotes}
    \footnotesize
    \item[] \textit{Note:} This figure compares the average SES composition of referrers' networks (dark gray) to the SES composition of referrals (light gray) for each SES group. The panels show referral patterns for low-SES (left), middle-SES (center), and high-SES referrers (right). Error bars represent 95\% confidence intervals.
  \end{tablenotes}
\end{figure}




% \begin{table}[H]
%   \centering
%   \begin{threeparttable}
%   \caption{Summary statistics for network members by referral status}
%   \begin{tabular*}{\textwidth}{@{\extracolsep{\fill}}lccccc}
%   \toprule
%   & \multicolumn{2}{c}{Verbal} & \multicolumn{2}{c}{Math} \\
%   \cmidrule{2-3} \cmidrule{4-5}
%   & Not Referred & Referred & Not Referred & Referred \\
%   \midrule
%   Reading z-score & 0.070 & 0.509 & 0.079 & 0.465 \\
%   & (0.003) & (0.039) & (0.003) & (0.040) \\
%   [1ex]
%   Math z-score & 0.079 & 0.452 & 0.087 & 0.590 \\
%   & (0.003) & (0.042) & (0.003) & (0.043) \\
%   [1ex]
%   GPA z-score & -0.066 & 0.705 & -0.069 & 0.711 \\
%   & (0.003) & (0.041) & (0.003) & (0.041) \\
%   [1ex]
%   Courses taken z-score & -0.153 & 2.690 & -0.184 & 2.488 \\
%   & (0.003) & (0.091) & (0.003) & (0.090) \\
%   [1ex]
%   Low-SES & 0.334 & 0.374 & 0.338 & 0.384 \\
%   & (0.001) & (0.019) & (0.001) & (0.019) \\
%   [1ex]
%   Med-SES & 0.515 & 0.513 & 0.513 & 0.507 \\
%   & (0.001) & (0.019) & (0.001) & (0.019) \\
%   [1ex]
%   High-SES & 0.151 & 0.113 & 0.149 & 0.109 \\
%   & (0.001) & (0.012) & (0.001) & (0.012) \\
%   % [1ex]
%   % Female & 0.524 & 0.562 & 0.520 & 0.428 \\
%   % & (0.001) & (0.019) & (0.001) & (0.019) \\
%   % [1ex]
%   % Age & 20.945 & 20.501 & 20.961 & 20.490 \\
%   % & (0.006) & (0.104) & (0.006) & (0.099) \\
%   \midrule
%   Observations & 128,174 & 673 & 127,481 & 669 \\
%   \bottomrule
%   \end{tabular*}
%   \begin{tablenotes}[flushleft]
%   \footnotesize
%   \item[] \textit{Note:} Standard errors in parentheses. GPA, test scores, and tie strength are standardized at the network level. For each referrer's network, we first calculated the mean and standard deviation of each measure. We then computed the average of these means and standard deviations across all referrers. Each individual's score was standardized using these network-level statistics. The standardization formula is $z = (x - \bar{x}_{network}) / \sigma_{network}$, where $\bar{x}_{network}$ and $\sigma_{network}$ are the average of network means and standard deviations, respectively. Low-SES, Med-SES, and High-SES are binary variables indicating the share of participants in estrato 1 and 2, 3 and 4, or 5 and 6, respectively. Tie strength measures the number of connections between individuals.
%   \end{tablenotes}
%   \label{tab:stats_by_nomination}
%   \end{threeparttable}
% \end{table}

\subsection{Ex post referral choice sets}

We now shed more light on the referral behavior after having characterized how referrals were made. Particularly interesting is that referrals go to peers with whom the median participant took 12 courses, with an average of 14. By restricting the networks for courses taken above the median, we can get a snapshot of how the referral choice set actually looked for participants before making referral decisions. As discussed in Section \ref{s:ses-diversity}, taking more courses with network members increases the share of same-SES individuals for both low- and high-SES students, and we had explored program selection as a potential mechanism. In Figure \ref{t15availability}, we show the effects of network segregation on \textit{ex post} referral choice sets for each SES group. Network compositions above the median number of courses taken reveal strong segregation effects: Low-SES networks contain 44.5\% low-SES peers, higher than the 35\% university-wide share by 9.5 percentage points. Conversely, high-SES students are under-represented in low-SES networks at only 8.6\% average share, compared to the 14\% population share ($-5.4$ pp.). At the other extreme, high-SES networks show the reverse pattern with average low-SES share dropping to just 15.7\%, a 19.3 percentage point decrease relative to the university average. High-SES students have a same-SES concentration at 26.5\%, doubling their 14\% population share ($+12.5$ pp.). Middle-SES networks remain relatively balanced and closely track population proportions across all SES groups. Taken together, these suggest observed referral rates of SES groups may follow the network compositions above median number of courses taken together. We will test this formally by setting up a choice model where we can take into account individual differences in network compositions across SES, and try to identify SES biases that go beyond SES groups' availability in the choice sets.

\begin{figure}[H]
  \centering
  \caption{Network size and courses taken together by courses taken}\label{t15availability}
  \includegraphics[width=0.6\linewidth]{../figures/t15availability.png}
  \begin{tablenotes}
    \footnotesize
    \item[] \textit{Note:} This figure compares the network shares of SES groups in the networks of low-, middle-, and high-SES participants above the median number of courses taken together with peers. Horizontal lines plot the university-wide shares of each SES group (Low: 35\%, Mid: 51\%, High: 14\%). Low- and high-SES networks both become same-SES dominated at the expense of each other while middle-SES networks remain balanced. Error bars represent 95\% confidence intervals.
  \end{tablenotes}
\end{figure}


\subsection{Identifying the SES bias in referrals}

To formally test for SES bias beyond network composition, we employ a choice modeling approach. We model a single referral outcome from mutually exclusive candidates, where our dependent variable outcome is multinomial distributed. Our design leverages the enrollment network to generate a dataset which includes alternative-specific variables for each referral decision, i.e., SES, courses taken together with the participant making the referral, as well as entry exam scores for not just the chosen alternative but all referral candidates. Using a conditional logit model on these data, we can identify whether an SES group has an aggregate bias controlling for each individual's unique enrollment network composition.

We follow an additive random utility model framework where individual $i$ and alternative $j$ have utility $U_{ij}$ that is the sum of a deterministic component, $V_{ij}$, that depends on regressors and unknown parameters, and an unobserved random component $\varepsilon_{ij}$:

We observe the outcome $y_i = j$ if alternative $j$ has the highest utility of the alternatives. The probability that the outcome for individual $i$ is alternative $j$, conditional on the regressors, is:

\begin{align}
p_{ij} = \Pr(y_i = j) = \Pr(U_{ij} \geq U_{ik}), \quad \text{for all } k
\end{align}

The conditional logit model specifies that the probability of individual $i$ choosing alternative $j$ from choice set $C_i$ is given by:

\begin{align}
p_{ij} = \frac{\exp(x'_{ij} \beta)}{\sum_{l \in C_i} \exp(x'_{il} \beta)}, \quad j \in C_i
\end{align}

where $x_{ij}$ are alternative-specific regressors, i.e., characteristics of potential referral candidates that vary across alternatives. In our context, individual $i$ chooses to refer candidate $j$ from their enrollment network $C_i$. The alternative-specific regressors include SES and entry exam scores of the referral candidate, and the number of courses taken together with the participant making the referral. Conditional logit structure eliminates participant-specific factors that might influence both network formation and referral decisions, allowing us to identify preferences within each participant's realized network. 

For causal identification of SES bias, we require two identifying assumptions. Specifically:

\begin{enumerate}
    \item \textbf{Conditional exogeneity.} SES and the number of courses taken together could be endogenous due to program selection. High-SES students sort into expensive programs while low-SES students choose affordable programs, creating systematic SES variation across enrollment networks. Similarly, the number of courses taken together reflects program selection decisions that may correlate with unobserved referral preferences. However, conditional on the realized enrollment network, the remaining variation in both SES and the number of courses taken together across referral candidates must be independent of unobserved factors affecting referral decisions. In the robustness checks, we show that being in the same program with the referrer does not impact our SES bias estimates, although it reduces the coefficient on the number of courses taken together.
    
    \item \textbf{Complete choice sets and independence of irrelevant alternatives.} Administrative data captures the complete enrollment network, with all peers who took at least one course with individual $i$ and represent the true choice set for referral decisions (unless participants have potential referral candidates with whom they never took classes). The independence of irrelevant alternatives (IIA) assumption requires that choices between any two alternatives be independent of other options in the choice set, which could be problematic if, e.g., peers within the same SES group are viewed as close substitutes. This concern does not apply to our setting because the design of our experiment ensures that choice sets are fixed by enrollment rather than arbitrary inclusion/exclusion of alternatives that create IIA violations.
\end{enumerate}

Under these assumptions, the conditional logit framework controls for individual heterogeneity in program selection (absorbed by conditioning on choice sets), selection into programs based on observable characteristics (through alternative-specific variables), and choice set composition effects (through the multinomial structure). Therefore, $\beta$ should identify the causal effect of referral candidate SES on referral probability, holding constant the number of courses taken together and the entry exam scores of candidates. A significant coefficient will then indicate taste-based discrimination.

We pool participants by their SES group, and estimate the above described conditional fixed effects logit model once for low-, middle-, and high-SES referrers. We standardize entry exam scores and the number of courses taken together at the individual network level. For each referrer's network, we first calculate the mean and standard deviation for both measures. We then compute the average of these means and standard deviations across all 734 referrers. Each referral candidate's entry exam score and the number of courses they have taken with the referrer is standardized using these network-level statistics. The standardization formula is $z_{i} = (x - \bar{X}_{i}) / \sigma_{i}$, where $\bar{X}_{i}$ and $\sigma_{i}$ are the average of network means and standard deviations for $C_{i}$.

We now present our empirical findings and describe our first set of findings in Table \ref{tab:ses-heterogeneity}. To begin with, the variance explained by all three models are extremely low, suggesting the role of potential SES biases in referrals that go beyond the network structure must be limited. Regardless, controlling for network composition, low-SES participants are more likely to refer other low-SES, and are less likely to refer high-SES relative to the probability of referring middle-SES peers. In contrast, we find that high-SES participants are less likely to refer other low-SES, relative to the probability of referring middle-SES peers.

\begin{table}[H]
\centering
\begin{threeparttable}
\caption{SES bias in referral decisions by referrer SES group}\label{tab:ses-heterogeneity}
\begin{tabular*}{0.85\textwidth}{@{\extracolsep{\fill}}l c c c@{}}
\toprule
& \multicolumn{3}{c}{Referrer SES} \\
& Low & Middle & High \\
& (1) & (2) & (3) \\
\midrule
Low-SES candidate & 0.453*** & -0.019 & -0.710** \\
& (0.109) & (0.098) & (0.333) \\
High-SES candidate & -0.584*** & -0.255* & 0.001 \\
& (0.211) & (0.145) & (0.261) \\
\midrule
$\chi^2$ & 33.47 & 3.18 & 4.94 \\
Observations & 110,142 & 127,088 & 19,767 \\
Individuals & 301 & 366 & 67 \\
\bottomrule
\end{tabular*}
\begin{tablenotes}[flushleft]
\footnotesize
\item[] \textit{Note:} Individual-level clustered standard errors in parentheses. * $p < 0.1$, ** $p < 0.05$, *** $p < 0.01$. Each column represents a separate conditional logit regression estimated on the subsample of referrers from the indicated SES group. Coefficients represent log-odds of referring candidates from the specified SES group relative to referring middle-SES candidates. All models include individual fixed effects that control for each referrer's choice set composition.
\end{tablenotes}
\end{threeparttable}
\end{table}

Next, we include social proximity controls in our analysis. We proceed by adding the standardized number of courses taken together as a control in our specification and describe the results in Table \ref{tab:ses-heterogeneity-extended}. A one standard deviation increase in the number of courses taken together proves to be highly significant across all models, with coefficients ranging from 0.856 to 1.049, indicating that stronger social connections substantially increase the probability of referral. The high $\chi^2$ statistics suggest that these models explain considerably more variance than specifications without this control, highlighting the importance of courses taken together in referral decisions. Nevertheless, low-SES participants still show a strong same-SES bias relative to referring middle-SES peers at the average number of courses taken together. This same-SES bias is not observed among middle-SES or high-SES referrers, who also display no statistically significant bias toward low-SES candidates. No referrer group shows a positive bias for high-SES candidates relative to middle-SES candidates. 

\begin{table}[H]
\centering
\begin{threeparttable}
\caption{SES bias in referral decisions by referrer SES group}\label{tab:ses-heterogeneity-extended}
\begin{tabular*}{0.85\textwidth}{@{\extracolsep{\fill}}l c c c@{}}
\toprule
& \multicolumn{3}{c}{Referrer SES} \\
& Low & Middle & High \\
& (1) & (2) & (3) \\
\midrule
Low-SES candidate & 0.348*** & -0.064 & -0.489 \\
& (0.123) & (0.115) & (0.337) \\
High-SES candidate & -0.366 & -0.165 & -0.140 \\
& (0.223) & (0.157) & (0.286) \\
Courses taken (z-score) & 0.856*** & 0.931*** & 1.049*** \\
& (0.035) & (0.037) & (0.126) \\
\midrule
$\chi^2$ & 626.15 & 636.10 & 71.43 \\
Observations & 110,142 & 127,088 & 19,767 \\
Individuals & 301 & 366 & 67 \\
\bottomrule
\end{tabular*}
\begin{tablenotes}[flushleft]
\footnotesize
\item[] \textit{Note:} Individual-level clustered standard errors in parentheses. * $p < 0.1$, ** $p < 0.05$, *** $p < 0.01$. Each column represents a separate conditional logit regression estimated on the subsample of referrers from the indicated SES group. Coefficients represent log-odds of referring candidates from the specified SES group relative to referring middle-SES candidates. All models include individual fixed effects that control for each referrer's choice set composition.
\end{tablenotes}
\end{threeparttable}
\end{table}

We add standardized entry exam scores (Math and Reading average) as a second control variable and describe our results in Table \ref{tab:ses-heterogeneity-scores}. A one standard deviation increase in the entry exam score proves highly significant across all models, with coefficients ranging from 0.587 to 0.883. This shows merit-based considerations due to the incentive structure of the experiment remained central to referral decisions. The slightly higher $\chi^2$ statistics compared to the earlier specification suggests that entry exam scores improve model fit. The inclusion of standardized entry exam scores strengthens SES biases. Low-SES referrers maintain their same-SES bias, with now a significant negative bias against high-SES. Middle-SES referrers, previously showing no SES bias, now show marginal negative bias against high-SES. Finally, high-SES referrers exhibit marginal negative bias against low-SES candidates. 

The evidence of a bias becoming significant when controlling for entry exam scores has a nuanced interpretation. While at the university-level, low-SES typically score lower in the entry exam, low-SES students appearing in high-SES networks are positively selected, scoring about 0.14 standard deviations higher than middle-SES students (see Appendix Table \ref{tab:network_connections}). Controlling for performance thus removes this positive selection and reveals the ``pure'' SES bias that was previously underestimated by above average performance of low-SES. Vice versa, high-SES in low-SES networks perform 0.12 standard deviations better than middle-SES students. The same bias was underestimated as high-SES candidates' better performance relative to middle-SES in the same networks provided a meritocratic justification for getting more referrals. Controlling for exam scores reveal that both high- and low-SES referrers have negative SES bias towards one another that operates independently of - and counter to - performance-based considerations. What makes interpretation difficult is that while biased against low-SES, high-SES referrers do not under any specification display a positive bias towards their in-group. For this final reason, we do not dig any further in this direction.

To conclude, we conduct joint significance tests, testing whether low- and high-SES regression coefficients are jointly different from middle-SES for each regression specification. For low-SES referrers, the joint test remains highly significant across all three specifications ($\chi^2 = 10.20$, $p = 0.006$ in the final model), indicating persistent SES bias across all specifications. In contrast, middle-SES referrers display no significant joint SES bias in any specification, with the test becoming increasingly non-significant as controls are added ($\chi^2 = 4.13$, $p = 0.127$ in the final model). High-SES referrers similarly show no significant joint SES bias across all three models ($\chi^2 = 4.28$, $p = 0.118$ in the final model). These results suggest that SES bias in referrals is primarily driven by low-SES. There is no sufficient evidence to conclude that middle- and high-SES referrers systematically discriminate against other-SES peers once we take into account the large differences in their network compositions due to program selection.


\begin{table}[H]
\centering
\begin{threeparttable}
\caption{SES bias in referral decisions by referrer SES group with academic performance controls}\label{tab:ses-heterogeneity-scores}
\begin{tabular*}{0.85\textwidth}{@{\extracolsep{\fill}}l c c c@{}}
\toprule
& \multicolumn{3}{c}{Referrer SES} \\
& Low & Middle & High \\
& (1) & (2) & (3) \\
\midrule
Low-SES candidate & 0.242** & -0.159 & -0.600* \\
& (0.123) & (0.114) & (0.327) \\
High-SES candidate& -0.445** & -0.274* & -0.345 \\
& (0.222) & (0.157) & (0.287) \\
Courses taken (z-score) & 0.859*** & 0.948*** & 1.043*** \\
& (0.036) & (0.038) & (0.118) \\
Entry exam (candidate z-score) & 0.607*** & 0.587*** & 0.883*** \\
& (0.052) & (0.047) & (0.111) \\
\midrule
$\chi^2$ & 789.87 & 756.06 & 120.54 \\
Observations & 110,142 & 127,088 & 19,767 \\
Individuals & 301 & 366 & 67 \\
\bottomrule
\end{tabular*}
\begin{tablenotes}[flushleft]
\footnotesize
\item[] \textit{Note:} Individual-level clustered standard errors in parentheses. * $p < 0.1$, ** $p < 0.05$, *** $p < 0.01$. Each column represents a separate conditional logit regression estimated on the subsample of referrers from the indicated SES group. Coefficients represent log-odds of referring candidates from the specified SES group relative to referring middle-SES candidates. All models include individual fixed effects that control for each referrer's choice set composition.
\end{tablenotes}
\end{threeparttable}
\end{table}

\section{Robustness check} \label{s6}

Does the number of courses taken together have an independent effect that goes beyond identifying peers in the same academic program? To evaluate this question we leverage our administrative data, and identify peers within the same program: In each individual network we observe the participant-specific academic program for the participant making the referral and alternative-specific academic program for each referral candidate. We add this new variable in our specification and describe our findings in Table \ref{tab:ses-heterogeneity-program}. Being in the same academic program has a substantial positive effect on referral likelihood, with coefficients ranging from 1.257 to 2.198 across all referrer SES groups. This confirms that program affiliation serves as a strong predictor of referral decisions, reflecting increased familiarity. Our comparison of interest is the point estimate for the standardized number of courses taken. Across all three referrer groups, the standardized number of courses taken together maintains its statistical significance after controlling for same program membership. The coefficient magnitudes are expectedly smaller compared to specifications without program controls (ranging from 0.688 to 0.930) as the newly added variable is a moderator: Matching academic programs leads to taking more courses together. The remaining estimates in our model remain robust to the inclusion of the same-program variable with little change in point estimates. The persistence of statistical significance (all $p < 0.001$) suggests that the number of courses taken together has an independent effect on referral decisions. To sum, our measure of tie strength seems to capture meaningful social interaction patterns that lead to referrals, and go beyond simply identifying matching academic programs. 

\begin{table}[H]
\centering
\begin{threeparttable}
\caption{SES bias in referral decisions by referrer SES group with program controls}\label{tab:ses-heterogeneity-program}
\begin{tabular*}{0.85\textwidth}{@{\extracolsep{\fill}}l c c c@{}}
\toprule
& \multicolumn{3}{c}{Referrer SES} \\
& Low & Middle & High \\
& (1) & (2) & (3) \\
\midrule
Low-SES candidate & 0.236** & -0.140 & -0.567* \\
& (0.119) & (0.111) & (0.331) \\
High-SES candidate& -0.421* & -0.249 & -0.383 \\
& (0.220) & (0.158) & (0.281) \\
Entry exam (candidate z-score) & 0.623*** & 0.590*** & 0.892*** \\
& (0.054) & (0.048) & (0.114) \\
Courses taken (z-score) & 0.688*** & 0.760*** & 0.930*** \\
& (0.032) & (0.035) & (0.119) \\
Same program & 2.074*** & 2.198*** & 1.257*** \\
& (0.215) & (0.185) & (0.467) \\
\midrule
$\chi^2$ & 865.35 & 981.99 & 135.47 \\
Observations & 110,142 & 127,088 & 19,767 \\
Individuals & 301 & 366 & 67 \\
\bottomrule
\end{tabular*}
\begin{tablenotes}[flushleft]
\footnotesize
\item[] \textit{Note:} Individual-level clustered standard errors in parentheses. * $p < 0.1$, ** $p < 0.05$, *** $p < 0.01$. Each column represents a separate conditional logit regression estimated on the subsample of referrers from the indicated SES group. Coefficients represent log-odds of referring candidates from the specified SES group relative to referring middle-SES candidates. All models include individual fixed effects that control for each referrer's choice set composition.
\end{tablenotes}
\end{threeparttable}
\end{table}

\pagebreak

\section{Conclusion} \label{s7}
In this paper, we study whether SES groups are biased toward one another beyond what is attributable to differences in their networks, and the effects of different incentive structures on referral behavior. Through a lab-in-the-field experiment that leverages enrollment networks at a socially diverse university, we find that the SES biases in referrals originate mostly from network structures, and referrals under performance-pay incentives do not exacerbate existing SES inequalities.

Our findings reveal that enrollment networks are surprisingly segregated and referrals from these networks reflect closely the choice sets of the referrers. We identify program selection as the key mechanism driving this segregation. Low-SES students select into more affordable programs, and program selection plays a major part in segregating SES groups where low- and high-SES take more courses with their own SES group. Consequently, referrals come almost exclusively from the same academic program as the referrer, limiting the SES-diversity of their choice sets. Regardless of the bonus for the referral candidate, participants also pick higher performing peers with whom they have taken many courses together. We find that only low-SES referrers exhibit a same-SES bias. These findings suggest that the underlying network structure plays a crucial role in referrals, where institutional action can remedy the network segregation. 

These results complement the broader literature where much of the bias in referrals can be attributable to the ``practical'' choice sets of the referrers. While previous work demonstrates that about half of referrals come from a smaller, elicited network of close friends \citep{hederos2025gender}, we go the other way and use administrative data to construct a complete network which presumably includes close social relationships at the institutional level. Having access to the complete network thus eliminates any potential for under or overestimating taste-based biases \citep{griffith2022jle}. Under performance-pay incentives, referrers identify productive others regardless of additional financial rewards for the referral candidate. Still, the lack of a treatment effect suggests that in both incentive structures referrers pick close ties, shifting the responsibility to institutional actors to create diverse environments where cross-SES social interaction can take place more frequently and allow more diversity in networks. 

These findings have policy implications. Looking forward, institutions can play a crucial role in achieving SES equality of opportunity in higher education. Universities are already a setting in which low-SES get exposed to typically a higher than population share of higher-SES individuals than at other settings \citep{chetty2022n}. Yet, segregation within the higher education institutions remain a source for SES inequality. If low-SES peers never get to interact in meaningful ways with higher-SES, e.g., by taking courses together, the premise of social mobility through social channels remains severely underexploited. Future studies should work on ways to reduce SES segregation in collaboration with institutions, where having access to complete enrollment networks in addition to the typical friendship elicitation methods could help identifying the exact overlap between the two distinct approaches.

\pagebreak

\bibliographystyle{apacite}

\bibliography{referrals}    
\pagebreak

\appendix \label{appendix-all}
\renewcommand{\thefigure}{A.\arabic{figure}}
\setcounter{figure}{0}

\section{Additional Figures and Tables}

\subsection*{Additional Figures}
\pagebreak
\subsection*{Additional Tables}
\renewcommand{\thetable}{A.\arabic{table}}
\setcounter{table}{0}
\setcounter{figure}{0}

\begin{table}[H]
  \centering
  \begin{threeparttable}
  \caption{Selection into the experiment}
  \begin{tabular*}{\textwidth}{@{\extracolsep{\fill}}lcc c@{}}
  \toprule
  & \multicolumn{1}{c}{\textbf{University}} & \multicolumn{1}{c}{\textbf{Sample}} & \multicolumn{1}{c}{\textit{\textbf{p}}} \\
  \midrule
  Reading score & 62.651 & 65.183 & $<0.001$ \\
  Math score & 63.973 & 67.477 & $<0.001$ \\
  GPA & 3.958 & 4.012 & $<0.001$ \\
  Low-SES & 0.343 & 0.410 & $<0.001$ \\
  Middle-SES & 0.505 & 0.499 & 0.763 \\
  High-SES & 0.153 & 0.091 & $<0.001$ \\
  Female & 0.567 & 0.530 & $<0.001$ \\
  Age & 21.154 & 20.651 & $<0.001$ \\
  \midrule
  Observations & 4,417 & 734 &  \\
  \bottomrule
  \end{tabular*}
  \begin{tablenotes}[flushleft]
  \footnotesize
  \item[] \textit{Note:} This table compares characteristics between the university and the experimental sample. $p$-values for binary outcomes (Low-SES, Med-SES, High-SES, Female) are from two-sample tests of proportions; for continuous variables, from two-sample $t$-tests with unequal variances. All reported \textit{p}-values are two-tailed.
  \end{tablenotes}
  \label{tab:selection}
  \end{threeparttable}
\end{table}


\begin{table}[H]
  \centering
  \begin{threeparttable}
  \caption{Distribution of referrals by area}
  \begin{tabular*}{\textwidth}{@{\extracolsep{\fill}}lccc}
  \toprule
  \textbf{Area} & \textbf{Only one area} & \textbf{Both areas} & \textbf{Total} \\
  \midrule
  Verbal & 65 & 608 & 673 \\
  Math & 61 & 608 & 669 \\
  \midrule
  Total & 126 & 1,216 & 1,342 \\
  \bottomrule
  \end{tabular*}
  \begin{tablenotes}[flushleft]
  \footnotesize
  \item[] \textit{Note:} The table shows how many referrers made referrals in only one area versus both areas. ``Only one area'' indicates individuals who made referrals exclusively for one area of the exam. ``Both areas'' shows individuals who made referrals in both verbal and math areas. The majority of referrers (608) made referrals in both areas.
  \end{tablenotes}
  \label{tab:referral_distribution}
  \end{threeparttable}
\end{table}


\begin{table}[H]
  \centering
  \begin{threeparttable}
  \caption{Referral characteristics by exam area (unique referrals only)}
  \begin{tabular*}{\textwidth}{@{\extracolsep{\fill}}lccc}
  \toprule
  & \multicolumn{1}{c}{\textbf{Reading}} & \multicolumn{1}{c}{\textbf{Math}} & \multicolumn{1}{c}{\textbf{\textit{p}}} \\
  \midrule
  Reading score & 67.733 & 67.126 & 0.252 \\
  Math score & 69.339 & 71.151 & 0.008 \\
  GPA & 4.136 & 4.136 & 0.987 \\
  Courses taken & 13.916 & 13.019 & 0.123 \\
  Low-SES & 0.372 & 0.385 & 0.666 \\
  Med-SES & 0.526 & 0.518 & 0.801 \\
  High-SES & 0.103 & 0.097 & 0.781 \\
  \midrule
  Observations & 487 & 483 &  \\
  \bottomrule
  \end{tabular*}
  \begin{tablenotes}[flushleft]
  \footnotesize
  \item[] \textit{Note:} This table compares characteristics of uniquely referred students by entry exam area for the referral (verbal vs. math). $p$-values are from two-sample t-tests with unequal variances. Referrals in Math area go to peers with significantly higher math scores ($p = 0.008$), while we find no significant differences for Reading scores, GPA, courses taken, or SES composition for referrals across the two areas. Excluding referrals going to the same individuals does not change the outcomes for referrals compared to Appendix Table \ref{tab:referral_area}
  \end{tablenotes}
  \label{tab:referral_by_area}
  \end{threeparttable}
\end{table}

\begin{table}[H]
  \centering
  \begin{threeparttable}
  \caption{Referral characteristics by academic area}
  \begin{tabular*}{\textwidth}{@{\extracolsep{\fill}}lcc c@{}}
  \toprule
   & \multicolumn{1}{c}{\textbf{Reading}} & \multicolumn{1}{c}{\textbf{Math}} & \multicolumn{1}{c}{\textit{\textbf{p}}} \\ 
  \midrule
  Reading score & 67.85 & 67.41 & 0.348 \\
  Math score & 70.04 & 71.36 & 0.029 \\
  GPA & 4.153 & 4.153 & 0.984 \\ 
  Courses taken & 14.467 & 13.822 & 0.206 \\
  Low-SES & 37\% & 38\% & 0.714 \\
  Middle-SES & 51\% & 51\% & 0.829 \\
  High-SES & 11\% & 11\% & 0.824 \\
  \midrule
  Observations & 673 & 669 &  \\
  \bottomrule
  \end{tabular*}
  \begin{tablenotes}[flushleft]
  \footnotesize
  \item[] \textit{Note:} This table compares characteristics of referred students by entry exam area for the referral (verbal vs. math). $p$-values are from two-sample t-tests with unequal variances. Referrals in Math area go to peers with significantly higher math scores ($p = 0.029$), while we find no significant differences for Reading scores, GPA, courses taken, or SES composition for referrals across the two areas.
  \end{tablenotes}
  \label{tab:referral_area}
  \end{threeparttable}
  \end{table}

\begin{table}[H]
  \centering
  \begin{threeparttable}
  \caption{Average entry exam z-scores by SES network connections}
  \begin{tabular*}{\textwidth}{@{\extracolsep{\fill}}lccc@{}}
  \toprule
   & \multicolumn{3}{c}{\textbf{Network average for SES group}} \\
  \cmidrule(lr){2-4}
  \textbf{Referrer SES} & \textbf{Low} & \textbf{Middle} & \textbf{High} \\ 
  \midrule
  Low & 0.086 & -0.018 & 0.144 \\
  Middle & 0.186 & 0.023 & 0.215 \\
  High & 0.204 & 0.064 & 0.285 \\
  \midrule
  All & -0.361 & -0.078 & 0.169 \\
  \bottomrule
  \end{tabular*}
  \begin{tablenotes}[flushleft]
  \footnotesize
  \item[] \textit{Note:} This table shows average (Math and Reading) standardized entry exam scores for individuals of different SES levels (rows) when connected to peers of specific SES levels (columns). The ``All'' row shows the overall average scores across all participant SES levels when connected to each network SES type. Higher values indicate better academic performance in SD's.
  \end{tablenotes}
  \label{tab:network_connections}
  \end{threeparttable}
\end{table}

\pagebreak

\section{Experiment} \label{instructions}
\renewcommand{\thefigure}{B.\arabic{figure}}
\textit{We include the English version of the instructions used in Qualtrics. Participansts saw the Spanish version. Horizontal lines in the text indicate page breaks and clarifiying comments are inside brackets.} 

\section*{Consent}
You have been invited to participate in this decision-making study. This study is directed by [omitted for anonymous review] and organized with the support of the Social Bee Lab (Social Behavior and Experimental Economics Laboratory) at UNAB.

\vspace{5mm}

\noindent In this study, we will pay \textbf{one (1)} out of every \textbf{ten (10)} participants, who will be randomly selected. Each selected person will receive a fixed payment of \textbf{70,000} (seventy thousand pesos) for completing the study. Additionally, they can earn up to \textbf{270,000} (two hundred and seventy thousand pesos), depending on their decisions. So, in total, if you are selected to receive payment, you can earn up to \textbf{340,000} (three hundred and forty thousand pesos) for completing this study.

\vspace{5mm}

\noindent If you are selected, you can claim your payment at any Banco de Bogotá office by presenting your ID. Your participation in this study is voluntary and you can leave the study at any time. If you withdraw before completing the study, you will not receive any payment.

\vspace{5mm}

\noindent The estimated duration of this study is 20 minutes.

\vspace{5mm}

\noindent  The purpose of this study is to understand how people make decisions. For this, we will use administrative information from the university such as the SABER 11 test scores of various students (including you). Your responses will not be shared with anyone and your participation will not affect your academic records. To maintain strict confidentiality, the research results will not be associated at any time with information that could personally identify you.

\vspace{5mm}

\noindent There are no risks associated with your participation in this study beyond everyday risks. However, if you wish to report any problems, you can contact Professor [omitted for anonymous review]. For questions related to your rights as a research study participant, you can contact the IRB office of [omitted for anonymous review].

\vspace{5mm}

\noindent By selecting the option ``I want to participate in the study" below, you give your consent to participate in this study and allow us to compare your responses with some administrative records from the university.

\begin{itemize}
  \item I want to participate in the study [advances to next page]
  \item I do not want to participate in the study
\end{itemize}

\noindent\rule{\textwidth}{1pt}

\section*{Student Information}

Please write your student code.
In case you are enrolled in more than one program simultaneously, write the code of the first program you entered:

\vspace{5mm}

\noindent[Student ID code]

\vspace{5mm}

\noindent What semester are you currently in?

\vspace{5mm}

\noindent[Slider ranging from 1 to 11]

\noindent\rule{\textwidth}{1pt}

\vspace{5mm}

\noindent[Random assignment to treatment or control]

\section*{Instructions}

The instructions for this study are presented in the following video. Please watch it carefully. We will explain your participation and how earnings are determined if you are selected to receive payment.

\vspace{5mm}

\noindent[Treatment-specific instructions in video format]

\vspace{5mm}

\noindent If you want to read the text of the instructions narrated in the video, press the ``Read instruction text" button. Also know that in each question, there will be a button with information that will remind you if that question has earnings and how it is calculated, in case you have any doubts.

\begin{itemize}
  \item I want to read the instructions text [text version below]
\end{itemize}

\noindent\rule{\textwidth}{1pt}

\vspace{5mm}

\noindent In this study, you will respond to three types of questions. First, are the belief questions. For belief questions, we will use as reference the results of the SABER 11 test that you and other students took to enter the university, focused on three areas of the exam: mathematics, reading, and English.

\vspace{5mm}

\noindent For each area, we will take the scores of all university students and order them from lowest to highest. We will then group them into 100 percentiles. The percentile is a position measure that indicates the percentage of students with an exam score that is above or below a value.

\vspace{5mm}

\noindent For example, if your score in mathematics is in the 20th percentile, it means that 20 percent of university students have a score lower than yours and the remaining 80 percent have a higher score. A sample belief question is: ``compared to university students, in what percentile is your score for mathematics?"

\vspace{5mm}

\noindent If your answer is correct, you can earn 20 thousand pesos. We say your answer is correct if the difference between the percentile you suggest and the actual percentile of your score is not greater than 7 units. For example, if you have a score that is in the 33rd percentile and you say it is in the 38th, the answer is correct because the difference is less than 7. But if you answer that it is in the 41st, the difference is greater than 7 and the answer is incorrect.

\vspace{5mm}

\noindent The second type of questions are recommendation questions and are also based on the mathematics, reading, and English areas of the SABER 11 test. We will ask you to think about the students with whom you have taken or are taking classes, to recommend from among them the person you consider best at solving problems similar to those on the SABER 11 test.

\vspace{5mm}

\noindent When you start typing the name of your recommended person, the computer will show suggestions with the full name, program, and university entry year of different students. Choose the person you want to recommend. If the name doesn't appear, check that you are writing it correctly. Do not use accents and use `n' instead of `ñ'. If it still doesn't appear, it may be because that person is not enrolled this semester or because they did not take the SABER 11 test. In that case, recommend someone else.

\vspace{5mm}

\noindent You can earn up to 250,000 pesos for your recommendation. We will multiply your recommended person's score by 100 pesos if they are in the first 50 percentiles. We will multiply it by 500 pesos if your recommended person's score is between the 51st and 65th percentile. If it is between the 66th and 80th percentile, we will multiply your recommended person's score by 1000 pesos. If the score is between the 81st and 90th percentile, you earn 1500 pesos multiplied by your recommended person's score. And if the score is between the 91st and 100th percentile, we will multiply your recommended person's score by 2500 pesos to determine the earnings.

\vspace{5mm}

\noindent The third type of questions are information questions and focus on aspects of your personal life or your relationship with the people you have recommended.


\subsection*{Earnings}

Now we will explain who gets paid for participating and how the earnings for this study are assigned. The computer will randomly select one out of every 10 participants to pay for their responses. For selected individuals, the computer will randomly choose one of the three areas, and from that chosen area, it will pay for one of the belief questions.

\vspace{5mm}

\noindent Similarly, the computer will randomly select one of the three areas to pay for one of the recommendation questions.

\vspace{5mm}

\noindent \textbf{Additionally, if you are selected to receive payment, your recommended person in the  chosen area will receive a fixed payment of 100 thousand pesos.} [Only seen if assigned to the treatment] 

\vspace{5mm}

\noindent Each person selected to receive payment for this study can earn: up to 20 thousand pesos for one of the belief questions, up to 250 thousand pesos for one of the recommendation questions, and a fixed payment of 70 thousand pesos for completing the study.

\vspace{5mm}

\noindent Selected individuals can earn up to 340 thousand pesos.

\noindent\rule{\textwidth}{1pt}

\vspace{5mm}

\noindent [Participants go through all three Subject Areas in randomized order]

\section*{Subject Areas}

\subsection*{Critical Reading}
For this section, we will use as reference the Critical Reading test from SABER 11, which evaluates the necessary competencies to understand, interpret, and evaluate texts that can be found in everyday life and in non-specialized academic fields.

\vspace{5mm} 

\noindent [Clicking shows the example question from SABER 11 below]

\vspace{5mm} 

\noindent Although the democratic political tradition dates back to ancient Greece, political thinkers did not address the democratic cause until the 19th century. Until then, democracy had been rejected as the government of the ignorant and unenlightened masses. Today it seems that we have all become democrats without having solid arguments in favor. Liberals, conservatives, socialists, communists, anarchists, and even fascists have rushed to proclaim the virtues of democracy and to show their democratic credentials (Andrew Heywood). According to the text, which political positions identify themselves as democratic?

\begin{itemize}
  \item Only political positions that are not extremist
  \item The most recent political positions historically
  \item The majority of existing political positions
  \item The totality of possible political currents
\end{itemize}

\vspace{5mm}

\noindent\rule{\textwidth}{1pt}


\subsection*{Mathematics}
This section references the Mathematics test from SABER 11, which evaluates people's competencies to face situations that can be resolved using certain mathematical tools.

\vspace{5mm} 

\noindent [Clicking shows the example question from SABER 11 below]

\vspace{5mm} 

\noindent A person living in Colombia has investments in dollars in the United States and knows that the exchange rate of the dollar against the Colombian peso will remain constant this month, with 1 dollar equivalent to 2,000 Colombian pesos. Their investment, in dollars, will yield profits of 3\% in the same period. A friend assures them that their profits in pesos will also be 3\%. Their friend's statement is:

\begin{itemize}
    \item Correct. The proportion in which the investment increases in dollars is the same as in pesos.
    \item Incorrect. The exact value of the investment should be known.
    \item Correct. 3\% is a fixed proportion in either currency.
    \item Incorrect. 3\% is a larger increase in Colombian pesos.
\end{itemize}

\vspace{5mm}

\noindent\rule{\textwidth}{1pt}


\subsection*{English}
This section uses the English test from SABER 11 as a reference, which evaluates that the person demonstrates their communicative abilities in reading and language use in this language.

\vspace{5mm} 

\noindent [Clicking shows the example question from SABER 11 below]

\vspace{5mm} 

\noindent Complete the conversations by marking the correct option.
    \begin{itemize}
        \item Conversation 1: I can't eat a cold sandwich. It is horrible!
        \begin{itemize}
            \item I hope so.
            \item I agree.
            \item I am not.
        \end{itemize}
        \item Conversation 2: It rained a lot last night!
        \begin{itemize}
            \item Did you accept?
            \item Did you understand?
            \item Did you sleep?
        \end{itemize}
    \end{itemize}

\noindent\rule{\textwidth}{1pt}

\vspace{5mm}

\noindent [Following parts are identical for all Subject Areas and are not repeated here for brevity]

\subsection*{Your Score}

Compared to university students, in which percentile do you think your [\textbf{Subject Area}] test score falls (1 is the lowest percentile and 100 the highest)?

\vspace{5mm} 

\noindent [Clicking shows the explanations below]

\vspace{5mm} 

\noindent How is a percentile calculated?

\vspace{5mm} 

\noindent A percentile is a position measurement. To calculate it, we take the test scores for all students currently enrolled in the university and order them from lowest to highest. The percentile value you choose refers to the percentage of students whose score is below yours. For example, if you choose the 20th percentile, you're indicating that 20\% of students have a score lower than yours and the remaining 80\% have a score higher than yours.

\vspace{5mm} 

\noindent What can I earn for this question?

\vspace{5mm} 

\noindent For your answer, you can earn \textbf{20,000 (twenty thousand) PESOS}, but only if the difference between your response and the correct percentile is less than 7. For example, if the percentile where your score falls is 33 and you respond with 38 (or 28), the difference is 5 and the answer is considered correct. But if you respond with 41 or more (or 25 or less), for example, the difference would be greater than 7 and the answer is incorrect.


\vspace{5mm} 

\noindent Please move the sphere to indicate which percentile you think your score falls in:

\vspace{5mm} 

\noindent[Slider with values from 0 to 100]


\vspace{5mm} 

\noindent\rule{\textwidth}{1pt}

\subsection*{Recommendation}

Among the people with whom you have taken any class at the university, who is your recommendation for the [\textbf{Subject Area}] test? Please write that person's name in the box below:

\vspace{5mm} 

\noindent \textbf{\textcolor{red}{Important:}} \textbf{You will not be considered for payment unless the recommended person is someone with whom you have taken at least one class during your studies.}


\vspace{5mm} 

\noindent Your response is only a recommendation for the purposes of this study and we will \textbf{not} contact your recommended person at any time.

\vspace{5mm} 

\noindent [Clicking shows the explanations below]

\vspace{5mm} 

\noindent Who can I recommend?

\vspace{5mm} 

\noindent Your recommendation \textbf{must} be someone with whom you have taken (or are taking) a class. If not, your answer will not be considered for payment. The person you recommend will not be contacted or receive any benefit from your recommendation.

\vspace{5mm} 

\noindent As you write, you will see up to 7 suggested student names containing the letters you have entered. The more you write, the more accurate the suggestions will be. Please write \textbf{without} accents and use the letter `n' instead of `ñ'. If the name of the person you're writing doesn't appear, it could be because you made an error while writing the name.

\vspace{5mm} 

\noindent If the name is correct and still doesn't appear, it could be because the student is not enrolled this semester or didn't take the SABER 11 test. In that case, you must recommend someone else.


\vspace{5mm} 

\noindent My earnings for this question?

\vspace{5mm} 

\noindent For your recommendation, you could receive earnings of up to 250,000 (two hundred and fifty thousand) PESOS. The earnings are calculated based on your recommendation's score and the percentile of that score compared to other UNAB students, as follows:

\begin{itemize}
  \item We will multiply your recommendation's score by \$100 (one hundred) pesos if it's between the 1st and 50th percentiles
  \item We will multiply your recommendation's score by \$500 (five hundred) pesos if it's between the 51st and 65th percentiles
  \item We will multiply your recommendation's score by \$1000 (one thousand) pesos if it's between the 66th and 80th percentiles
  \item We will multiply your recommendation's score by \$1500 (one thousand five hundred) pesos if it's between the 81st and 90th percentiles
  \item We will multiply your recommendation's score by \$2500 (two thousand five hundred) pesos if it's between the 91st and 100th percentiles
\end{itemize}


\vspace{5mm} 

\noindent This is illustrated in the image below:

\begin{figure}[H]
    \centering
    \includegraphics[width=0.65\textwidth]{/Users/reha.tuncer/Documents/GitHub/icfes-referrals/figures/bonus_structure.png}
    \caption{Earnings for recommendation questions}
    \label{fig:earnings}
\end{figure}


\vspace{5mm} 

\noindent For example, if your recommendation got 54 points and the score is in the 48th percentile, you could earn 54x100 = 5400 PESOS. But, if the same score of 54 points were in the 98th percentile, you could earn 54x2500 = 135,000 PESOS.

\vspace{5mm} 

\noindent [Text field with student name suggestions popping up as participant types]

\vspace{5mm} 

\noindent\rule{\textwidth}{1pt}

\subsection*{Relationship with your recommendation}
How close is your relationship with your recommendedation: ``[Name of the student selected from earlier]"? (0 indicates you are barely acquaintances and 10 means you are very close)

\vspace{5mm} 

\noindent [Slider with values from 0 to 10]

\vspace{5mm} 

\noindent\rule{\textwidth}{1pt}

\subsection*{Your recommendation's score}
Compared to university students, in which percentile do you think [Name of the student selected from earlier]'s score falls in the [\textbf{Subject Area}] test (1 is the lowest percentile and 100 the highest)?

\vspace{5mm} 

\noindent [Clicking shows the explanations below]

\vspace{5mm} 

\noindent How is a percentile calculated?

\vspace{5mm} 

\noindent A percentile is a position measurement. To calculate it, we take the test scores for all students currently enrolled in the university and order them from lowest to highest. The percentile value you choose refers to the percentage of students whose score is below yours. For example, if you choose the 20th percentile, you're indicating that 20\% of students have a score lower than yours and the remaining 80\% have a score higher than yours.

\vspace{5mm} 

\noindent What can I earn for this question?

\vspace{5mm} 

\noindent For your answer, you can earn \textbf{20,000 (twenty thousand) PESOS}, but only if the difference between your response and the correct percentile is less than 7. For example, if the percentile where your recommended person's score falls is 33 and you respond with 38 (or 28), the difference is 5 and the answer is considered correct. But if you respond with 41 or more (or 25 or less), for example, the difference would be greater than 7 and the answer is incorrect.

\vspace{5mm} 

\noindent  Please move the sphere to indicate which percentile you think your recommended person's score falls in:

\vspace{5mm} 

\noindent[Slider with values from 0 to 100]

\vspace{5mm} 

\noindent\rule{\textwidth}{1pt}

\section*{Demographic Information}

What is the highest level of education achieved by your father?

\vspace{5mm} 

\noindent [Primary, High School, University, Graduate Studies, Not Applicable]

\vspace{5mm} 

\noindent What is the highest level of education achieved by your mother?

\vspace{5mm} 

\noindent [Primary, High School, University, Graduate Studies, Not Applicable]

\vspace{5mm} 

\noindent Please indicate the socio-economic group to which your family belongs:

\vspace{5mm}

\noindent [Group A (Strata 1 or 2), Group B (Strata 3 or 4), Group C (Strata 5 or 6)]

\vspace{5mm}

\noindent\rule{\textwidth}{1pt}

\section*{UNAB Students Distribution}
Thinking about UNAB students, in your opinion, what percentage belongs to each socio-economic group? The total must sum to 100\%:

\vspace{5mm}

\noindent [Group A (Strata 1 or 2) percentage input area] 

\noindent [Group B (Strata 3 or 4) percentage input area] 

\noindent [Group C (Strata 5 or 6) percentage input area]

\noindent [Shows sum of above percentages]

\vspace{5mm} 

\noindent\rule{\textwidth}{1pt}

\section*{End of the Experiment}
Thank you for participating in this study.

\vspace{5mm}

\noindent If you are chosen to receive payment for your participation, you will receive a confirmation to your UNAB email and a link to fill out a form with your information. The process of processing payments is done through Nequi and takes approximately 15 business days, counted from the day of your participation.

\vspace{5mm}

\noindent [Clicking shows the explanations below]

\vspace{5mm}

\noindent Who gets paid and how is it decided?

\vspace{5mm}

\noindent The computer will randomly select one out of every ten participants in this study to be paid for their
decisions.

\vspace{5mm}

\noindent For selected individuals, the computer will randomly select one area: mathematics, reading, or English, and from that area will select one of the belief questions. If the answer to that question is correct, the participant will receive 20,000 pesos.

\vspace{5mm}

\noindent The computer will randomly select an area (mathematics, critical reading, or English) to pay for one of the recommendation questions. The area chosen for the recommendation question is independent of the area chosen for the belief question. The computer will take one of the two recommendations you have made for the chosen area. Depending on your recommendation's score, you could win up to 250,000 pesos.

\vspace{5mm}

\noindent Additionally, people selected to receive payment for their participation will have a fixed earnings of 70,000 pesos for completing the study.

\vspace{5mm}

\noindent\rule{\textwidth}{1pt}

\section*{Participation}
In the future, we will conduct studies similar to this one where people can earn money for their participation. The participation in these studies is by invitation only. Please indicate if you are interested in being invited to other studies similar to this one:

\vspace{5mm}

\noindent [Yes, No]


\end{document}
