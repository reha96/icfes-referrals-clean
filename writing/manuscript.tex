\documentclass[11pt,a4paper,oneside]{article}
\linespread{1.5}
\usepackage{amsmath,amsthm,amssymb,amsfonts,adjustbox,bm}
\usepackage{geometry}
\usepackage{threeparttable}
%\usepackage[dvipsnames]{xcolor}
\usepackage{mathtools}
\usepackage{refstyle}
\usepackage{pdflscape}
\usepackage{longtable}
\usepackage{eurosym}
\usepackage{enumerate}
\usepackage{booktabs}
\usepackage{siunitx}
\usepackage{rotating}
\usepackage{graphicx}
\usepackage{bbm}
\usepackage{subcaption}
\usepackage{caption}
\captionsetup{width=.8\textwidth}
\usepackage[section]{placeins}
\usepackage{xcolor}
\usepackage{geometry}
\usepackage{changepage}
\usepackage{float}
\usepackage[natbibapa]{apacite}
\usepackage{threeparttable}
\newcommand{\footremember}[2]{%
   \footnote{#2}
    \newcounter{#1}
    \setcounter{#1}{\value{footnote}}%
}
\newcommand{\footrecall}[1]{%
    \footnotemark[\value{#1}]%
} 
\usepackage{lineno}
\makeatletter
\def\makeLineNumberLeft{%
  \linenumberfont\llap{\hb@xt@\linenumberwidth{\LineNumber\hss}\hskip\linenumbersep}% left line number
  \hskip\columnwidth% skip over column of text
  \rlap{\hskip\linenumbersep\hb@xt@\linenumberwidth{\hss\LineNumber}}\hss}% right line number
\leftlinenumbers% Re-issue [left] option
\makeatother

\usepackage{blindtext}

\usepackage[colorlinks=true, citecolor=blue, linkcolor=blue, urlcolor=blue,breaklinks]{hyperref}


\title{Project ICFES:
Evidence from a referral field experiment\footnote{We obtained Institutional Review Board approvals from NYU Abu Dhabi (HRPP 2024-50) and the University of Luxembourg (ERP 24–028). The study design was preregistered in the OSF Registries prior to data collection (see \url{https://doi.org/10.17605/OSF.IO/V9T3W}).} }
% \\ \large Experimental Design \\
% \title{Skills, Inequality, and Referrals}
\author{Manuel Munoz\footremember{liser}{Luxembourg Institute of Socio-Economic Research}, Ernesto Reuben\footnote{Division of Social Science, New York University Abu Dhabi} \footrecall{liser}, Reha Tuncer\footnote{University of Luxembourg}}  
% \footnote{Center for Behavioral Institutional Design at New York University Abu Dhabi} 
\begin{document}
\linenumbers 
% HERE FOR LINES
\maketitle



\section*{Abstract}
Lorem Ipsum \citep{beaman_job_2018}
\medskip \\
\textbf{JEL Classification:} C93, D03, D83, J24 \\
\textbf{Keywords:} productivity beliefs, referrals, field experiment, skill identification, social class

% Problem: Labor markets increasingly value both cognitive and social skills, but firms struggle to identify these skills in candidates. Referrals are a common hiring tool, but their effectiveness across different skill types and potential social class biases remain unclear.
% Method: Using a field experiment with 849 university students, we study how well peers identify cognitive and social skills of their classmates.
% Key Results: Students effectively identify cognitive skills but struggle with social skills. Limited evidence of social class bias in referrals.
% Implications: Referrals may be more effective for identifying certain types of skills than others.
% We study how well referrals help screening for the two types of worker skills within the same network and compare the differential skill identification ability of referrers.

\pagebreak
\section{Introduction}
Equally qualified individuals in the labor market may face very different outcomes depending on their socioeconomic status (SES). A key driver of inequality according to  sociologists is due to differences in social capital (e.g., Bourdieu). A lack of social capital means a lack of access to individuals with influential (higher paid) jobs and job opportunities. In economic terms, it implies having worse outcomes when using one's network to find jobs conditional on the capacity on leveraging one's social network.\footnote{\textcolor{red}{fill in citaiton sociology reading from slides}}. 

Referral hiring, the formal or informal process where firms ask workers to recommend qualified candidates for job opportunities, is a vivid labor market application where the role of differences in social capital becomes evident. As referrals must orginate from the networks of referrers, the composition of referrer networks becomes a crucial channel that may propagate inequality: Similar individuals across socio-demographic characteristics tend to form connections at higher rates \citep{mcpherson2001birds}, making across SES (low-to-high) connections less likely than same-SES connections \citep{chetty2022social_a}.\footnote{Chetty and colleagues suggests the share of high-SES connections among low-SES networks as the definition of social capital, as it correlates so strongly with labor market income.} This implies referrals will reflect differences in network structures even in the absence of biases in the referral procedure, i.e., referring at random from one's network according to some productivity criteria.

Yet, in case of gender, evidence shows referrals may be biased under substantial pay-for-performance incentives beyond what is attribuable to differences in network compositions \citep{beaman_job_2018,hederos2025gender}. A similar bias against low-SES may further exacerbate outcomes of low-SES individuals in the labor market: If job information are in the hands of a select few high-SES which low-SES have already limited network access to (social capital hypothesis), and high-SES referrers are biased against low-SES, referring other high-SES at higher rates than their network composition, we can be confident that referral hiring disadvantages low-SES. The empirical question we answer is whether there is a bias against low-SES once we account for the network SES compositon in a controlled setting.

In this study, we focus on the role of SES in referrals by investigating whether individuals who are asked to refer someone tend to refer a candidate of their own SES. We also explore potential mechanisms behind referral patterns under different incentives. To this end, we conducted a lab-in-the-field experiment with 734 participants in a Colombian university where different SES groups mix together. Participants were instructed to refer a qualified student for tasks similar to the Math and Reading parts of the national university entry exam (akin to SAT). To incentivize participants to refer qualified candidates, we set earnings dependent on referred candidates' actual university entry exam scores.

Referral hiring in the labor market range form formal referral program at firm asking employees to refer to simply passing on job opportunities between network members \citep{topa2019social}. As participants in our study are students at the university referring based on exam scores, we abstract away from formal referral programs with defined job openings. Our setting instead resembles situations where contacts share opportunities with each other. This also eliminates reputational concerns as there is no firm, which play a largers role in the labor market compared to our setting. At the same time, national university entry exam scores are objective, widely accepted measures of ability, and referrers in our setting possess accurate information about these signals. 





\section{Background and Setting} \label{s2}





\section{Design} \label{s3}


\section{Results} \label{s5}

\subsection{Descriptives}

\begin{table}[H]
  \centering
  \begin{threeparttable}
  \caption{Selection into the experiment}
  \begin{tabular*}{\textwidth}{@{\extracolsep{\fill}}lcc c@{}}
  \toprule
  & \multicolumn{1}{c}{\textbf{Admin Data}} & \multicolumn{1}{c}{\textbf{Sample}} & \multicolumn{1}{c}{\textit{\textbf{p}}} \\
  \midrule
  Reading score & 62.651 & 65.183 & 0.000 \\
  Math score & 63.973 & 67.477 & 0.000 \\
  GPA & 3.958 & 4.012 & 0.000 \\
  Low-SES & 0.343 & 0.410 & 0.000 \\
  Med-SES & 0.505 & 0.499 & 0.763 \\
  High-SES & 0.153 & 0.091 & 0.000 \\
  Female & 0.567 & 0.530 & 0.060 \\
  Age & 21.154 & 20.651 & 0.000 \\
  \midrule
  Observations & 4,417 & 734 & 5,151 \\
  \bottomrule
  \end{tabular*}
  \begin{tablenotes}[flushleft]
  \footnotesize
  \item[] \textit{Note:} This table compares characteristics between the full administrative sample and the experimental sample. $p$-values for binary outcomes (Low-SES, Med-SES, High-SES, Female) are from two-sample tests of proportions; for continuous variables, from two-sample $t$-tests with unequal variances. All reported \textit{p}-values are two-tailed.
  \end{tablenotes}
  \label{tab:selection}
  \end{threeparttable}
\end{table}

\begin{table}[H]
  \centering
  \begin{threeparttable}
  \caption{Balance between treatments}
  \begin{tabular*}{\textwidth}{@{\extracolsep{\fill}}lcc c@{}}
  \toprule
   & \multicolumn{1}{c}{\textbf{Baseline}} & \multicolumn{1}{c}{\textbf{Bonus}} & \multicolumn{1}{c}{\textit{\textbf{p}}} \\ 
  \midrule
  Reading score & 64.712 & 65.693 & \multicolumn{1}{c}{0.134} \\
  Math score & 67.366 & 67.597 & \multicolumn{1}{c}{0.780} \\
  GPA & 4.003 & 4.021 & \multicolumn{1}{c}{0.445} \\
  \# connections & 173.40 & 176.88 & \multicolumn{1}{c}{0.574} \\
  Tie strength & 3.939 & 3.719 & \multicolumn{1}{c}{0.443} \\
  Low-SES & 0.419 & 0.401 & \multicolumn{1}{c}{0.615} \\
  Med-SES & 0.492 & 0.506 & \multicolumn{1}{c}{0.714} \\
  High-SES & 0.089 & 0.094 & \multicolumn{1}{c}{0.824} \\
  Female & 0.529 & 0.531 & \multicolumn{1}{c}{0.947} \\
  Age & 20.576 & 20.733 & \multicolumn{1}{c}{0.380} \\
  \midrule
  Observations & 382 & 352 & 734 \\
  \bottomrule
  \end{tabular*}
  \begin{tablenotes}[flushleft]
  \footnotesize
  \item[] \textit{Note:} This table presents balance tests between \textbf{Baseline} and \textbf{Bonus} conditions. $p$-values for binary outcomes are from two-sample tests of proportions; for continuous variables, from two-sample $t$-tests with unequal variances. All reported \textit{p}-values are two-tailed. Tie strength refers to the number of classes taken together. \# connections refers to the number of indiviuduals in referrer choice sets, otherwise called the ``network degree''. Low-SES, Med-SES, and High-SES are binary variables indicating the share of participants in estrato 1 and 2, 3 and 4, or 5 and 6, respectively.
  \end{tablenotes}
  \label{tab:balance}
  \end{threeparttable}
\end{table}

\begin{table}[H]
  \centering
  \begin{threeparttable}
  \caption{Distribution of referrals by area}
  \begin{tabular*}{\textwidth}{@{\extracolsep{\fill}}lccc}
  \toprule
  Area & Only one referral & Both areas & Total \\
  \midrule
  Verbal & 65 & 608 & 673 \\
  Math & 61 & 608 & 669 \\
  \midrule
  Total & 126 & 1,216 & 1,342 \\
  \bottomrule
  \end{tabular*}
  \begin{tablenotes}[flushleft]
  \footnotesize
  \item[] \textit{Note:} The table shows how many referrers made referrals in only one area versus both areas. ``Only one referral'' indicates individuals who made referrals exclusively in that area. ``Both areas'' shows individuals who made referrals in both verbal and math areas. The majority of referrers (608) made referrals in both areas.
  \end{tablenotes}
  \label{tab:referral_distribution}
  \end{threeparttable}
\end{table}

\begin{table}[H]
  \centering
  \begin{threeparttable}
  \caption{Summary statistics for network members by nomination status}
  \begin{tabular*}{\textwidth}{@{\extracolsep{\fill}}lccccc}
  \toprule
  & \multicolumn{2}{c}{Verbal} & \multicolumn{2}{c}{Math} \\
  \cmidrule{2-3} \cmidrule{4-5}
  & Not Referred & Referred & Not Referred & Referred \\
  \midrule
  Reading z-score & 0.070 & 0.509 & 0.079 & 0.465 \\
  & (0.003) & (0.039) & (0.003) & (0.040) \\
  [1ex]
  Math z-score & 0.079 & 0.452 & 0.087 & 0.590 \\
  & (0.003) & (0.042) & (0.003) & (0.043) \\
  [1ex]
  GPA z-score & -0.066 & 0.705 & -0.069 & 0.711 \\
  & (0.003) & (0.041) & (0.003) & (0.041) \\
  [1ex]
  Tie strength z-score & -0.153 & 2.690 & -0.184 & 2.488 \\
  & (0.003) & (0.091) & (0.003) & (0.090) \\
  [1ex]
  Low-SES & 0.334 & 0.374 & 0.338 & 0.384 \\
  & (0.001) & (0.019) & (0.001) & (0.019) \\
  [1ex]
  Med-SES & 0.515 & 0.513 & 0.513 & 0.507 \\
  & (0.001) & (0.019) & (0.001) & (0.019) \\
  [1ex]
  High-SES & 0.151 & 0.113 & 0.149 & 0.109 \\
  & (0.001) & (0.012) & (0.001) & (0.012) \\
  % [1ex]
  % Female & 0.524 & 0.562 & 0.520 & 0.428 \\
  % & (0.001) & (0.019) & (0.001) & (0.019) \\
  % [1ex]
  % Age & 20.945 & 20.501 & 20.961 & 20.490 \\
  % & (0.006) & (0.104) & (0.006) & (0.099) \\
  \midrule
  Observations & 128,174 & 673 & 127,481 & 669 \\
  \bottomrule
  \end{tabular*}
  \begin{tablenotes}[flushleft]
  \footnotesize
  \item[] \textit{Note:} Standard errors in parentheses. GPA, test scores, and tie strength are standardized at the network level. For each referrer's network, we first calculated the mean and standard deviation of each measure. We then computed the average of these means and standard deviations across all referrers. Each individual's score was standardized using these network-level statistics. The standardization formula is $z = (x - \bar{x}_{network}) / \sigma_{network}$, where $\bar{x}_{network}$ and $\sigma_{network}$ are the average of network means and standard deviations, respectively. Low-SES, Med-SES, and High-SES are binary variables indicating the share of participants in estrato 1 and 2, 3 and 4, or 5 and 6, respectively. Tie strength measures the number of connections between individuals.
  \end{tablenotes}
  \label{tab:stats_by_nomination}
  \end{threeparttable}
\end{table}

\begin{table}[H]
  \centering
  \begin{threeparttable}
  \caption{Comparison of math and verbal scores by SES group and data source}
  \begin{tabular*}{\textwidth}{@{\extracolsep{\fill}}lcccccc}
  \toprule
  & \multicolumn{3}{c}{Math} & \multicolumn{3}{c}{Verbal} \\
  \cmidrule{2-4} \cmidrule{5-7}
  & Network & Admin & Sample & Network & Admin & Sample \\
  \midrule
  Low-SES & 66.976 & 61.653 & 67.813 & 64.738 & 60.974 & 66.058 \\
  & (0.052) & (0.346) & (0.694) & (0.043) & (0.274) & (0.574) \\
  [1ex]
  Mid-SES & 65.627 & 64.531 & 66.859 & 63.685 & 63.154 & 64.779 \\
  & (0.039) & (0.224) & (0.580) & (0.032) & (0.183) & (0.436) \\
  [1ex]
  High-SES & 67.781 & 67.330 & 70.610 & 64.966 & 64.892 & 66.397 \\
  & (0.077) & (0.416) & (1.295) & (0.063) & (0.341) & (1.214) \\
  \midrule
  Observations & 128,150 & 4,415 & 669 & 128,847 & 4,403 & 673 \\
  \bottomrule
  \end{tabular*}
  \begin{tablenotes}[flushleft]
  \footnotesize
  \item[] \textit{Note:} Standard errors in parentheses. The table presents mean scores with standard errors for math and verbal tests across the entire network, the admin data, and the sample. Admin data consistently shows lower scores than both network and the sample across all SES groups consistent with selection, with the largest gaps occurring for the Low-SES. Differences between network and sample scores are generally smaller than those between either and the admin data.
  \end{tablenotes}
  \label{tab:ses_scores_comparison}
  \end{threeparttable}
\end{table}

\begin{figure}[H]
  \centering
  \caption{Participant network size and tie strength by time spent at UNAB}\label{connections_class}
  \includegraphics[width=0.6\linewidth]{/Users/reha.tuncer/Documents/GitHub/icfes-referrals/figures/connection_class.png}
  \begin{tablenotes}
    \footnotesize
    \item[] \textit{Note:} This figure displays the average number of connections for referrers in blue and the average number of classes they have taken together with their connections in green across semesters spent at UNAB. The data shows an increase in the number of classes taken together as students progress in their programs, with the connections peaking around 7 semesters and dropping as certain students finish their bachelor's.
  \end{tablenotes}
\end{figure}

\begin{figure}[H]
  \centering
  \caption{Effect of the Bonus on Referrals}
  \begin{subfigure}[t]{0.7\textwidth}
    \centering
      \includegraphics[width=\linewidth]{/Users/reha.tuncer/Documents/GitHub/icfes-referrals/figures/reading_combined.png}
      \caption{Reading}  
      \label{fig:reading_combined}
  \end{subfigure}
  \begin{subfigure}[t]{0.7\textwidth}
    \centering
      \includegraphics[width=\linewidth]{/Users/reha.tuncer/Documents/GitHub/icfes-referrals/figures/math_combined.png}
      \caption{Math}
      \label{fig:math_combined}
  \end{subfigure}
  \label{fig:both_combined}
  \caption*{\footnotesize\textit{Note:} The top panel compares the reading scores and tie strength of referrals across conditions. The bottom panel shows the average standardized math and tie strength of referrals across conditions. We test differences in across conditions using two-sample $t$-tests and find no meaningful differences. For both math and reading, treatment causes no significant changes in referral performance or tie strength.}
\end{figure}


\begin{figure}[H]
  \centering
  \caption{Top decile performer share across the sample, network and referals }\label{topdecile}
  \includegraphics[width=0.6\linewidth]{/Users/reha.tuncer/Documents/GitHub/icfes-referrals/figures/top_decile_comparison.png}
  \begin{tablenotes}
    \footnotesize
    \item[] \textit{Note:} This figure displays the percentage share of top decile individuals according to the admin data across three dimensions. First bar shows referrers in the sample of participants. Second bar is the share of top decile individuals in their networks. Third column shows the share of top decile among the referrals made. We test differences between proportions across these three groups using two-sample tests of proportions. For both math and reading scores, the differences between Sample and Network ($p<0.001$), Sample and Referrals ($p<0.005$), and Network and Referrals ($p<0.001$) are all statistically significant.
  \end{tablenotes}
\end{figure}


  \begin{figure}[H]
    \centering
    \caption{Participant network performance by subject and SES}\label{performance}
    \includegraphics[width=0.6\linewidth]{/Users/reha.tuncer/Documents/GitHub/icfes-referrals/figures/ses_peer_performance.png}
    \begin{tablenotes}
      \footnotesize
      \item[] \textit{Note:} This figure displays the network average math and reading z-scores across referrer SES. We test differences between scores across SES using paired $t$-tests. For both math and reading scores, all differences between SES groups are statistically significant (all $p \leq 0.001$).
    \end{tablenotes}
  \end{figure}

  % \begin{figure}[H]
  %   \centering
  %   \caption{Distribution of top decile performers by subject}\label{topdecile}
  %   \includegraphics[width=0.6\linewidth]{/Users/reha.tuncer/Documents/GitHub/icfes-referrals/figures/strata_scores.png}
  %   \begin{tablenotes}
  %     \footnotesize
  %     \item[] \textit{Note:} This figure displays the percentage of participants across SES who score in the top decile for math and reading. We test differences between math and reading proportions within each SES group using two-sample tests of proportions and find no statistically significant differences ($p > 0.28$). Middle-SES students show the highest representation in the top decile for both subjects in the sample.
  %   \end{tablenotes}
  % \end{figure}



  \begin{figure}[H]
    \centering
    \caption{Participant network composition by SES}
    \includegraphics[width=0.6\textwidth]{/Users/reha.tuncer/Documents/GitHub/icfes-referrals/figures/ses_distribution.png}
    \label{fig:baseline_ses}
    \caption*{\footnotesize\textit{Note:} This figure displays the composition of networks by SES. We test differences in proportions of peer connections across SES groups using two-sample tests of proportions. All differences are statistically significant ($p <$ 0.001): Low SES students are more likely to connect with Low SES peers than Middle or High SES students; Middle SES students form more connections with Middle SES peers than Low SES students; and High SES students have the highest proportion of High SES connections.
    }
\end{figure}

\begin{figure}[H]
  \centering
  \caption{Participant network composition by SES}
  \includegraphics[width=0.6\textwidth]{/Users/reha.tuncer/Documents/GitHub/icfes-referrals/figures/z_ses_tie_strength.png}
  \label{fig:tie_ses}
  \caption*{\footnotesize\textit{Note:} This figure displays the standardized tie strength by SES. We test differences in standardized tie strength across SES groups using two-sample $t$-tests. All differences are statistically significant ($p < 0.001$) except for the comparison between Middle and High SES students' connections to Low SES peers ($p = 0.65$). The standardized tie strength for High SES students with other High SES students is substantially positive (0.26), while all other tie strengths are negative or near zero.}
\end{figure}

%   \begin{figure}[H]
%     \centering
%     \caption{Baseline Referral Patterns by SES}
%     \begin{subfigure}[t]{0.49\textwidth}
%         \centering
%         \includegraphics[width=\linewidth]{/Users/reha.tuncer/Documents/GitHub/icfes-referrals/figures/ses_tie_strength.png}
%         \caption{Referral Rates}
%         \label{fig:tie_ses}
%     \end{subfigure}
%     \hfill
%     \begin{subfigure}[t]{0.49\textwidth}
%         \centering
%         \includegraphics[width=\linewidth]{/Users/reha.tuncer/Documents/GitHub/icfes-referrals/figures/ses_top_tie_strength.png}
%         \caption{Academic Performance of Referrals}
%         \label{fig:tie_topses}
%     \end{subfigure}
%     \label{fig:tie_combinedses}
%     \caption*{\footnotesize\textit{Note:} The left panel shows the availability of SES groups across socioeconomic strata. The right panel shows the availability of top decile performers within each SES group.}
%   \end{figure}

\pagebreak


% \begin{figure}[H]
%   \centering
%   \caption{Participant network composition by SES}
%   \includegraphics[width=\textwidth]{/Users/reha.tuncer/Documents/GitHub/icfes-referrals/figures/combined_baseline.png}
%   \label{fig:baseline_combined}
%   \caption*{\footnotesize\textit{Note:} This figure displays the standardized tie strength by SES. We test differences in z-score tie strength across SES groups using two-sample $t$-tests. All differences are statistically significant ($p < 0.001$) except for the comparison between Middle and High SES students' connections to Low SES peers ($p = 0.65$). The standardized tie strength for High SES students with other High SES students is substantially positive (0.26), while all other tie strengths are negative or near zero. These patterns reveal strong homophily in network formation, particularly among High SES students, while Low SES students show moderate homophily patterns.}
% \end{figure}

  \begin{figure}[H]
    \centering
    \caption{Baseline Referral Patterns by SES}
    \begin{subfigure}[t]{0.49\textwidth}
      \centering
        \includegraphics[width=\linewidth]{/Users/reha.tuncer/Documents/GitHub/icfes-referrals/figures/ses_referral_distribution.png}
        \caption{Referral Rates}  
        \label{fig:baseline_rates}
    \end{subfigure}
    \hfill
    \begin{subfigure}[t]{0.49\textwidth}
      \centering
        \includegraphics[width=\linewidth]{/Users/reha.tuncer/Documents/GitHub/icfes-referrals/figures/baseline_performance.png}
        \caption{Referral Performance}
        \label{fig:baseline_performance}
    \end{subfigure}
    \label{fig:baseline_combined}
    \caption*{\footnotesize\textit{Note:} The left panel shows the distribution of referrals across SES in the baseline condition. We test differences in SES shares across SES groups using two-sample tests of proportions. All differences are statistically significant ($p < 0.1$). The right panel shows the average standardized math and reading scores of referred students by referrer's SES. We test differences in z-scores across SES groups using two-sample $t$-tests and find no statistically significant differences in reading scores across SES groups (all $p > 0.36$). For math scores, we observe marginally significant differences between Low and High SES students ($p = 0.08$) and between Middle and High SES students ($p = 0.18$), with High SES referring peers with higher math performance.}
  \end{figure}

  \begin{figure}[H]
    \centering
    \caption{Effect of the Bonus}
    \label{fig:treat_combined}
    \begin{subfigure}[t]{0.49\textwidth}
        \centering
        \includegraphics[width=\linewidth]{/Users/reha.tuncer/Documents/GitHub/icfes-referrals/figures/ses_treatment_effects.png}
        \caption{Changes in Referral Rates}
        \label{fig:treat_rates}
    \end{subfigure}
    \hfill
    \begin{subfigure}[t]{0.49\textwidth}
        \centering
        \includegraphics[width=\linewidth]{/Users/reha.tuncer/Documents/GitHub/icfes-referrals/figures/treatment_effect.png}
        \caption{Changes in Referral Performance}
        \label{fig:treat_performance}
    \end{subfigure}
    \caption*{\footnotesize\textit{Note:} The left panel shows the changes in referral rates across SES. We test differences in SES shares across conditions using two-sample tests of proportions. For Low-SES, only the change in referral share of Middle-SES is statistically significant ($p = 0.034$). For Middle-SES, only the change in referral share of High-SES is statistically significant ($p = 0.027$). For High-SES, only the change in referral share of Middle-SES is statistically significant ($p = 0.059$). The right panel shows the differences in math and reading z-scores across SES. We test differences in SES shares across conditions using two-sample $t$-tests. For both reading and math scores, the only statistically significant difference is in the reading scores for Low-SES ($p = 0.026$).}
\end{figure}

\begin{figure}[H]
  \centering
  \caption{Effect of the Bonus on Tie Strength}
  \label{fig:treat_combined2}
  \begin{subfigure}[t]{0.49\textwidth}
      \centering
      \includegraphics[width=\linewidth]{/Users/reha.tuncer/Documents/GitHub/icfes-referrals/figures/baseline_ties.png}
      \caption{Changes in Referral Rates}
      \label{fig:tietreat_rates}
  \end{subfigure}
  \hfill
  \begin{subfigure}[t]{0.49\textwidth}
      \centering
      \includegraphics[width=\linewidth]{/Users/reha.tuncer/Documents/GitHub/icfes-referrals/figures/ties_treatment_effects.png}
      \caption{Changes in Referral Performance}
      \label{fig:tietreat_performance}
  \end{subfigure}
  \caption*{\footnotesize\textit{Note:} The left panel shows the changes in referral rates across socioeconomic strata (bonus minus baseline). The right panel shows the differences in average standardized math and reading scores of referred students by referrer's SES.}
\end{figure}





\begin{figure}[H]
  \centering
  \caption{Performance by Tie Strength and SES}
  \includegraphics[width=0.85\textwidth]{/Users/reha.tuncer/Documents/GitHub/icfes-referrals/figures/tie_combined.png}
  \label{fig:tie_score_ses}
  \caption*{\footnotesize\textit{Note:} This figure shows local polynomial regressions of network math and reading z-scores by social tie strength across socioeconomic status groups with 95\% confidence intervals. Higher SES have steeper positive relationships between tie strength and the average performance those in their network across reading and math scores.}
\end{figure}


    \section{Conclusion} \label{s6}

\pagebreak

\bibliographystyle{apacite}

\bibliography{referrals}    
\pagebreak

\appendix \label{appendix-all}
\renewcommand{\thefigure}{A.\arabic{figure}}
\setcounter{figure}{0}

\section{Additional Figures and Tables}

\subsection{Additional Figures}

\pagebreak

\section{Experiment} \label{instructions}
\renewcommand{\thefigure}{B.\arabic{figure}}
\textit{We include the English version of the instructions used in Qualtrics. Participansts saw the Spanish version. Horizontal lines in the text indicate page breaks and clarifiying comments are inside brackets.} 

\section*{Consent}
You have been invited to participate in this decision-making study. This study is directed by [omitted for anonymous review] and organized with the support of the Social Bee Lab (Social Behavior and Experimental Economics Laboratory) at UNAB.

\vspace{5mm}

\noindent In this study, we will pay \textbf{one (1)} out of every \textbf{ten (10)} participants, who will be randomly selected. Each selected person will receive a fixed payment of \textbf{70,000} (seventy thousand pesos) for completing the study. Additionally, they can earn up to \textbf{270,000} (two hundred and seventy thousand pesos), depending on their decisions. So, in total, if you are selected to receive payment, you can earn up to \textbf{340,000} (three hundred and forty thousand pesos) for completing this study.

\vspace{5mm}

\noindent If you are selected, you can claim your payment at any Banco de Bogotá office by presenting your ID. Your participation in this study is voluntary and you can leave the study at any time. If you withdraw before completing the study, you will not receive any payment.

\vspace{5mm}

\noindent The estimated duration of this study is 20 minutes.

\vspace{5mm}

\noindent  The purpose of this study is to understand how people make decisions. For this, we will use administrative information from the university such as the SABER 11 test scores of various students (including you). Your responses will not be shared with anyone and your participation will not affect your academic records. To maintain strict confidentiality, the research results will not be associated at any time with information that could personally identify you.

\vspace{5mm}

\noindent There are no risks associated with your participation in this study beyond everyday risks. However, if you wish to report any problems, you can contact Professor [omitted for anonymous review]. For questions related to your rights as a research study participant, you can contact the IRB office of [omitted for anonymous review].

\vspace{5mm}

\noindent By selecting the option ``I want to participate in the study" below, you give your consent to participate in this study and allow us to compare your responses with some administrative records from the university.

\begin{itemize}
  \item I want to participate in the study [advances to next page]
  \item I do not want to participate in the study
\end{itemize}

\noindent\rule{\textwidth}{1pt}

\section*{Student Information}

Please write your student code.
In case you are enrolled in more than one program simultaneously, write the code of the first program you entered:

\vspace{5mm}

\noindent[Student ID code]

\vspace{5mm}

\noindent What semester are you currently in?

\vspace{5mm}

\noindent[Slider ranging from 1 to 11]

\noindent\rule{\textwidth}{1pt}

\vspace{5mm}

\noindent[Random assignment to treatment or control]

\section*{Instructions}

The instructions for this study are presented in the following video. Please watch it carefully. We will explain your participation and how earnings are determined if you are selected to receive payment.

\vspace{5mm}

\noindent[Treatment-specific instructions in video format]

\vspace{5mm}

\noindent If you want to read the text of the instructions narrated in the video, press the ``Read instruction text" button. Also know that in each question, there will be a button with information that will remind you if that question has earnings and how it is calculated, in case you have any doubts.

\begin{itemize}
  \item I want to read the instructions text [text version below]
\end{itemize}

\noindent\rule{\textwidth}{1pt}

\vspace{5mm}

\noindent In this study, you will respond to three types of questions. First, are the belief questions. For belief questions, we will use as reference the results of the SABER 11 test that you and other students took to enter the university, focused on three areas of the exam: mathematics, reading, and English.

\vspace{5mm}

\noindent For each area, we will take the scores of all university students and order them from lowest to highest. We will then group them into 100 percentiles. The percentile is a position measure that indicates the percentage of students with an exam score that is above or below a value.

\vspace{5mm}

\noindent For example, if your score in mathematics is in the 20th percentile, it means that 20 percent of university students have a score lower than yours and the remaining 80 percent have a higher score. A sample belief question is: ``compared to university students, in what percentile is your score for mathematics?"

\vspace{5mm}

\noindent If your answer is correct, you can earn 20 thousand pesos. We say your answer is correct if the difference between the percentile you suggest and the actual percentile of your score is not greater than 7 units. For example, if you have a score that is in the 33rd percentile and you say it is in the 38th, the answer is correct because the difference is less than 7. But if you answer that it is in the 41st, the difference is greater than 7 and the answer is incorrect.

\vspace{5mm}

\noindent The second type of questions are recommendation questions and are also based on the mathematics, reading, and English areas of the SABER 11 test. We will ask you to think about the students with whom you have taken or are taking classes, to recommend from among them the person you consider best at solving problems similar to those on the SABER 11 test.

\vspace{5mm}

\noindent When you start typing the name of your recommended person, the computer will show suggestions with the full name, program, and university entry year of different students. Choose the person you want to recommend. If the name doesn't appear, check that you are writing it correctly. Do not use accents and use `n' instead of `ñ'. If it still doesn't appear, it may be because that person is not enrolled this semester or because they did not take the SABER 11 test. In that case, recommend someone else.

\vspace{5mm}

\noindent You can earn up to 250,000 pesos for your recommendation. We will multiply your recommended person's score by 100 pesos if they are in the first 50 percentiles. We will multiply it by 500 pesos if your recommended person's score is between the 51st and 65th percentile. If it is between the 66th and 80th percentile, we will multiply your recommended person's score by 1000 pesos. If the score is between the 81st and 90th percentile, you earn 1500 pesos multiplied by your recommended person's score. And if the score is between the 91st and 100th percentile, we will multiply your recommended person's score by 2500 pesos to determine the earnings.

\vspace{5mm}

\noindent The third type of questions are information questions and focus on aspects of your personal life or your relationship with the people you have recommended.


\subsection*{Earnings}

Now we will explain who gets paid for participating and how the earnings for this study are assigned. The computer will randomly select one out of every 10 participants to pay for their responses. For selected individuals, the computer will randomly choose one of the three areas, and from that chosen area, it will pay for one of the belief questions.

\vspace{5mm}

\noindent Similarly, the computer will randomly select one of the three areas to pay for one of the recommendation questions.

\vspace{5mm}

\noindent \textbf{Additionally, if you are selected to receive payment, your recommended person in the  chosen area will receive a fixed payment of 100 thousand pesos.} [Only seen if assigned to the treatment] 

\vspace{5mm}

\noindent Each person selected to receive payment for this study can earn: up to 20 thousand pesos for one of the belief questions, up to 250 thousand pesos for one of the recommendation questions, and a fixed payment of 70 thousand pesos for completing the study.

\vspace{5mm}

\noindent Selected individuals can earn up to 340 thousand pesos.

\noindent\rule{\textwidth}{1pt}

\vspace{5mm}

\noindent [Participants go through all three Subject Areas in randomized order]

\section*{Subject Areas}

\subsection*{Critical Reading}
For this section, we will use as reference the Critical Reading test from SABER 11, which evaluates the necessary competencies to understand, interpret, and evaluate texts that can be found in everyday life and in non-specialized academic fields.

\vspace{5mm} 

\noindent [Clicking shows the example question from SABER 11 below]

\vspace{5mm} 

\noindent Although the democratic political tradition dates back to ancient Greece, political thinkers did not address the democratic cause until the 19th century. Until then, democracy had been rejected as the government of the ignorant and unenlightened masses. Today it seems that we have all become democrats without having solid arguments in favor. Liberals, conservatives, socialists, communists, anarchists, and even fascists have rushed to proclaim the virtues of democracy and to show their democratic credentials (Andrew Heywood). According to the text, which political positions identify themselves as democratic?

\begin{itemize}
  \item Only political positions that are not extremist
  \item The most recent political positions historically
  \item The majority of existing political positions
  \item The totality of possible political currents
\end{itemize}

\vspace{5mm}

\noindent\rule{\textwidth}{1pt}


\subsection*{Mathematics}
This section references the Mathematics test from SABER 11, which evaluates people's competencies to face situations that can be resolved using certain mathematical tools.

\vspace{5mm} 

\noindent [Clicking shows the example question from SABER 11 below]

\vspace{5mm} 

\noindent A person living in Colombia has investments in dollars in the United States and knows that the exchange rate of the dollar against the Colombian peso will remain constant this month, with 1 dollar equivalent to 2,000 Colombian pesos. Their investment, in dollars, will yield profits of 3\% in the same period. A friend assures them that their profits in pesos will also be 3\%. Their friend's statement is:

\begin{itemize}
    \item Correct. The proportion in which the investment increases in dollars is the same as in pesos.
    \item Incorrect. The exact value of the investment should be known.
    \item Correct. 3\% is a fixed proportion in either currency.
    \item Incorrect. 3\% is a larger increase in Colombian pesos.
\end{itemize}

\vspace{5mm}

\noindent\rule{\textwidth}{1pt}


\subsection*{English}
This section uses the English test from SABER 11 as a reference, which evaluates that the person demonstrates their communicative abilities in reading and language use in this language.

\vspace{5mm} 

\noindent [Clicking shows the example question from SABER 11 below]

\vspace{5mm} 

\noindent Complete the conversations by marking the correct option.
    \begin{itemize}
        \item Conversation 1: I can't eat a cold sandwich. It is horrible!
        \begin{itemize}
            \item I hope so.
            \item I agree.
            \item I am not.
        \end{itemize}
        \item Conversation 2: It rained a lot last night!
        \begin{itemize}
            \item Did you accept?
            \item Did you understand?
            \item Did you sleep?
        \end{itemize}
    \end{itemize}

\noindent\rule{\textwidth}{1pt}

\vspace{5mm}

\noindent [Following parts are identical for all Subject Areas and are not repeated here for brevity]

\subsection*{Your Score}

Compared to university students, in which percentile do you think your [\textbf{Subject Area}] test score falls (1 is the lowest percentile and 100 the highest)?

\vspace{5mm} 

\noindent [Clicking shows the explanations below]

\vspace{5mm} 

\noindent How is a percentile calculated?

\vspace{5mm} 

\noindent A percentile is a position measurement. To calculate it, we take the test scores for all students currently enrolled in the university and order them from lowest to highest. The percentile value you choose refers to the percentage of students whose score is below yours. For example, if you choose the 20th percentile, you're indicating that 20\% of students have a score lower than yours and the remaining 80\% have a score higher than yours.

\vspace{5mm} 

\noindent What can I earn for this question?

\vspace{5mm} 

\noindent For your answer, you can earn \textbf{20,000 (twenty thousand) PESOS}, but only if the difference between your response and the correct percentile is less than 7. For example, if the percentile where your score falls is 33 and you respond with 38 (or 28), the difference is 5 and the answer is considered correct. But if you respond with 41 or more (or 25 or less), for example, the difference would be greater than 7 and the answer is incorrect.


\vspace{5mm} 

\noindent Please move the sphere to indicate which percentile you think your score falls in:

\vspace{5mm} 

\noindent[Slider with values from 0 to 100]


\vspace{5mm} 

\noindent\rule{\textwidth}{1pt}

\subsection*{Recommendation}

Among the people with whom you have taken any class at the university, who is your recommendation for the [\textbf{Subject Area}] test? Please write that person's name in the box below:

\vspace{5mm} 

\noindent \textbf{\textcolor{red}{Important:}} \textbf{You will not be considered for payment unless the recommended person is someone with whom you have taken at least one class during your studies.}


\vspace{5mm} 

\noindent Your response is only a recommendation for the purposes of this study and we will \textbf{not} contact your recommended person at any time.

\vspace{5mm} 

\noindent [Clicking shows the explanations below]

\vspace{5mm} 

\noindent Who can I recommend?

\vspace{5mm} 

\noindent Your recommendation \textbf{must} be someone with whom you have taken (or are taking) a class. If not, your answer will not be considered for payment. The person you recommend will not be contacted or receive any benefit from your recommendation.

\vspace{5mm} 

\noindent As you write, you will see up to 7 suggested student names containing the letters you have entered. The more you write, the more accurate the suggestions will be. Please write \textbf{without} accents and use the letter `n' instead of `ñ'. If the name of the person you're writing doesn't appear, it could be because you made an error while writing the name.

\vspace{5mm} 

\noindent If the name is correct and still doesn't appear, it could be because the student is not enrolled this semester or didn't take the SABER 11 test. In that case, you must recommend someone else.


\vspace{5mm} 

\noindent My earnings for this question?

\vspace{5mm} 

\noindent For your recommendation, you could receive earnings of up to 250,000 (two hundred and fifty thousand) PESOS. The earnings are calculated based on your recommendation's score and the percentile of that score compared to other UNAB students, as follows:

\begin{itemize}
  \item We will multiply your recommendation's score by \$100 (one hundred) pesos if it's between the 1st and 50th percentiles
  \item We will multiply your recommendation's score by \$500 (five hundred) pesos if it's between the 51st and 65th percentiles
  \item We will multiply your recommendation's score by \$1000 (one thousand) pesos if it's between the 66th and 80th percentiles
  \item We will multiply your recommendation's score by \$1500 (one thousand five hundred) pesos if it's between the 81st and 90th percentiles
  \item We will multiply your recommendation's score by \$2500 (two thousand five hundred) pesos if it's between the 91st and 100th percentiles
\end{itemize}


\vspace{5mm} 

\noindent This is illustrated in the image below:

\begin{figure}[H]
    \centering
    \includegraphics[width=0.65\textwidth]{/Users/reha.tuncer/Documents/GitHub/icfes-referrals/figures/bonus_structure.png}
    \caption{Earnings for recommendation questions}
    \label{fig:earnings}
\end{figure}


\vspace{5mm} 

\noindent For example, if your recommendation got 54 points and the score is in the 48th percentile, you could earn 54x100 = 5400 PESOS. But, if the same score of 54 points were in the 98th percentile, you could earn 54x2500 = 135,000 PESOS.

\vspace{5mm} 

\noindent [Text field with student name suggestions popping up as participant types]

\vspace{5mm} 

\noindent\rule{\textwidth}{1pt}

\subsection*{Relationship with your recommendation}
How close is your relationship with your recommendedation: ``[Name of the student selected from earlier]"? (0 indicates you are barely acquaintances and 10 means you are very close)

\vspace{5mm} 

\noindent [Slider with values from 0 to 10]

\vspace{5mm} 

\noindent\rule{\textwidth}{1pt}

\subsection*{Your recommendation's score}
Compared to university students, in which percentile do you think [Name of the student selected from earlier]'s score falls in the [\textbf{Subject Area}] test (1 is the lowest percentile and 100 the highest)?

\vspace{5mm} 

\noindent [Clicking shows the explanations below]

\vspace{5mm} 

\noindent How is a percentile calculated?

\vspace{5mm} 

\noindent A percentile is a position measurement. To calculate it, we take the test scores for all students currently enrolled in the university and order them from lowest to highest. The percentile value you choose refers to the percentage of students whose score is below yours. For example, if you choose the 20th percentile, you're indicating that 20\% of students have a score lower than yours and the remaining 80\% have a score higher than yours.

\vspace{5mm} 

\noindent What can I earn for this question?

\vspace{5mm} 

\noindent For your answer, you can earn \textbf{20,000 (twenty thousand) PESOS}, but only if the difference between your response and the correct percentile is less than 7. For example, if the percentile where your recommended person's score falls is 33 and you respond with 38 (or 28), the difference is 5 and the answer is considered correct. But if you respond with 41 or more (or 25 or less), for example, the difference would be greater than 7 and the answer is incorrect.

\vspace{5mm} 

\noindent  Please move the sphere to indicate which percentile you think your recommended person's score falls in:

\vspace{5mm} 

\noindent[Slider with values from 0 to 100]

\vspace{5mm} 

\noindent\rule{\textwidth}{1pt}

\section*{Demographic Information}

What is the highest level of education achieved by your father?

\vspace{5mm} 

\noindent [Primary, High School, University, Graduate Studies, Not Applicable]

\vspace{5mm} 

\noindent What is the highest level of education achieved by your mother?

\vspace{5mm} 

\noindent [Primary, High School, University, Graduate Studies, Not Applicable]

\vspace{5mm} 

\noindent Please indicate the socio-economic group to which your family belongs:

\vspace{5mm}

\noindent [Group A (Strata 1 or 2), Group B (Strata 3 or 4), Group C (Strata 5 or 6)]

\vspace{5mm}

\noindent\rule{\textwidth}{1pt}

\section*{UNAB Students Distribution}
Thinking about UNAB students, in your opinion, what percentage belongs to each socio-economic group? The total must sum to 100\%:

\vspace{5mm}

\noindent [Group A (Strata 1 or 2) percentage input area] 

\noindent [Group B (Strata 3 or 4) percentage input area] 

\noindent [Group C (Strata 5 or 6) percentage input area]

\noindent [Shows sum of above percentages]

\vspace{5mm} 

\noindent\rule{\textwidth}{1pt}

\section*{End of the Experiment}
Thank you for participating in this study.

\vspace{5mm}

\noindent If you are chosen to receive payment for your participation, you will receive a confirmation to your UNAB email and a link to fill out a form with your information. The process of processing payments is done through Nequi and takes approximately 15 business days, counted from the day of your participation.

\vspace{5mm}

\noindent [Clicking shows the explanations below]

\vspace{5mm}

\noindent Who gets paid and how is it decided?

\vspace{5mm}

\noindent The computer will randomly select one out of every ten participants in this study to be paid for their
decisions.

\vspace{5mm}

\noindent For selected individuals, the computer will randomly select one area: mathematics, reading, or English, and from that area will select one of the belief questions. If the answer to that question is correct, the participant will receive 20,000 pesos.

\vspace{5mm}

\noindent The computer will randomly select an area (mathematics, critical reading, or English) to pay for one of the recommendation questions. The area chosen for the recommendation question is independent of the area chosen for the belief question. The computer will take one of the two recommendations you have made for the chosen area. Depending on your recommendation's score, you could win up to 250,000 pesos.

\vspace{5mm}

\noindent Additionally, people selected to receive payment for their participation will have a fixed earnings of 70,000 pesos for completing the study.

\vspace{5mm}

\noindent\rule{\textwidth}{1pt}

\section*{Participation}
In the future, we will conduct studies similar to this one where people can earn money for their participation. The participation in these studies is by invitation only. Please indicate if you are interested in being invited to other studies similar to this one:

\vspace{5mm}

\noindent [Yes, No]


\end{document}
