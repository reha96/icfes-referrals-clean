\documentclass[11pt,a4paper,oneside]{article}
\linespread{1.5}
\usepackage{amsmath,amsthm,amssymb,amsfonts,adjustbox,bm}
\usepackage{geometry}
\usepackage{threeparttable}
%\usepackage[dvipsnames]{xcolor}
\usepackage{mathtools}
\usepackage{refstyle}
\usepackage{pdflscape}
\usepackage{longtable}
\usepackage{eurosym}
\usepackage{enumerate}
\usepackage{booktabs}
\usepackage{siunitx}
\usepackage{rotating}
\usepackage{graphicx}
\usepackage{bbm}
\usepackage{subcaption}
\usepackage{caption}
\captionsetup{width=.8\textwidth}
\usepackage[section]{placeins}
\usepackage{xcolor}
\usepackage{geometry}
\usepackage{changepage}
\usepackage{float}
\usepackage[natbibapa]{apacite}
\usepackage{threeparttable}
\newcommand{\footremember}[2]{%
   \footnote{#2}
    \newcounter{#1}
    \setcounter{#1}{\value{footnote}}%
}
\newcommand{\footrecall}[1]{%
    \footnotemark[\value{#1}]%
} 
\usepackage{lineno}
\makeatletter
\def\makeLineNumberLeft{%
  \linenumberfont\llap{\hb@xt@\linenumberwidth{\LineNumber\hss}\hskip\linenumbersep}% left line number
  \hskip\columnwidth% skip over column of text
  \rlap{\hskip\linenumbersep\hb@xt@\linenumberwidth{\hss\LineNumber}}\hss}% right line number
\leftlinenumbers% Re-issue [left] option
\makeatother

\usepackage{blindtext}

\usepackage[colorlinks=true, citecolor=blue, linkcolor=blue, urlcolor=blue,breaklinks]{hyperref}


\title{Class differences in social networks: Evidence from a referral experiment}
% \\ \large Experimental Design \\
% \title{Skills, Inequality, and Referrals}
\author{Manuel Munoz\footremember{liser}{Luxembourg Institute of Socio-Economic Research}, Ernesto Reuben\footnote{Division of Social Science, New York University Abu Dhabi} \footrecall{liser}, Reha Tuncer\footnote{University of Luxembourg}}  
% \footnote{Center for Behavioral Institutional Design at New York University Abu Dhabi} 
\begin{document}
\linenumbers 
% HERE FOR LINES
\maketitle



\section*{Abstract}
Lorem Ipsum \citep{beaman_job_2018}
\medskip \\
\textbf{JEL Classification:} C93, D03, D83, J24 \\
\textbf{Keywords:} productivity beliefs, referrals, field experiment, skill identification, social class

% Problem: Labor markets increasingly value both cognitive and social skills, but firms struggle to identify these skills in candidates. Referrals are a common hiring tool, but their effectiveness across different skill types and potential social class biases remain unclear.
% Method: Using a field experiment with 849 university students, we study how well peers identify cognitive and social skills of their classmates.
% Key Results: Students effectively identify cognitive skills but struggle with social skills. Limited evidence of social class bias in referrals.
% Implications: Referrals may be more effective for identifying certain types of skills than others.
% We study how well referrals help screening for the two types of worker skills within the same network and compare the differential skill identification ability of referrers.

\pagebreak
\section{Introduction}
Equally qualified individuals face different labor market outcomes depending on their socioeconomic status \citep{stansbury2024class}. A key driver of this inequality is due to differences in social capital.\footnote{See for example \citet{loury_dynamic_1977,bourdieu_forms_1986} for pioneering work on the relationship between social position and human capital acquisition.} Because it correlates strongly with labor market income, the most important facet of social capital is the share of high-SES connections among low-SES individuals \citep{chetty2022na}. A lack of social capital means lack of access to individuals with influential (higher paid) jobs and job opportunities. In economic terms, it implies having worse outcomes when using one's network to find jobs conditional on the capacity on leveraging one's social network.\footnote{See for example \citet{lin1981asr,mouw2003asr} for differential outcomes while using contacts in job search, and \citet{smith2005ajs,Pedulla2019RaceAN} specifically for the effects of race conditional on network use.} 

Referral hiring, the formal or informal process where firms ask workers to recommend qualified candidates for job opportunities, is a common labor market practice which makes evident the role of differences in social capital. As referrals originate from the networks of referrers, the composition of referrer networks becomes a crucial channel that propagates inequality: Similar individuals across socio-demographic characteristics form connections at higher rates \citep{mcpherson2001birds}, making across SES (low-to-high) connections less likely than same-SES connections \citep{chetty2022na}. Referrals will thus reflect similarities in socio-demographic characteristics present in networks even in the absence of biases in the referral procedure, i.e., referring at random from one's network according to some productivity criteria.

Yet, experimental evidence shows referrals can be biased even under substantial pay-for-performance incentives beyond what is attribuable to differences in network compositions, at least for the case of gender \citep{beaman_job_2018,hederos2025gender}. A similar bias against low-SES may further exacerbate outcomes of low-SES individuals: If job information are in the hands of a select few high-SES which low-SES have already limited network access to (social capital hypothesis), and high-SES referrers are biased against low-SES, referring other high-SES at higher rates than their network composition, we should expect referral hiring to further disadvantagee low-SES. 

The empirical question we answer in this paper is whether referrers are biased against low-SES peers after accounting for differences in the network SES compositon. We also evaluate the causal impact of two different incentive structures on referral behavior.

In this study, we study inequalities related to SES combining a university-wide cross-sectional network data set comprising over 4,500 students in which classroom interactions are recorded along with individual attributes. We focus on the role of SES in referrals by experimentally investigating whether individuals who are asked to refer a peer tend to refer a same-SES candidate. We also explore potential mechanisms behind referral patterns by randomizing participants into two different incentive structures. To this end, we conducted a lab-in-the-field experiment with 734 students in a Colombian university. Participants were instructed to refer a qualified student for tasks similar to the math and reading parts of the national university entry exam (equivalent of SAT in US system). To incentivize participants to refer qualified candidates, we set earnings dependent on referred candidates' actual university entry exam scores.

Referral hiring in the labor market can range from firm-level formal referral programs asking employees to bring candidates to simply passing on job opportunities between network members \citep{topa2019social}. As our participants are students at the university and refer based on exam scores, we abstract away from formal referral programs with defined job openings. Our setting instead resembles situations where contacts share opportunities with each other without the need for the referred candidate to take any action and without revealing the identity of the referrer. This eliminates reputational concerns as there is no hiring firm, and puts a lower bound on the expected reciprocity for the referrer in combination with pay-for-performance incentives \citep{witte2021workers,bandiera2009social}. At the same time, referring based on university entry exam scores are still an objective, widely accepted measure of ability, and we show evidence that referrers in our setting not only possess accurate information about these signals but are also able to screen more productive individuals from their university network.

In a university setting, class attendance provides essential opportunities for face-to-face interaction between students. On the one hand, this reduces network segregation by providing ample opportunities to meet across-SES, because of the exposure to an equal or higher level of high-SES compared to the population \citep{chetty2022n}.\footnote{In a different sample from the same university population, \citet{diaz_peer_2025} show this holds true for the highest-SES individuals at this institution, accounting for about 6\% of their sample but less than 5\% of Colombian high-school graduates \citet{fergusson2021desigualdad}.} On the other hand, as students take more and more classes together, their similarities across all observable characteristics tend to increase \citep{kossinets_origins_2009}, which should drive the high- and low-SES networks to segregate. Our setting is ideal to study these opposing forces: First, The very high level of income inequality and existence of deeply rooted historical groups in Colombia makes SES differences extremely visible in access to tertiary education, where the rich and poor typically select into different institutions \citep{jaramillo-echeverri2023}. Yet, thanks to the particular standing of the institution we have chosen for this study (Figure \ref{school_quality_income}), all SES groups including both low- and high-SES mix together in this university. Second, using administrative data, we are able to reconstruct 734 participants' complete university network based on the number of common courses they have taken together with other students. This allows directly identifying the individual characteristics of those getting referrals among all possible candidates, as well as descriptive characterizations of similarity (e.g., in same-SES share) in student networks as a function of the number of classes taken. 

We find strong evidence that networks of high- and low-SES participants exhibit same-SES bias. Both groups are connected at higher rates with their own SES group than what would be at random given actual group shares at the university (Figure \ref{availability_lses}). As students take more courses together within the same program, their networks dwindle in size (Figures \ref{share_same_program_by_tie} and \ref{conns_by_tie}), and become more homogenous in SES-shares (Figure \ref{share_same_ses_by_tie}). We identify selection into academic programs as a key mechanism. The private university where our study took place implements exogenous cost-based program pricing and does not offer SES-based price reductions. These result in programs with very large cost differences within the same university (Figure \ref{fees_with_boxplot}). We find that average yearly fee paid per student increases with SES, and the high-SES share in the most expensive program at the university, medicine, drives the network segregation across SES (Figure \ref{ses_distribution_by_fees}). 

Do segregated networks account for all the differences in SES referral rates across SES groups? Although same-SES referrals are 17\% more common than is suggested by referrer networks, controlling for these, we find no general SES-bias against beyond what is attribuable to network composition. Regardless of SES, participants refer productive individuals, and referred candidates are characterized by a very high number of courses taken together. The latter underlies the impact of program selection, where smaller and more homogenous parts of the networks are activated for referrals made in our setting. Our treatment randomized participants across two different incentive schemes by adding a substantial monetary bonus (\$25) for the referred candidate on top of the pay-for-performance incentives. We provide evidence that treatment incentives did not change the referral behavior across the same-SES referral rate, the number of courses taken together with the referral candidate, and the candidate's exam scores.  

This paper contributes to the literature on referral experiments by solving the challenge of observing the entire referral network. Earlier research could only compare referrals made across different incentive structures or experimental instructions and make according conclusions. For example, when participants are paid on the basis of their referred candidate's productivity instead of receiving a fixed finder's fee \citep{beaman_who_2012}, or when participants are restrictred to refer either a male or female candidate instead of freely \citep{beaman_job_2018}. \citet{pallais_why_2016} recruited a random sample of nonreferred workers to compare with referred ones, but none of the previous studies could provide a direct comparison of the referral choice set with those who were selected by participants. Closest to our work is the work of \citet{hederos2025gender}, who elicited friendship networks by asking referrers to name 5 friends. Their findings suggest only half of those who were referred were from the elicited friendship network, and thus is not a complete observation of the referral choice set. Although commonplace, censored elicitation methods also result in underestimating network effects \citep{griffith2022jle} and may suffer from biases in recall. We are able to take our analysis one step further by asking for referrals from the enrollment network, where we have complete information on every single connection that may or may not get a referral. This allows us to neatly separate the effect of the network composition from any potential biases stemming from the referral procedure itself.  

Second, we build upon to the earlier work on SES-biases in referrals. To our knowledge, the first to study SES-biases in referals are \citet{diaz_peer_2025}, and our study is conceptually the closest to theirs. Drawing from a similar sample from the same institution, \citet{diaz_peer_2025} focus on referrals from first year students made within mixed-program classrooms, and find no evidence for an aggregate bias against low-SES. We also find no aggregate bias against low-SES in referrals. Our setup differs as we sample from students who completed their first year and impose no limits on referring from a classroom. This has several implications: We find that referrals in our setup go to individuals within the same program, and that programs have different SES-shares which become more even more accentuated as students take more courses together. While networks drive inequality in referral outcomes because of the institional environment in our sample, we have no reason to believe first year student networks in \citet{diaz_peer_2025} have similar levels of segregation to begin with. Following the recent evidence, implementing more mixed-program courses which allow for across-SES mixing can be a clear policy goal \citep{alan2023troeasa,rohrer2021po}.

Finally, we contribute to the growing literature on SES differences in the labor market, expliciting the role of networks as a driver of inequality.
\citet{stansbury2024class} find that low-SES researchers coauthor more often with other low-SES, and have networks that have lower values which can explain why 

The remainder of the paper is organized as follows. Section \ref{s2} begins with the background and setting in Colombia. In Section \ref{s3} we present the design of the experiment. In Section \ref{s4} we describe the data and procedures. Section \ref{s5} discusses the results of the experiment. Section \ref{s6} concludes. The Appendix presents additional tables and figures as well as the experiment instructions.

\begin{figure}[H]
  \centering
  \caption{Income, performance, and university choice in Colombia}\label{school_quality_income}
  \includegraphics[width=0.6\linewidth]{../figures/school_quality_income.png}
  \begin{tablenotes}
    \footnotesize
    \item[] \textit{Note:} This figure shows the average score national university entry exam by family income and type of higher education instition. With average student score in the 65-70 band, the private university where we conducted this experiment caters to low- and high-SES students. Figure reproduced from \citet{fergusson2021distincion}.
  \end{tablenotes}
\end{figure}

\section{Background and Setting} \label{s2}

Our study takes place at UNAB, a medium-sized private university in Bucaramanga, Colombia with approximately 6,000 enrolled students. The university's student body is remarkably diverse with about 35\% of the students classified as low-SES, and 15\% high-SES. Diversity at this institution provides a unique research setting as Colombian society is highly unequal and generally characterized by limited interaction between social classes, with different socioeconomic groups separated by education and geographic residence.\footnote{Colombia has consistently ranked as one of the most unequal countries in Latin America \citep{barretoherrera2024regional}, with the richest decile earning 50 times more than the poorest decile \citep{comisioneconomicaparaamericalatinayel2023}. This economic disparity is reflected by a highly stratified society with significant class inequalities and limited class mobility \citep{angulo2012movilidad,garcia2015loteria}.} Despite significant financial barriers, many lower and middle-SES families prioritize university education for their children \citep[p. 103]{hudson_colombia_2010}, and UNAB represents one of the few environments in Colombia where sustained inter-SES contact occurs naturally (see Figure \ref{school_quality_income}).

In 1994, Colombia introduced a nationwide classification system dividing the population into 6 strata based on housing characteristics and neighborhood amenities.\footnote{Initially designed for utility subsidies from higher strata (5 and 6) to support lower strata (1 to 3), it now extends to university fees and social program eligibility. Stratum 4 neither receives subsidies nor pays extra taxes. This stratification system largely aligns with and potentially reinforces existing social class divisions \citep{uribe2008estratificacion, guevara2019spatializing}.} We use this classification as the measure of SES in our experiment: Students in strata 1 to 2 are categorized as low-SES, strata 3 to 4 as middle-SES and those in strata 5 to 6 as high-SES.

We invited via email all 4,417 UNAB undergraduate students who had at the time of recruitment completed their first year at the university to participate in our experiment. 837 students who joined (19\%) vary in terms of their academic programs, SES, and progress in their studies. This setup provides a unique opportunity for collaborative inter-class contact on equal status, whose positive effects on reducing discrimination are casually documented \citep{rao2019familiarity, mousa2020building,lowe2021types}.

Undergraduate programs at UNAB are spread across two semesters, with each individual course lasting one semester. Students take between 5 to 7 courses per semester, with programs lasting anywhere between 4 to 12 semesters (2 to 6 years). Medicine, the largest program by size at UNAB, lasts for 12 semesters, followed by engineering programs at 10 semesters. Most remaining programs lasting for about 8 to 10 semesters, with specialized programs for immediate entry into the workforce lasting only 4.       

\section{Design} \label{s3}

We designed an experiment to assess peer referral behavior from an SES perspective and to causally evaluate the effect of different incentive structures on referrals. The study design consists of a single online experiment organized at the university level (see Figure \ref{timeline}). The instructions are provided in Appendix \ref{instructions}.

\begin{figure}[H]
\centering
\caption{Experiment Timeline}\label{timeline}
\includegraphics[width=\linewidth]{../figures/timeline.jpg}
\begin{tablenotes}
\footnotesize
\item[] \textit{Note:} Participants first report beliefs about their own national university entry exam performance, then recommend peers for each academic area. In the final part, they report beliefs about their recommendations' performance and provide demographic information. This order is implemented for all participants.
\end{tablenotes}
\end{figure}

\subsection{Productivity measures}
To establish an objective basis for referral productivity, we use national university entry exam scores (SABER 11). These scores provide pre-existing, comparable measures of ability across two domains relevant for the labor market. By using existing administrative data, we eliminate the need for additional testing and ensure that all eligible students have comparable productivity measures. The scores we use in this experiment comprise of critical reading and mathematics parts.

Critical reading evaluates competencies necessary to understand, interpret, and evaluate texts found in everyday life and broad academic fields (e.g., history). This measures students' ability to comprehend and critically evaluate written material. Mathematics assesses students' competency in using undergraduate level mathematical tools (e.g., reasoning in proportions, financial literacy). This captures quantitative reasoning and problem-solving abilities.

For each area, we calculate percentile rankings based on the distribution of scores among all currently enrolled UNAB students, providing a standardized measure of relative performance within the university population.

\subsection{Referral task}
After eliciting beliefs about their own performance, participants engage in incentivized peer recommendations. For both test areas (critical reading and mathematics), participants recommend one peer they believe excels in that domain. We first present an example question from the relevant test area to clarify what skills are being assessed. Participants then type the name of their recommended peer, with the system providing autocomplete suggestions from enrolled students who have taken the test (see Figure \ref{interface}).

\begin{figure}[H]
  \centering
  \caption{Referral task interface}\label{interface}
  \includegraphics[width=0.6\linewidth]{/Users/reha.tuncer/Documents/GitHub/icfes-referrals/figures/interface.png}
  \begin{tablenotes} 
    \footnotesize
    \item[] \textit{Note:} This illustration shows how the system provides suggestions from enrolled students with their program and year of study from the administrative database.
  \end{tablenotes}
\end{figure}

Participants can only recommend students with whom they have taken at least one class during their university studies. This requirement ensures that referrals are based on actual peer interactions and overlap with the enrollment network that we construct. The order in which participants make recommendations across the two areas is randomized.

We incentivize referrals using a productivity-based payment scheme. Referrers earn increasing monetary rewards as the percentile ranking of their recommendation increases (see Figure \ref{piece_rate_by_score}). We multiply the piece rate coefficient associated to the percentile rank with the actual test scores of the recommendation to calculate earnings. This payment structure provides strong incentives to screen for highly ranked peers, with potential earnings up to \$60 per recommendation.\footnote{Due to the selection into the university, the actual test score distribution has limited variance. Below a certain threshold students cannot qualify for the institution and choose a lower ranked university, and above a certain threshold they have better options to choose from.}

\begin{figure}[H]
  \centering
  \caption{Referral incentives}\label{piece_rate_by_score}
  \includegraphics[width=0.6\linewidth]{/Users/reha.tuncer/Documents/GitHub/icfes-referrals/figures/piece_rate_by_score.png}
  \begin{tablenotes}
    \footnotesize
    \item[] \textit{Note:} This figure shows how the piece rate coefficient increases as a function of the referral ranking in the university, providing incrementally higher rewards for higher ranked peers.
  \end{tablenotes}
\end{figure}

\subsection{Treatment variation}
We implement a between-subjects treatment that varies whether the recommended peer also receives payment. In the \textbf{Baseline} treatment, only the referrer can earn money based on their recommendation's productivity. The \textbf{Bonus} treatment adds an additional fixed payment of \$25 to any peer who is recommended in the randomly selected area for payment. This payment is independent of the peer's actual productivity (see Figure \ref{tab:treatments}).

\begin{table}[htbp]
  \centering
  \begin{threeparttable}
  \caption{Incentive structure by treatment}
  \begin{tabular*}{\textwidth}{@{\extracolsep{\fill}} l c c @{}}
  \toprule
  & \multicolumn{1}{c}{\textbf{Baseline}} & \multicolumn{1}{c}{\textbf{Bonus}} \\
  \midrule
  Referrer (sender) & Productivity-based & Productivity-based \\
  Recommendation (receiver) & No payment  & Fixed reward  \\
  \bottomrule
  \end{tabular*}
  \label{tab:treatments}
  \end{threeparttable}
\end{table}

Participants are informed about their treatment condition before making recommendations through both video and text instructions. The treatment is assigned at the individual level, allowing us to compare referral outcomes across conditions.

\subsection{Belief elicitation}
We elicit incentivized beliefs at two points in the experiment. First, before making referrals, participants report their beliefs about their own percentile ranking in each test area. Second, after making each referral, participants report their beliefs about their recommendedation's percentile ranking. For both belief elicitation tasks, participants earn \$5 if their guess is within 7 percentiles of the true value. This tolerance level is expected to balance precision with the difficulty of the task.

% participants report the closeness of their relationship with each recommended peer on a 0-10 scale. This allows us to examine whether social proximity influences referral decisions and accuracy. . Finally, participants estimate the distribution of university students across these SES groups, providing insight into their perceptions of the university-wide SES distribution.

\section{Sample, Incentives, and Procedure} \label{s4}
We invited all 4,417 UNAB undergraduate students who had at the time of recruitment completed their first year at the university to participate in our experiment. A total of 837 students took part in the data collection with a 19\% response rate. Our final sample consists of 734 individuals who referred peers with whom they have taken at least one class together, resulting in an 88\% success rate for the sample. We randomly allocated half of the participants into either \textbf{Baseline} or \textbf{Bonus} treatments. Table \ref{tab:balance} presents key demographic characteristics and academic performance indicators across treatments (see Appendix Table \ref{tab:selection} for selection). The sample is well-balanced between the \textbf{Baseline} and \textbf{Bonus} conditions and we observe no statistically significant differences in any of the reported variables (all \textit{p} values $> 0.1$). Our sample is characterized by a majority of middle-SES students with about one-tenth of the sample being high-SES students. The test scores and GPA distributions are balanced. On average, participants took 3.8 courses together with their network, and the average network consisted of 175 peers.   

\begin{table}[H]
  \centering
  \begin{threeparttable}
  \caption{Balance between treatments}
  \begin{tabular*}{\textwidth}{@{\extracolsep{\fill}}lcc c@{}}
  \toprule
   & \multicolumn{1}{c}{\textbf{Baseline}} & \multicolumn{1}{c}{\textbf{Bonus}} & \multicolumn{1}{c}{\textit{\textbf{p}}} \\ 
  \midrule
  Reading score & 64.712 & 65.693 & \multicolumn{1}{c}{0.134} \\
  Math score & 67.366 & 67.597 & \multicolumn{1}{c}{0.780} \\
  GPA & 4.003 & 4.021 & \multicolumn{1}{c}{0.445} \\
  Connections & 173.40 & 176.88 & \multicolumn{1}{c}{0.574} \\
  Courses taken & 3.939 & 3.719 & \multicolumn{1}{c}{0.443} \\
  Low-SES & 0.419 & 0.401 & \multicolumn{1}{c}{0.615} \\
  Middle-SES & 0.492 & 0.506 & \multicolumn{1}{c}{0.714} \\
  High-SES & 0.089 & 0.094 & \multicolumn{1}{c}{0.824} \\
  \midrule
  Observations & 382 & 352 & 734 \\
  \bottomrule
  \end{tabular*}
  \begin{tablenotes}[flushleft]
  \footnotesize
  \item[] \textit{Note:} This table presents balance tests between \textbf{Baseline} and \textbf{Bonus} conditions. $p$-values for binary outcomes are from two-sample tests of proportions; for continuous variables, from two-sample $t$-tests with unequal variances. All reported \textit{p}-values are two-tailed. Reading and math scores are in original scale units out of 100. GPA is grade point average out of 5. Connections refers to the average number of network members. Low-SES, Med-SES, and High-SES indicate SES categories based on strata.
  \end{tablenotes}
  \label{tab:balance}
  \end{threeparttable}
\end{table}

The experiment was conducted online through Qualtrics, with participants recruited from active UNAB students. To manage budget constraints while maintaining sufficient incentives, we randomly selected one in ten participants for payment. Selected participants received a fixed payment of \$17 for completion, plus potential earnings from one randomly selected belief question (up to \$5) and one randomly selected recommendation question (up to \$60), for maximum total earnings of \$82. The average time to complete the survey was 30 minutes, with an average compensation of \$80 for one in ten participants randomly selected for payment. Payment processing occured through online banking platform Nequi within 15 business days of participation.


\section{Results} \label{s5}

\subsection{Network characteristics}
We begin by describing the characteristic features of the ``enrollment network'' for all participants. This data set pairwise associates every participant in our sample with another university student if they have taken at least one course together at the time of the data collection. By doing so, we construct the entire referral choice set for participants. We include in this data set both the participant's and their potential candidate's individual characterics, as well as the number of common courses they have taken together. In Figure \ref{connections_class}, we describe the evolution of the enrollment network across the average number of network connections in network and the number of common courses taken with network members as participants progress through semesters.

\begin{figure}[H]
  \centering
  \caption{Network size and courses taken together by time spent at the university}\label{connections_class}
  \includegraphics[width=0.6\linewidth]{../figures/connections.png}
  \begin{tablenotes}
    \footnotesize
    \item[] \textit{Note:} This figure displays the average number of connections in blue and the average number of classes they have taken together with their connections in grey across semesters spent. The data shows an increase in the number of classes taken together as students progress in their programs, with the connections peaking around 7 semesters and dropping as certain students finish their bachelor's.
  \end{tablenotes}
\end{figure}

Are enrollment networks different across SES groups? We look at how the number of connections (network size) and number of courses taken together (tie strength) change across SES groups in Figure \ref{connections_by_ses}. Low- and middle-SES students have larger networks but take fewer courses together with network members, while high-SES students have smaller, ``denser'' networks. Specifically, both low- and middle-SES students have significantly larger networks than high-SES students ($t = 3.03, p = .003$ and $t = 2.49, p = .013$, respectively), but high-SES take significantly more courses with their network members than both low- ($t = -3.70, p < .001$) and middle-SES ($t = -4.20, p < .001$).

\begin{figure}[H]
  \centering
  \caption{Network size and courses taken together by SES}\label{connections_by_ses}
  \includegraphics[width=0.6\linewidth]{../figures/connections_by_ses.png}
  \begin{tablenotes}
    \footnotesize
    \item[] \textit{Note:} This figure displays the average number of connections and the average number of classes taken together across SES groups. The data shows a decrease in the number of connections with SES, and an associated increase in the number of classes taken together.
  \end{tablenotes}
\end{figure}

\subsection{SES diversity in networks}\label{s:ses-diversity}
What are diversity related consequences of SES-driven differences across networks? In terms of network compositions, SES groups may connect at different rates with other SES groups than at random (Figure \ref{availability_lses}).\footnote{Because we estimate the share of SES groups in every individual network, we get very precise estimates of the actual means. However, it is important to note that these are not independent observations for each network. Estimates are precise because each network is a draw with replacement from the same pool of university population, from which we calculate the proportion of SES groups per individual network, and take the average over an SES group. Pooling over SES groups who are connected with similar others systematically reduces variance (similar to resampling in bootstrapping). For this reason we choose not reporting test results in certain sections including this one and focus on describing the relationships between SES groups.} Our results reveal modest deviations from university-wide SES composition across groups. Low-SES students have networks with 38.4\% low-SES peers compared to the university average of 34.3\%, middle-SES students connect with 52.9\% middle-SES peers versus the university average of 50.5\%, and high-SES students show 20.4\% high-SES connections compared to the university average of 15.3\%.

\begin{figure}[H]
  \centering
  \caption{Network shares of SES groups}\label{availability_lses}
  \includegraphics[width=0.6\linewidth]{../figures/availability_lses.png}
  \begin{tablenotes}
    \footnotesize
    \item[] \textit{Note:} This figure compares the network shares of SES groups in the networks of low-, middle-, and high-SES participants. Horizontal lines plot the university-wide shares of each SES group (Low: 35\%, Mid: 51\%, High: 14\%). While the share of low-SES peers in the network decreases as the SES of the group increases, the share of high-SES peers in the network increases.
  \end{tablenotes}
\end{figure}

At the same time, we observe much larger differences between SES groups in how they connect on average with others. Low-SES students connect with other low-SES students at higher rates than middle-SES students (38.4\% vs 31.4\%) and high-SES students (38.4\% vs 25.1\%). Conversely, high-SES students connect more with other high-SES students than both low-SES students (20.4\% vs 12.6\%) and middle-SES students (20.4\% vs 15.8\%). Middle-SES students are in between the two extreme patterns, connecting with middle-SES peers at higher rates than low-SES students (52.9\% vs 49.0\%) but lower rates than high-SES students (52.9\% vs 54.5\%). These findings indicate SES-based network segregation, with same-SES homophily patterns across groups.

So far we have looked at the entire network without considering the intensity of connections between students. In our network data set, this variable amounts to the number of classes taken together with peers. As we will see in the next section, referrals go to peers with whom participants have taken on average 14 courses with, implying the intensity of the connection matters. We begin by dissecting what the intensity means in our context. As students take more courses together, the proportion of peers from the same academic program quickly goes beyond 95\% (see Figure \ref{share_same_program_by_tie}). Similarly, the average network size drops very quickly from above 210 to below 50 (see Figure \ref{conns_by_tie}). Both results indicate that actual referral considerations originate from a much smaller pool of individuals from the same academic program.

\begin{figure}[H]
  \centering
  \caption{Network characteristics and courses taken together}
  \begin{subfigure}[t]{0.49\textwidth}
    \centering
      \includegraphics[width=\linewidth]{../figures/share_same_program_by_tie.png}
      \caption{Same-program share}  
      \label{share_same_program_by_tie}
  \end{subfigure}
  \begin{subfigure}[t]{0.49\textwidth}
    \centering
      \includegraphics[width=\linewidth]{../figures/conns_by_tie.png}
      \caption{Network size}
      \label{conns_by_tie}
  \end{subfigure}
  \caption*{\footnotesize\textit{Note:} The left panel illustrates the share of connections within the same program as a function of the number of courses taken together. The right panel shows the average network size as a function of the number of courses taken together. Taking more than 5 courses together with a network member means on average 90\% chance to be in the same program. Similarly, past 5 courses together, the average network size dwindles by 80\%, from more than 210 individuals to below 50. 
  }
\end{figure}

What are the diversity implications of increasing the intensity of connections between students? As students take more courses together with peers, the share of same-SES peers in the networks of low- and high-SES increases while the share of middle-SES declines (see Figure \ref{share_same_ses_by_tie}). Both increases are substantial, amounting to 50\% for high-, and 30\% for low-SES. Combining these with the earlier result that beyond 5 courses taken together network members are almost entirely within the same program, these suggest program selection may have strong consequences for SES diversity in our setting.

\begin{figure}[H]
  \centering
  \caption{Network size and courses taken together by courses taken}\label{share_same_ses_by_tie}
  \includegraphics[width=0.6\linewidth]{../figures/share_same_ses_by_tie.png}
  \begin{tablenotes}
    \footnotesize
    \item[] \textit{Note:} This figure illustrates the shares of same-SES connections for low-, middle-, and high-SES as a function of the average number of courses taken together with network members. Low- and high-SES networks both become more homogenous as the average number of courses taken together with their connections increase.
  \end{tablenotes}
\end{figure}

\subsection{Program selection and SES diversity}
Academic programs at this university are priced based on how much they cost, and typically less than 5\% of students receive any kind of scholarship \citep{diaz_peer_2025}. Based on these, we first calculate how much every program at the university is expected to cost students per year (see Figure \ref{fees_with_boxplot}). Considering that net minimum montly wage stands at \$200 and the average Colombian salary around \$350, the cost difference between programs are large enough to make an impact of program selection. Is it the case that SES groups select into programs with financial considerations?

\begin{figure}[H]
  \centering
  \caption{Programs sorted by fee}\label{fees_with_boxplot}
  \includegraphics[width=0.6\linewidth]{../figures/fees_with_boxplot.png}
  \begin{tablenotes}
    \footnotesize
    \item[] \textit{Note:} This figure illustrates the distribution of programs at the university by their average yearly fee. The average yearly fee stands at \$3000, and medicine is an outlier at \$6000.
  \end{tablenotes}
\end{figure}

We look at how SES groups are distributed across programs to see evidence of SES-based selection (see Figure \ref{ses_distribution_by_fees}). Indeed, low-SES students select into more affordable programs, followed by middle-SES students. High-SES students sort almost exclusively into above-average costing programs, with a third selecting into medicine and creating a very skewed distribution. The distributions are significantly different across all pairwise comparisons: low-SES vs. middle-SES (Kolmogorov-Smirnov test $D=33.89$, $p<0.001$), low-SES vs. high-SES ($D=31.31$, $p<0.001$), and middle-SES vs. high-SES ($D=31.31$, $p<0.001$). With this finding, program selection could be the reason why low- and high-SES networks tend to segregate as the number of courses taken increases. The next section characterizes the referrals, and we will return to the diversity implications of program selection once we propose an understanding of how referrals were made.

\begin{figure}[H]
  \centering
  \caption{Programs sorted by fee}\label{ses_distribution_by_fees}
  \includegraphics[width=0.6\linewidth]{../figures/ses_distribution_by_fees.png}
  \begin{tablenotes}
    \footnotesize
    \item[] \textit{Note:} This figure illustrates the distribution of each SES group across programs sorted by fee. the majority of low-SES select into programs with below average cost, while high-SES select into programs with above average cost. Medicine accounts for a third of all high-SES students at this university.
  \end{tablenotes}
\end{figure}

\subsection{Characterizing referrals}

We observe 1342 referrals from our 734 participants in our final data set. More than 90\% of these consist from participants referring for both areas of the national entry exam (see Appendix Table \ref{tab:referral_distribution}). While participants made one referral for Math and Reading parts of the exam, about 70\% of these referrals went to two separate individuals. We compare the outcomes across areas for unique referrals in Appendix Table \ref{tab:referral_by_area} and all referrals in Appendix Table \ref{tab:referral_area}. In both cases, we find no meaningful differences between referrals made Math or Reading areas of the entry exam. As referrals in both exam areas come from the same referrer network, we pool referrals per  participant and report their average in terms of outcomes in our main analysis to avoid unintentionally increasing the statistical power of our tests when making comparisions. 

What are the characteristics of the individuals who receive referrals, and how do they compare to others in the enrollment network? Because we have an entire pool of potential candidates with one referral chosen from it, we compare the distributions for our variables of interest between the referred and non-referred students. First, referrals go to peers with whom the referrer has taken around 14 courses with on average, compared to almost 4 on average with others in their network (see Figure \ref{tie_hist}). This difference of 10.1 courses is significant ($t = 34.98$, $p < 0.001$), indicating that referrers choose individuals with whom they have stronger ties. While the median referral recipient has taken 12 courses together with the referrer, the median network member has shared only 2.8 courses. The interquartile range for referrals spans from 7.5 to 19.5 courses, compared to just 2.1 to 4.0 courses for the broader network, highlighting the concentration of referrals among peers with high social proximity and within same program (93\%).

\begin{figure}[H]
  \centering
  \caption{Courses taken together with network members and referrals}\label{tie_hist}
  \includegraphics[width=0.6\linewidth]{../figures/tie_hist.png}
  \begin{tablenotes}
    \footnotesize
    \item[] \textit{Note:} This figure compares the distributions of the number of courses taken together between referrers and their network members (orange) versus referrers and their chosen referral recipients (dark blue) for all 734 participants. 75\% of referral recipients having taken more than 7.5 courses together with the referrer, compared to only 25\% of network members. The distributions are significantly different (Kolmogorov-Smirnov test $D = 33.37$, $p < 0.001$).
  \end{tablenotes}
\end{figure}

Second, we examine entry exam score differences between referred students and the broader network. Referrals go to peers with an average score of 69.5 points, compared to 64.5 points for other network members (see Figure \ref{other_score_hist}). This difference of 5 points is significant ($t = 18.97$, $p < 0.001$), indicating that referrers choose higher-performing peers. While the median referral recipient scores 71 points, the median network member scores 65.1 points. The interquartile range for referrals spans from 65.5 to 75 points, compared to 63.5 to 66.9 points for the broader network, highlighting the clear concentration of referrals among higher performing peers.

\begin{figure}[H]
  \centering
  \caption{Entry exam scores of network members and referrals}\label{other_score_hist}
  \includegraphics[width=0.6\linewidth]{../figures/other_score_hist.png}
  \begin{tablenotes}
    \footnotesize
    \item[] \textit{Note:} This figure compares the distributions of entry exam scores (Math and Reading average) between referrers' network members (orange) versus their chosen referral recipients (dark blue) for all 734 participants. 75\% of referral recipients score above 65.5 points compared to only 25\% of network members scoring above 66.9 points. The distributions are significantly different (Kolmogorov-Smirnov test $D = 71.16$, $p < 0.001$). 
  \end{tablenotes}
\end{figure}

\subsection{Effect of the Bonus treatment}

Do referred individuals have different outcomes across treatments? We compare the performance, number of courses taken together, and SES shares of referred individuals between the \textbf{Baseline} and \textbf{Bonus} treatments in Table \ref{tab:referral_by_treatment}. While performance of referrals across Reading, Math, and GPA are similar across treatments, middle- and high-SES shares have significant differences. We find that referrals under the \textbf{Bonus} condition referred a higher proportion of high-SES individuals (13.5\% vs 8.8\%, $p=0.041$) and a lower proportion of middle-SES individuals on average (47.0\% vs 53.7\%, $p=0.072$). However, these differences do not appear to stem from systematic behavioral changes by any particular SES group of referrers, and the overall patterns remain largely consistent across treatments. The similarities in academic performance and number of courses taken together suggest that the core selection criteria, i.e., academic merit and social proximity, remain unchanged between conditions. For this reason, in the remainder of the paper, we report pooled results combining the averages of referral outcomes across treatments.

\begin{table}[H]
  \centering
  \begin{threeparttable}
  \caption{Characteristics of referrals by treatment condition}
  \begin{tabular*}{\textwidth}{@{\extracolsep{\fill}}lccc}
  \toprule
  & \multicolumn{1}{c}{\textbf{Baseline Referred}} & \multicolumn{1}{c}{\textbf{Bonus Referred}} & \multicolumn{1}{c}{\textbf{\textit{p}}} \\
  \midrule
  Reading score & 67.806 & 67.210 & 0.308 \\
  Math score & 70.784 & 70.155 & 0.406 \\
  GPA & 4.155 & 4.149 & 0.799 \\
  Courses taken & 13.840 & 14.065 & 0.723 \\
  Low-SES & 0.376 & 0.395 & 0.593 \\
  Middle-SES & 0.537 & 0.470 & 0.072 \\
  High-SES & 0.088 & 0.135 & 0.041 \\
  \midrule
  Observations & 382 & 352 \\
  \bottomrule
  \end{tabular*}
  \begin{tablenotes}[flushleft]
  \footnotesize
  \item[] \textit{Note:} This table compares the characteristics of network members who were referred under baseline vs bonus treatment conditions. $p$-values for binary outcomes are from two-sample tests of proportions; for continuous variables, from two-sample $t$-tests with unequal variances. All reported \textit{p}-values are two-tailed. Reading and math scores are raw test scores out of 100. GPA is grade point average out of 5. Courses taken is the number of courses participant has taken  with their referral. Low-SES, Med-SES, and High-SES are binary variables indicating the share of participants in estrato 1-2, 3-4, or 5-6, respectively. Both columns include only network members who were actually nominated for referral in each treatment condition.
  \end{tablenotes}
  \label{tab:referral_by_treatment}
  \end{threeparttable}
\end{table}

\subsection{Referral SES composition}
We first examine the overall SES-compositions in referral selection. Referrals to low-SES peers constitute 37.9\% of all referrals, compared to 33.7\% low-SES representation in individual networks (see Figure \ref{fig:all_referral_rates}). This represents a modest over-representation of 4.3 percentage points. For middle-SES students, referrals constitute 51.0\% versus 51.4\% network representation, showing virtually no difference (-0.5 pp.). High-SES referrals account for 11.1\% compared to 14.9\% network share, an under-representation of 3.8 percentage points. While these patterns suggest some deviation from proportional representation - with slight over-referral to low-SES peers and under-referral to high-SES peers - the magnitudes are relatively modest. Overall, referral compositions are largely balanced and closely mirror the underlying network structure, with the largest deviation being less than 5 percentage points for any SES group.

\begin{figure}[H]
  \centering
  \caption{Referral patterns compared to network composition}
  \label{fig:all_referral_rates}
  \includegraphics[width=0.6\linewidth]{../figures/all_referral_rates.png}
  \begin{tablenotes}
    \footnotesize
    \item[] \textit{Note:} This figure compares the average SES composition of referrers' networks (dark gray) to the SES composition of referrals (light gray). Error bars represent 95\% confidence intervals.
  \end{tablenotes}
\end{figure}

Then, we examine referral patterns by referrer SES to identify potential SES biases across groups. Figure \ref{fig:all_ses_referral_rates} reveals mixed patterns of deviation from network composition that vary by referrer SES. Most patterns show modest deviations from network composition, with differences typically ranging from 1-6 percentage points. However, at the very extremes, i.e., low-SES to high-SES connections and vice versa, we observe the the largest discrepancies between network share (which were already biased toward same-SES connections to begin with) and referral rates. Low-SES referrers show the strongest same-SES preference, referring 12.9 percentage points more to low-SES students than their network composition would suggest, while under-referring to high-SES recipients by 6.3 percentage points. Conversely, high-SES referrers under-refer to low-SES students by 10.9 percentage points compared to their network composition. Middle-SES referrers show the most balanced patterns, with deviations generally under 3 percentage points across all recipient groups. These findings indicate that cross-SES referral patterns - particularly between the most socioeconomically distant groups - show the largest departures from network availability, suggesting that when SES differences are most pronounced, referral behavior diverges most from underlying network structure.

\begin{figure}[H]
  \centering
  \caption{Referral patterns by referrer SES compared to network composition}
  \label{fig:all_ses_referral_rates}
  \includegraphics[width=0.6\linewidth]{../figures/all_ses_referral_rates.png}
  \begin{tablenotes}
    \footnotesize
    \item[] \textit{Note:} This figure compares the average SES composition of referrers' networks (dark gray) to the SES composition of referrals (light gray) for each SES group. The panels show referral patterns for low-SES (left), middle-SES (center), and high-SES referrers (right). Error bars represent 95\% confidence intervals.
  \end{tablenotes}
\end{figure}




% \begin{table}[H]
%   \centering
%   \begin{threeparttable}
%   \caption{Summary statistics for network members by referral status}
%   \begin{tabular*}{\textwidth}{@{\extracolsep{\fill}}lccccc}
%   \toprule
%   & \multicolumn{2}{c}{Verbal} & \multicolumn{2}{c}{Math} \\
%   \cmidrule{2-3} \cmidrule{4-5}
%   & Not Referred & Referred & Not Referred & Referred \\
%   \midrule
%   Reading z-score & 0.070 & 0.509 & 0.079 & 0.465 \\
%   & (0.003) & (0.039) & (0.003) & (0.040) \\
%   [1ex]
%   Math z-score & 0.079 & 0.452 & 0.087 & 0.590 \\
%   & (0.003) & (0.042) & (0.003) & (0.043) \\
%   [1ex]
%   GPA z-score & -0.066 & 0.705 & -0.069 & 0.711 \\
%   & (0.003) & (0.041) & (0.003) & (0.041) \\
%   [1ex]
%   Courses taken z-score & -0.153 & 2.690 & -0.184 & 2.488 \\
%   & (0.003) & (0.091) & (0.003) & (0.090) \\
%   [1ex]
%   Low-SES & 0.334 & 0.374 & 0.338 & 0.384 \\
%   & (0.001) & (0.019) & (0.001) & (0.019) \\
%   [1ex]
%   Med-SES & 0.515 & 0.513 & 0.513 & 0.507 \\
%   & (0.001) & (0.019) & (0.001) & (0.019) \\
%   [1ex]
%   High-SES & 0.151 & 0.113 & 0.149 & 0.109 \\
%   & (0.001) & (0.012) & (0.001) & (0.012) \\
%   % [1ex]
%   % Female & 0.524 & 0.562 & 0.520 & 0.428 \\
%   % & (0.001) & (0.019) & (0.001) & (0.019) \\
%   % [1ex]
%   % Age & 20.945 & 20.501 & 20.961 & 20.490 \\
%   % & (0.006) & (0.104) & (0.006) & (0.099) \\
%   \midrule
%   Observations & 128,174 & 673 & 127,481 & 669 \\
%   \bottomrule
%   \end{tabular*}
%   \begin{tablenotes}[flushleft]
%   \footnotesize
%   \item[] \textit{Note:} Standard errors in parentheses. GPA, test scores, and tie strength are standardized at the network level. For each referrer's network, we first calculated the mean and standard deviation of each measure. We then computed the average of these means and standard deviations across all referrers. Each individual's score was standardized using these network-level statistics. The standardization formula is $z = (x - \bar{x}_{network}) / \sigma_{network}$, where $\bar{x}_{network}$ and $\sigma_{network}$ are the average of network means and standard deviations, respectively. Low-SES, Med-SES, and High-SES are binary variables indicating the share of participants in estrato 1 and 2, 3 and 4, or 5 and 6, respectively. Tie strength measures the number of connections between individuals.
%   \end{tablenotes}
%   \label{tab:stats_by_nomination}
%   \end{threeparttable}
% \end{table}

\subsection{Ex post referral choice sets}

We now shed more light on the referral behavior after having characterized how referrals were made. Particularly interesting is that referrals go to peers with whom the median participant took 12 courses, with an average of 14. By restricting the networks for courses taken above the median, we can get a snapshot of how the referral choice set actually looked for participants before making referral decisions. As discussed in Section \ref{s:ses-diversity}, taking more courses with network members increases the share of same-SES individuals for both low- and high-SES students, and we had explored program selection as a potential mechanism. In Figure \ref{t15availability}, we show the effects of network segregation on \textit{ex post} referral choice sets for each SES group. Network compositions above the median number of courses taken reveal strong segregation effects: Low-SES networks contain 44.5\% low-SES peers, higher than the 35\% university-wide share by 9.5 percentage points. Conversely, high-SES are  under-represented in low-SES networks at only 8.6\% average share, compared to the 14\% population share ($-5.4$ pp.). At the other extreme, high-SES networks show the reverse pattern with average low-SES share dropping to just 15.7\%, a 19.3 percentage point decrease relative to the university average. High-SES students have a same-SES concentration at 26.5\%, doubling their 14\% population share ($+12.5$ pp.). Middle-SES networks remain relatively balanced and closely track population proportions across all SES groups. Taken together, these suggest observed referral rates of SES groups may follow the network compostions above median number of courses taken together. We will test this formally by setting up a choice model where we can take into account individual differences in network compositions across SES, and try to identify SES biases that go beyond SES groups' availability in the choice sets.

\begin{figure}[H]
  \centering
  \caption{Network size and courses taken together by courses taken}\label{t15availability}
  \includegraphics[width=0.6\linewidth]{../figures/t15availability.png}
  \begin{tablenotes}
    \footnotesize
    \item[] \textit{Note:} This figure compares the network shares of SES groups in the networks of low-, middle-, and high-SES participants above the median number of courses taken together with peers. Horizontal lines plot the university-wide shares of each SES group (Low: 35\%, Mid: 51\%, High: 14\%). Low- and high-SES networks both become same-SES dominated at the expense of each other while middle-SES networks remain balanced. Error bars represent 95\% confidence intervals.
  \end{tablenotes}
\end{figure}


\subsection{Identifying the SES bias in referrals}

We model a single referral outcome from candidates that are mutually exclusive, where our the dependent variable outcome is multinomial distributed. Our design leverages the enrollment network to generate dataset which includes alternative-specific variables for each referral decision, i.e., SES, courses taken together with the participant making the referral, as well as entry exam scores for not just the chosen alternative but all referral candidates. Using a conditional logit model on these data, we can identify whether an SES group has an aggregate bias controlling for each individual's unique enrollment network composition.

We follow an additive random utility model framework where individual $i$ and alternative $j$ have utility $U_{ij}$ that is the sum of a deterministic component, $V_{ij}$, that depends on regressors and unknown parameters, and an unobserved random component $\varepsilon_{ij}$:

We observe the outcome $y_i = j$ if alternative $j$ has the highest utility of the alternatives. The probability that the outcome for individual $i$ is alternative $j$, conditional on the regressors, is:

\begin{align}
p_{ij} = \Pr(y_i = j) = \Pr(U_{ij} \geq U_{ik}), \quad \text{for all } k
\end{align}

The CL model specifies that the probability of individual $i$ choosing alternative $j$ from choice set $C_i$ is given by:

\begin{align}
p_{ij} = \frac{\exp(x'_{ij} \beta)}{\sum_{l \in C_i} \exp(x'_{il} \beta)}, \quad j \in C_i
\end{align}

where $x_{ij}$ are alternative-specific regressors, i.e., characteristics of potential referees that vary across alternatives.

In our context, individual $i$ chooses to refer candidate $j$ from their enrollment network $C_i$. The alternative-specific regressors include SES and entry exam scores of the referral candidate, and the number of courses taken together with the participant making the referral. Conditional logit structure eliminates participant-specific factors that might influence both network formation and referral decisions, allowing us to identify preferences within each participant's realized network.

For causal identification of SES bias, we require two identifying assumptions. Specifically:

\begin{enumerate}
    \item \textbf{Conditional exogeneity}: Both SES and the number of courses taken together could be endogenous due to program selection. High-SES students sort into expensive programs while low-SES students choose affordable programs, creating systematic SES variation across enrollment networks. Similarly, the number of courses taken together reflects program selection decisions that may correlate with unobserved referral preferences. However, conditional on the realized enrollment network, the remaining variation in both SES and the number of courses taken together across referral candidates must be independent of unobserved factors affecting referral decisions. In the robustness checks, we show that being in the same program with the referrer does not impact our SES bias estimates, although it reduces the coefficient on the number of courses taken together.
    
    \item \textbf{Complete choice set specification and independence of irrelevant alternatives}: Administrative data captures the complete enrollment network, with all peers who took at least one course with individual $i$ and represent the true choice set for referral decisions (unless participants have potential referral candidates with whom they never took classes). The independence of irrelevant alternatives (IIA) assumption requires that choices between any two alternatives be independent of other options in the choice set, which could be problematic if, e.g., peers within the same SES group are viewed as close substitutes. This concern does not apply to our setting because the design of our experiment ensures that choice sets are fixed by enrollment rather than arbitrary inclusion/exclusion of alternatives that create IIA violations.
\end{enumerate}

Under these assumptions, the conditional logit framework controls for individual heterogeneity in program selection (absorbed by conditioning on choice sets), selection into programs based on observable characteristics (through alternative-specific variables), and choice set composition effects (through the multinomial structure). Therefore, $\beta$ should identify the causal effect of referral candidate SES on referral probability, holding constant the number of courses taken together and the entry exam scores of candidates. A significant coefficient will then indicate taste-based discrimination.

We pool participants by their SES group, and estimate the above described conditional fixed effects logit model once for low-, middle-, and high-SES referrers. We standardize entry exam scores and the number of courses taken together at the individual network level. For each referrer's network, we first calculate the mean and standard deviation for both measures. We then compute the average of these means and standard deviations across all 734 referrers. Each referral candidate's entry exam score and the number of courses they haven taken with the referrer is standardized using these network-level statistics. The standardization formula is $z_{i} = (x - \bar{X}_{i}) / \sigma_{i}$, where $\bar{X}_{i}$ and $\sigma_{i}$ are the average of network means and standard deviations for $C_{i}$.

We describe our first set of findings in Table \ref{tab:ses-heterogeneity}. To begin with, the variance explained by all three models are extremely low, suggesting the role of potential SES biases in referrals that go beyond the network structure must be limited. Regardless, controlling for network composition, low-SES participants are more likely to refer other low-SES, and are less likely to refer high-SES relative to the probability of referring middle-SES peers. In contrast, we find that high-SES participants are less likely to refer other low-SES, relative to the probability of referring middle-SES peers.

\begin{table}[H]
\centering
\begin{threeparttable}
\caption{SES bias in referral decisions by referrer SES group}\label{tab:ses-heterogeneity}
\begin{tabular*}{0.85\textwidth}{@{\extracolsep{\fill}}l c c c@{}}
\toprule
& \multicolumn{3}{c}{Referrer SES} \\
& Low & Middle & High \\
& (1) & (2) & (3) \\
\midrule
Low-SES candidate & 0.453*** & -0.019 & -0.710** \\
& (0.109) & (0.098) & (0.333) \\
High-SES candidate & -0.584*** & -0.255* & 0.001 \\
& (0.211) & (0.145) & (0.261) \\
\midrule
$\chi^2$ & 33.47 & 3.18 & 4.94 \\
Observations & 110,142 & 127,088 & 19,767 \\
Individuals & 301 & 366 & 67 \\
\bottomrule
\end{tabular*}
\begin{tablenotes}[flushleft]
\footnotesize
\item[] \textit{Note:} Individual-level clustered standard errors in parentheses. * $p < 0.1$, ** $p < 0.05$, *** $p < 0.01$. Each column represents a separate conditional logit regression estimated on the subsample of referrers from the indicated SES group. Coefficients represent log-odds of referring candidates from the specified SES group relative to referring middle-SES candidates. All models include individual fixed effects that control for each referrer's choice set composition.
\end{tablenotes}
\end{threeparttable}
\end{table}

We proceed by adding the standardized number of courses taken together as a control in our specification and describe the results in Table \ref{tab:ses-heterogeneity-extended}. A one standard deviation increase in the number of courses taken together proves to be highly significant across all models, with coefficients ranging from 0.856 to 1.049, indicating that stronger social connections substantially increase the probability of referral. The high $\chi^2$ statistics suggest that these models explain considerably more variance than specifications without this control, highlighting the importance of courses taken together in referral decisions. Nevertheless, low-SES participants still show a strong same-SES bias relative to referring middle-SES peers at the average number of courses taken together. This same-SES bias is not observed among middle-SES or high-SES referrers, who also display no statistically significant bias toward low-SES candidates. None referrer group shows a positive bias for high-SES candidates relative to middle-SES candidates. 

\begin{table}[H]
\centering
\begin{threeparttable}
\caption{SES bias in referral decisions by referrer SES group}\label{tab:ses-heterogeneity-extended}
\begin{tabular*}{0.85\textwidth}{@{\extracolsep{\fill}}l c c c@{}}
\toprule
& \multicolumn{3}{c}{Referrer SES} \\
& Low & Middle & High \\
& (1) & (2) & (3) \\
\midrule
Low-SES candidate & 0.348*** & -0.064 & -0.489 \\
& (0.123) & (0.115) & (0.337) \\
High-SES candidate & -0.366 & -0.165 & -0.140 \\
& (0.223) & (0.157) & (0.286) \\
Courses taken (z-score) & 0.856*** & 0.931*** & 1.049*** \\
& (0.035) & (0.037) & (0.126) \\
\midrule
$\chi^2$ & 626.15 & 636.10 & 71.43 \\
Observations & 110,142 & 127,088 & 19,767 \\
Individuals & 301 & 366 & 67 \\
\bottomrule
\end{tabular*}
\begin{tablenotes}[flushleft]
\footnotesize
\item[] \textit{Note:} Individual-level clustered standard errors in parentheses. * $p < 0.1$, ** $p < 0.05$, *** $p < 0.01$. Each column represents a separate conditional logit regression estimated on the subsample of referrers from the indicated SES group. Coefficients represent log-odds of referring candidates from the specified SES group relative to referring middle-SES candidates. All models include individual fixed effects that control for each referrer's choice set composition.
\end{tablenotes}
\end{threeparttable}
\end{table}

We add standardized entry exam scores (Math and Reading average) as a second control variable and describe our results in Table \ref{tab:ses-heterogeneity-scores}. A one standard deviation increase in the entry exam score proves highly significant across all models, with coefficients ranging from 0.587 to 0.883. This shows merit-based considerations due to the incentive structure of the experiment remained central to referral decisions. The slightly higher $\chi^2$ statistics compared to the earlier specification suggests that entry exam scores improve model fit. The inclusion of standardized entry exam scores strengthens SES biases. Low-SES referrers maintain their same-SES bias, with now a significant negative bias against high-SES . Middle-SES referrers, previously showing no SES bias, now show marginal negative bias agaisnt high-SES. Finally, high-SES referrers exhibit marginal negative bias against low-SES candidates. 

The evidence of a biases becoming significant when controlling for entry exam scores has a nuanced interpretation. While at the university-level, low-SES typically score lower in the entry exam, low-SES students appearing in high-SES networks are positively selected, scoring about 0.14 standard deviations higher than middle-SES students. Controlling for performance thus removes this positive selection and reveals the ``pure'' SES bias that was previously underestimated by above average performance of low-SES. Vice versa, high-SES in low-SES networks perform 0.12 standard deviations better than middle-SES students. The same bias was underestimated as high-SES candidates' better performance relative to middle-SES in the same networks provided a meritocratic justification for getting more referrals. Controlling for exam scores reveal that both high- and low-SES referrers have negative SES bias towards one another that operates independently of - and counter to - performance-based considerations. What makes interpretation difficult is that while biased against low-SES, high-SES referrers do not under any specification display a positive bias towards their in-group. For this final reason, we do not dig any further in this direction.\textcolor{red}{add tests if you want} 

To conclude, we conduct joint significance tests, testing whether low- and high-SES regression coefficients are jointly different from middle-SES for each regression specificaiton. For low-SES referrers, the joint test remains highly significant across all three specifications ($\chi^2 = 10.20$, $p = 0.006$ in the final model), indicating persistent SES bias across all specifications. In contrast, middle-SES referrers display no significant joint SES bias in any specification, with the test becoming increasingly non-significant as controls are added ($\chi^2 = 4.13$, $p = 0.127$ in the final model). High-SES referrers similarly show no significant joint SES bias across all three models ($\chi^2 = 4.28$, $p = 0.118$ in the final model). These results suggest that SES bias in referrals is primarily driven by low-SES. There is no sufficient evidence to conclude that middle- and high-SES referrers systematically discriminate against other-SES peers once we take into account the large differences in their network compositions due to program selection.


\begin{table}[H]
\centering
\begin{threeparttable}
\caption{SES bias in referral decisions by referrer SES group with academic performance controls}\label{tab:ses-heterogeneity-scores}
\begin{tabular*}{0.85\textwidth}{@{\extracolsep{\fill}}l c c c@{}}
\toprule
& \multicolumn{3}{c}{Referrer SES} \\
& Low & Middle & High \\
& (1) & (2) & (3) \\
\midrule
Low-SES candidate & 0.242** & -0.159 & -0.600* \\
& (0.123) & (0.114) & (0.327) \\
High-SES candidate& -0.445** & -0.274* & -0.345 \\
& (0.222) & (0.157) & (0.287) \\
Courses taken (z-score) & 0.859*** & 0.948*** & 1.043*** \\
& (0.036) & (0.038) & (0.118) \\
Entry exam (candidate z-score) & 0.607*** & 0.587*** & 0.883*** \\
& (0.052) & (0.047) & (0.111) \\
\midrule
$\chi^2$ & 789.87 & 756.06 & 120.54 \\
Observations & 110,142 & 127,088 & 19,767 \\
Individuals & 301 & 366 & 67 \\
\bottomrule
\end{tabular*}
\begin{tablenotes}[flushleft]
\footnotesize
\item[] \textit{Note:} Individual-level clustered standard errors in parentheses. * $p < 0.1$, ** $p < 0.05$, *** $p < 0.01$. Each column represents a separate conditional logit regression estimated on the subsample of referrers from the indicated SES group. Coefficients represent log-odds of referring candidates from the specified SES group relative to referring middle-SES candidates. All models include individual fixed effects that control for each referrer's choice set composition.
\end{tablenotes}
\end{threeparttable}
\end{table}



% \begin{table}[H]
%   \centering
%   \begin{threeparttable}
%   \caption{Comparison of math and verbal scores by SES group and data source}
%   \begin{tabular*}{\textwidth}{@{\extracolsep{\fill}}lcccccc}
%   \toprule
%   & \multicolumn{3}{c}{Math} & \multicolumn{3}{c}{Verbal} \\
%   \cmidrule{2-4} \cmidrule{5-7}
%   & Network & Admin & Sample & Network & Admin & Sample \\
%   \midrule
%   Low-SES & 66.976 & 61.653 & 67.813 & 64.738 & 60.974 & 66.058 \\
%   & (0.052) & (0.346) & (0.694) & (0.043) & (0.274) & (0.574) \\
%   [1ex]
%   Mid-SES & 65.627 & 64.531 & 66.859 & 63.685 & 63.154 & 64.779 \\
%   & (0.039) & (0.224) & (0.580) & (0.032) & (0.183) & (0.436) \\
%   [1ex]
%   High-SES & 67.781 & 67.330 & 70.610 & 64.966 & 64.892 & 66.397 \\
%   & (0.077) & (0.416) & (1.295) & (0.063) & (0.341) & (1.214) \\
%   \midrule
%   Observations & 128,150 & 4,415 & 669 & 128,847 & 4,403 & 673 \\
%   \bottomrule
%   \end{tabular*}
%   \begin{tablenotes}[flushleft]
%   \footnotesize
%   \item[] \textit{Note:} Standard errors in parentheses. The table presents mean scores with standard errors for math and verbal tests across the entire network, the admin data, and the sample. Admin data consistently shows lower scores than both network and the sample across all SES groups consistent with selection, with the largest gaps occurring for the Low-SES. Differences between network and sample scores are generally smaller than those between either and the admin data.
%   \end{tablenotes}
%   \label{tab:ses_scores_comparison}
%   \end{threeparttable}
% \end{table}



% \begin{figure}[H]
%   \centering
%   \caption{Effect of the Bonus on Referrals}
%   \begin{subfigure}[t]{0.7\textwidth}
%     \centering
%       \includegraphics[width=\linewidth]{/Users/reha.tuncer/Documents/GitHub/icfes-referrals/figures/reading_combined.png}
%       \caption{Reading}  
%       \label{fig:reading_combined}
%   \end{subfigure}
%   \begin{subfigure}[t]{0.7\textwidth}
%     \centering
%       \includegraphics[width=\linewidth]{/Users/reha.tuncer/Documents/GitHub/icfes-referrals/figures/math_combined.png}
%       \caption{Math}
%       \label{fig:math_combined}
%   \end{subfigure}
%   \label{fig:both_combined}
%   \caption*{\footnotesize\textit{Note:} The top panel compares the reading scores and tie strength of referrals across conditions. The bottom panel shows the average standardized math and tie strength of referrals across conditions. We test differences in across conditions using two-sample $t$-tests and find no meaningful differences. For both math and reading, treatment causes no significant changes in referral performance or tie strength.}
% \end{figure}


% \begin{figure}[H]
%   \centering
%   \caption{Top decile performer share across the sample, network and referals }\label{topdecile}
%   \includegraphics[width=0.6\linewidth]{/Users/reha.tuncer/Documents/GitHub/icfes-referrals/figures/top_decile_comparison.png}
%   \begin{tablenotes}
%     \footnotesize
%     \item[] \textit{Note:} This figure displays the percentage share of top decile individuals according to the admin data across three dimensions. First bar shows referrers in the sample of participants. Second bar is the share of top decile individuals in their networks. Third column shows the share of top decile among the referrals made. We test differences between proportions across these three groups using two-sample tests of proportions. For both math and reading scores, the differences between Sample and Network ($p<0.001$), Sample and Referrals ($p<0.005$), and Network and Referrals ($p<0.001$) are all statistically significant.
%   \end{tablenotes}
% \end{figure}


  % \begin{figure}[H]
  %   \centering
  %   \caption{Participant network performance by subject and SES}\label{performance}
  %   \includegraphics[width=0.6\linewidth]{/Users/reha.tuncer/Documents/GitHub/icfes-referrals/figures/ses_peer_performance.png}
  %   \begin{tablenotes}
  %     \footnotesize
  %     \item[] \textit{Note:} This figure displays the network average math and reading z-scores across referrer SES. We test differences between scores across SES using paired $t$-tests. For both math and reading scores, all differences between SES groups are statistically significant (all $p \leq 0.001$).
  %   \end{tablenotes}
  % \end{figure}

  % \begin{figure}[H]
  %   \centering
  %   \caption{Distribution of top decile performers by subject}\label{topdecile}
  %   \includegraphics[width=0.6\linewidth]{/Users/reha.tuncer/Documents/GitHub/icfes-referrals/figures/strata_scores.png}
  %   \begin{tablenotes}
  %     \footnotesize
  %     \item[] \textit{Note:} This figure displays the percentage of participants across SES who score in the top decile for math and reading. We test differences between math and reading proportions within each SES group using two-sample tests of proportions and find no statistically significant differences ($p > 0.28$). Middle-SES students show the highest representation in the top decile for both subjects in the sample.
  %   \end{tablenotes}
  % \end{figure}



%   \begin{figure}[H]
%     \centering
%     \caption{Participant network composition by SES}
%     \includegraphics[width=0.6\textwidth]{/Users/reha.tuncer/Documents/GitHub/icfes-referrals/figures/ses_distribution.png}
%     \label{fig:baseline_ses}
%     \caption*{\footnotesize\textit{Note:} This figure displays the composition of networks by SES. We test differences in proportions of peer connections across SES groups using two-sample tests of proportions. All differences are statistically significant ($p <$ 0.001): Low SES students are more likely to connect with Low SES peers than Middle or High SES students; Middle SES students form more connections with Middle SES peers than Low SES students; and High SES students have the highest proportion of High SES connections.
%     }
% \end{figure}

% \begin{figure}[H]
%   \centering
%   \caption{Participant network composition by SES}
%   \includegraphics[width=0.6\textwidth]{/Users/reha.tuncer/Documents/GitHub/icfes-referrals/figures/z_ses_tie_strength.png}
%   \label{fig:tie_ses}
%   \caption*{\footnotesize\textit{Note:} This figure displays the standardized tie strength by SES. We test differences in standardized tie strength across SES groups using two-sample $t$-tests. All differences are statistically significant ($p < 0.001$) except for the comparison between Middle and High SES students' connections to Low SES peers ($p = 0.65$). The standardized tie strength for High SES students with other High SES students is substantially positive (0.26), while all other tie strengths are negative or near zero.}
% \end{figure}

%   \begin{figure}[H]
%     \centering
%     \caption{Baseline Referral Patterns by SES}
%     \begin{subfigure}[t]{0.49\textwidth}
%         \centering
%         \includegraphics[width=\linewidth]{/Users/reha.tuncer/Documents/GitHub/icfes-referrals/figures/ses_tie_strength.png}
%         \caption{Referral Rates}
%         \label{fig:tie_ses}
%     \end{subfigure}
%     \hfill
%     \begin{subfigure}[t]{0.49\textwidth}
%         \centering
%         \includegraphics[width=\linewidth]{/Users/reha.tuncer/Documents/GitHub/icfes-referrals/figures/ses_top_tie_strength.png}
%         \caption{Academic Performance of Referrals}
%         \label{fig:tie_topses}
%     \end{subfigure}
%     \label{fig:tie_combinedses}
%     \caption*{\footnotesize\textit{Note:} The left panel shows the availability of SES groups across socioeconomic strata. The right panel shows the availability of top decile performers within each SES group.}
%   \end{figure}

\pagebreak


% \begin{figure}[H]
%   \centering
%   \caption{Participant network composition by SES}
%   \includegraphics[width=\textwidth]{/Users/reha.tuncer/Documents/GitHub/icfes-referrals/figures/combined_baseline.png}
%   \label{fig:baseline_combined}
%   \caption*{\footnotesize\textit{Note:} This figure displays the standardized tie strength by SES. We test differences in z-score tie strength across SES groups using two-sample $t$-tests. All differences are statistically significant ($p < 0.001$) except for the comparison between Middle and High SES students' connections to Low SES peers ($p = 0.65$). The standardized tie strength for High SES students with other High SES students is substantially positive (0.26), while all other tie strengths are negative or near zero. These patterns reveal strong homophily in network formation, particularly among High SES students, while Low SES students show moderate homophily patterns.}
% \end{figure}

  % \begin{figure}[H]
  %   \centering
  %   \caption{Baseline Referral Patterns by SES}
  %   \begin{subfigure}[t]{0.49\textwidth}
  %     \centering
  %       \includegraphics[width=\linewidth]{/Users/reha.tuncer/Documents/GitHub/icfes-referrals/figures/ses_referral_distribution.png}
  %       \caption{Referral Rates}  
  %       \label{fig:baseline_rates}
  %   \end{subfigure}
  %   \hfill
  %   \begin{subfigure}[t]{0.49\textwidth}
  %     \centering
  %       \includegraphics[width=\linewidth]{/Users/reha.tuncer/Documents/GitHub/icfes-referrals/figures/baseline_performance.png}
  %       \caption{Referral Performance}
  %       \label{fig:baseline_performance}
  %   \end{subfigure}
  %   \label{fig:baseline_combined}
  %   \caption*{\footnotesize\textit{Note:} The left panel shows the distribution of referrals across SES in the baseline condition. We test differences in SES shares across SES groups using two-sample tests of proportions. All differences are statistically significant ($p < 0.1$). The right panel shows the average standardized math and reading scores of referred students by referrer's SES. We test differences in z-scores across SES groups using two-sample $t$-tests and find no statistically significant differences in reading scores across SES groups (all $p > 0.36$). For math scores, we observe marginally significant differences between Low and High SES students ($p = 0.08$) and between Middle and High SES students ($p = 0.18$), with High SES referring peers with higher math performance.}
  % \end{figure}

%   \begin{figure}[H]
%     \centering
%     \caption{Effect of the Bonus}
%     \label{fig:treat_combined}
%     \begin{subfigure}[t]{0.49\textwidth}
%         \centering
%         \includegraphics[width=\linewidth]{/Users/reha.tuncer/Documents/GitHub/icfes-referrals/figures/ses_treatment_effects.png}
%         \caption{Changes in Referral Rates}
%         \label{fig:treat_rates}
%     \end{subfigure}
%     \hfill
%     \begin{subfigure}[t]{0.49\textwidth}
%         \centering
%         \includegraphics[width=\linewidth]{/Users/reha.tuncer/Documents/GitHub/icfes-referrals/figures/treatment_effect.png}
%         \caption{Changes in Referral Performance}
%         \label{fig:treat_performance}
%     \end{subfigure}
%     \caption*{\footnotesize\textit{Note:} The left panel shows the changes in referral rates across SES. We test differences in SES shares across conditions using two-sample tests of proportions. For Low-SES, only the change in referral share of Middle-SES is statistically significant ($p = 0.034$). For Middle-SES, only the change in referral share of High-SES is statistically significant ($p = 0.027$). For High-SES, only the change in referral share of Middle-SES is statistically significant ($p = 0.059$). The right panel shows the differences in math and reading z-scores across SES. We test differences in SES shares across conditions using two-sample $t$-tests. For both reading and math scores, the only statistically significant difference is in the reading scores for Low-SES ($p = 0.026$).}
% \end{figure}

% \begin{figure}[H]
%   \centering
%   \caption{Effect of the Bonus on Tie Strength}
%   \label{fig:treat_combined2}
%   \begin{subfigure}[t]{0.49\textwidth}
%       \centering
%       \includegraphics[width=\linewidth]{/Users/reha.tuncer/Documents/GitHub/icfes-referrals/figures/baseline_ties.png}
%       \caption{Changes in Referral Rates}
%       \label{fig:tietreat_rates}
%   \end{subfigure}
%   \hfill
%   \begin{subfigure}[t]{0.49\textwidth}
%       \centering
%       \includegraphics[width=\linewidth]{/Users/reha.tuncer/Documents/GitHub/icfes-referrals/figures/ties_treatment_effects.png}
%       \caption{Changes in Referral Performance}
%       \label{fig:tietreat_performance}
%   \end{subfigure}
%   \caption*{\footnotesize\textit{Note:} The left panel shows the changes in referral rates across socioeconomic strata (bonus minus baseline). The right panel shows the differences in average standardized math and reading scores of referred students by referrer's SES.}
% \end{figure}





% \begin{figure}[H]
%   \centering
%   \caption{Performance by Tie Strength and SES}
%   \includegraphics[width=0.85\textwidth]{/Users/reha.tuncer/Documents/GitHub/icfes-referrals/figures/tie_combined.png}
%   \label{fig:tie_score_ses}
%   \caption*{\footnotesize\textit{Note:} This figure shows local polynomial regressions of network math and reading z-scores by social tie strength across socioeconomic status groups with 95\% confidence intervals. Higher SES have steeper positive relationships between tie strength and the average performance those in their network across reading and math scores.}
% \end{figure}


    \section{Conclusion} \label{s6}

\pagebreak

\bibliographystyle{apacite}

\bibliography{referrals}    
\pagebreak

\appendix \label{appendix-all}
\renewcommand{\thefigure}{A.\arabic{figure}}
\setcounter{figure}{0}

\section{Additional Figures and Tables}

\subsection*{Additional Figures}
\pagebreak
\subsection*{Additional Tables}
\renewcommand{\thetable}{A.\arabic{table}}
\setcounter{table}{0}
\setcounter{figure}{0}

\begin{table}[H]
  \centering
  \begin{threeparttable}
  \caption{Selection into the experiment}
  \begin{tabular*}{\textwidth}{@{\extracolsep{\fill}}lcc c@{}}
  \toprule
  & \multicolumn{1}{c}{\textbf{University}} & \multicolumn{1}{c}{\textbf{Sample}} & \multicolumn{1}{c}{\textit{\textbf{p}}} \\
  \midrule
  Reading score & 62.651 & 65.183 & 0.000 \\
  Math score & 63.973 & 67.477 & 0.000 \\
  GPA & 3.958 & 4.012 & 0.000 \\
  Low-SES & 0.343 & 0.410 & 0.000 \\
  Middle-SES & 0.505 & 0.499 & 0.763 \\
  High-SES & 0.153 & 0.091 & 0.000 \\
  Female & 0.567 & 0.530 & 0.060 \\
  Age & 21.154 & 20.651 & 0.000 \\
  \midrule
  Observations & 4,417 & 734 &  \\
  \bottomrule
  \end{tabular*}
  \begin{tablenotes}[flushleft]
  \footnotesize
  \item[] \textit{Note:} This table compares characteristics between the university and the experimental sample. $p$-values for binary outcomes (Low-SES, Med-SES, High-SES, Female) are from two-sample tests of proportions; for continuous variables, from two-sample $t$-tests with unequal variances. All reported \textit{p}-values are two-tailed.
  \end{tablenotes}
  \label{tab:selection}
  \end{threeparttable}
\end{table}


\begin{table}[H]
  \centering
  \begin{threeparttable}
  \caption{Distribution of referrals by area}
  \begin{tabular*}{\textwidth}{@{\extracolsep{\fill}}lccc}
  \toprule
  \textbf{Area} & \textbf{Only one area} & \textbf{Both areas} & \textbf{Total} \\
  \midrule
  Verbal & 65 & 608 & 673 \\
  Math & 61 & 608 & 669 \\
  \midrule
  Total & 126 & 1,216 & 1,342 \\
  \bottomrule
  \end{tabular*}
  \begin{tablenotes}[flushleft]
  \footnotesize
  \item[] \textit{Note:} The table shows how many referrers made referrals in only one area versus both areas. ``Only one area'' indicates individuals who made referrals exclusively for one area of the exam. ``Both areas'' shows individuals who made referrals in both verbal and math areas. The majority of referrers (608) made referrals in both areas.
  \end{tablenotes}
  \label{tab:referral_distribution}
  \end{threeparttable}
\end{table}


\begin{table}[H]
  \centering
  \begin{threeparttable}
  \caption{Referral characteristics by exam area (unique referrals only)}
  \begin{tabular*}{\textwidth}{@{\extracolsep{\fill}}lccc}
  \toprule
  & \multicolumn{1}{c}{\textbf{Reading}} & \multicolumn{1}{c}{\textbf{Math}} & \multicolumn{1}{c}{\textbf{\textit{p}}} \\
  \midrule
  Reading score & 67.733 & 67.126 & 0.252 \\
  Math score & 69.339 & 71.151 & 0.008 \\
  GPA & 4.136 & 4.136 & 0.987 \\
  Courses taken & 13.916 & 13.019 & 0.123 \\
  Low-SES & 0.372 & 0.385 & 0.666 \\
  Med-SES & 0.526 & 0.518 & 0.801 \\
  High-SES & 0.103 & 0.097 & 0.781 \\
  \midrule
  Observations & 487 & 483 &  \\
  \bottomrule
  \end{tabular*}
  \begin{tablenotes}[flushleft]
  \footnotesize
  \item[] \textit{Note:} This table compares characteristics of uniquely referred students by entry exam area for the referral (verbal vs. math). $p$-values are from two-sample t-tests with unequal variances. Referrals in Math area go to peers with significantly higher math scores ($p = 0.008$), while we find no significant differences for Reading scores, GPA, courses taken, or SES composition for referrals across the two areas. Excluding referrals going to the same individuals does not change the outcomes for referrals compared to Appendix Table \ref{tab:referral_area}
  \end{tablenotes}
  \label{tab:referral_by_area}
  \end{threeparttable}
\end{table}

\begin{table}[H]
  \centering
  \begin{threeparttable}
  \caption{Referral characteristics by academic area}
  \begin{tabular*}{\textwidth}{@{\extracolsep{\fill}}lcc c@{}}
  \toprule
   & \multicolumn{1}{c}{\textbf{Reading}} & \multicolumn{1}{c}{\textbf{Math}} & \multicolumn{1}{c}{\textit{\textbf{p}}} \\ 
  \midrule
  Reading score & 67.85 & 67.41 & 0.348 \\
  Math score & 70.04 & 71.36 & 0.029 \\
  GPA & 4.153 & 4.153 & 0.984 \\ 
  Courses taken & 14.467 & 13.822 & 0.206 \\
  Low-SES & 37\% & 38\% & 0.714 \\
  Middle-SES & 51\% & 51\% & 0.829 \\
  High-SES & 11\% & 11\% & 0.824 \\
  \midrule
  Observations & 673 & 669 &  \\
  \bottomrule
  \end{tabular*}
  \begin{tablenotes}[flushleft]
  \footnotesize
  \item[] \textit{Note:} This table compares characteristics of referred students by entry exam area for the referral (verbal vs. math). $p$-values are from two-sample t-tests with unequal variances. Referrals in Math area go to peers with significantly higher math scores ($p = 0.029$), while we find no significant differences for Reading scores, GPA, courses taken, or SES composition for referrals across the two areas.
  \end{tablenotes}
  \label{tab:referral_area}
  \end{threeparttable}
  \end{table}

\pagebreak

\section{Experiment} \label{instructions}
\renewcommand{\thefigure}{B.\arabic{figure}}
\textit{We include the English version of the instructions used in Qualtrics. Participansts saw the Spanish version. Horizontal lines in the text indicate page breaks and clarifiying comments are inside brackets.} 

\section*{Consent}
You have been invited to participate in this decision-making study. This study is directed by [omitted for anonymous review] and organized with the support of the Social Bee Lab (Social Behavior and Experimental Economics Laboratory) at UNAB.

\vspace{5mm}

\noindent In this study, we will pay \textbf{one (1)} out of every \textbf{ten (10)} participants, who will be randomly selected. Each selected person will receive a fixed payment of \textbf{70,000} (seventy thousand pesos) for completing the study. Additionally, they can earn up to \textbf{270,000} (two hundred and seventy thousand pesos), depending on their decisions. So, in total, if you are selected to receive payment, you can earn up to \textbf{340,000} (three hundred and forty thousand pesos) for completing this study.

\vspace{5mm}

\noindent If you are selected, you can claim your payment at any Banco de Bogotá office by presenting your ID. Your participation in this study is voluntary and you can leave the study at any time. If you withdraw before completing the study, you will not receive any payment.

\vspace{5mm}

\noindent The estimated duration of this study is 20 minutes.

\vspace{5mm}

\noindent  The purpose of this study is to understand how people make decisions. For this, we will use administrative information from the university such as the SABER 11 test scores of various students (including you). Your responses will not be shared with anyone and your participation will not affect your academic records. To maintain strict confidentiality, the research results will not be associated at any time with information that could personally identify you.

\vspace{5mm}

\noindent There are no risks associated with your participation in this study beyond everyday risks. However, if you wish to report any problems, you can contact Professor [omitted for anonymous review]. For questions related to your rights as a research study participant, you can contact the IRB office of [omitted for anonymous review].

\vspace{5mm}

\noindent By selecting the option ``I want to participate in the study" below, you give your consent to participate in this study and allow us to compare your responses with some administrative records from the university.

\begin{itemize}
  \item I want to participate in the study [advances to next page]
  \item I do not want to participate in the study
\end{itemize}

\noindent\rule{\textwidth}{1pt}

\section*{Student Information}

Please write your student code.
In case you are enrolled in more than one program simultaneously, write the code of the first program you entered:

\vspace{5mm}

\noindent[Student ID code]

\vspace{5mm}

\noindent What semester are you currently in?

\vspace{5mm}

\noindent[Slider ranging from 1 to 11]

\noindent\rule{\textwidth}{1pt}

\vspace{5mm}

\noindent[Random assignment to treatment or control]

\section*{Instructions}

The instructions for this study are presented in the following video. Please watch it carefully. We will explain your participation and how earnings are determined if you are selected to receive payment.

\vspace{5mm}

\noindent[Treatment-specific instructions in video format]

\vspace{5mm}

\noindent If you want to read the text of the instructions narrated in the video, press the ``Read instruction text" button. Also know that in each question, there will be a button with information that will remind you if that question has earnings and how it is calculated, in case you have any doubts.

\begin{itemize}
  \item I want to read the instructions text [text version below]
\end{itemize}

\noindent\rule{\textwidth}{1pt}

\vspace{5mm}

\noindent In this study, you will respond to three types of questions. First, are the belief questions. For belief questions, we will use as reference the results of the SABER 11 test that you and other students took to enter the university, focused on three areas of the exam: mathematics, reading, and English.

\vspace{5mm}

\noindent For each area, we will take the scores of all university students and order them from lowest to highest. We will then group them into 100 percentiles. The percentile is a position measure that indicates the percentage of students with an exam score that is above or below a value.

\vspace{5mm}

\noindent For example, if your score in mathematics is in the 20th percentile, it means that 20 percent of university students have a score lower than yours and the remaining 80 percent have a higher score. A sample belief question is: ``compared to university students, in what percentile is your score for mathematics?"

\vspace{5mm}

\noindent If your answer is correct, you can earn 20 thousand pesos. We say your answer is correct if the difference between the percentile you suggest and the actual percentile of your score is not greater than 7 units. For example, if you have a score that is in the 33rd percentile and you say it is in the 38th, the answer is correct because the difference is less than 7. But if you answer that it is in the 41st, the difference is greater than 7 and the answer is incorrect.

\vspace{5mm}

\noindent The second type of questions are recommendation questions and are also based on the mathematics, reading, and English areas of the SABER 11 test. We will ask you to think about the students with whom you have taken or are taking classes, to recommend from among them the person you consider best at solving problems similar to those on the SABER 11 test.

\vspace{5mm}

\noindent When you start typing the name of your recommended person, the computer will show suggestions with the full name, program, and university entry year of different students. Choose the person you want to recommend. If the name doesn't appear, check that you are writing it correctly. Do not use accents and use `n' instead of `ñ'. If it still doesn't appear, it may be because that person is not enrolled this semester or because they did not take the SABER 11 test. In that case, recommend someone else.

\vspace{5mm}

\noindent You can earn up to 250,000 pesos for your recommendation. We will multiply your recommended person's score by 100 pesos if they are in the first 50 percentiles. We will multiply it by 500 pesos if your recommended person's score is between the 51st and 65th percentile. If it is between the 66th and 80th percentile, we will multiply your recommended person's score by 1000 pesos. If the score is between the 81st and 90th percentile, you earn 1500 pesos multiplied by your recommended person's score. And if the score is between the 91st and 100th percentile, we will multiply your recommended person's score by 2500 pesos to determine the earnings.

\vspace{5mm}

\noindent The third type of questions are information questions and focus on aspects of your personal life or your relationship with the people you have recommended.


\subsection*{Earnings}

Now we will explain who gets paid for participating and how the earnings for this study are assigned. The computer will randomly select one out of every 10 participants to pay for their responses. For selected individuals, the computer will randomly choose one of the three areas, and from that chosen area, it will pay for one of the belief questions.

\vspace{5mm}

\noindent Similarly, the computer will randomly select one of the three areas to pay for one of the recommendation questions.

\vspace{5mm}

\noindent \textbf{Additionally, if you are selected to receive payment, your recommended person in the  chosen area will receive a fixed payment of 100 thousand pesos.} [Only seen if assigned to the treatment] 

\vspace{5mm}

\noindent Each person selected to receive payment for this study can earn: up to 20 thousand pesos for one of the belief questions, up to 250 thousand pesos for one of the recommendation questions, and a fixed payment of 70 thousand pesos for completing the study.

\vspace{5mm}

\noindent Selected individuals can earn up to 340 thousand pesos.

\noindent\rule{\textwidth}{1pt}

\vspace{5mm}

\noindent [Participants go through all three Subject Areas in randomized order]

\section*{Subject Areas}

\subsection*{Critical Reading}
For this section, we will use as reference the Critical Reading test from SABER 11, which evaluates the necessary competencies to understand, interpret, and evaluate texts that can be found in everyday life and in non-specialized academic fields.

\vspace{5mm} 

\noindent [Clicking shows the example question from SABER 11 below]

\vspace{5mm} 

\noindent Although the democratic political tradition dates back to ancient Greece, political thinkers did not address the democratic cause until the 19th century. Until then, democracy had been rejected as the government of the ignorant and unenlightened masses. Today it seems that we have all become democrats without having solid arguments in favor. Liberals, conservatives, socialists, communists, anarchists, and even fascists have rushed to proclaim the virtues of democracy and to show their democratic credentials (Andrew Heywood). According to the text, which political positions identify themselves as democratic?

\begin{itemize}
  \item Only political positions that are not extremist
  \item The most recent political positions historically
  \item The majority of existing political positions
  \item The totality of possible political currents
\end{itemize}

\vspace{5mm}

\noindent\rule{\textwidth}{1pt}


\subsection*{Mathematics}
This section references the Mathematics test from SABER 11, which evaluates people's competencies to face situations that can be resolved using certain mathematical tools.

\vspace{5mm} 

\noindent [Clicking shows the example question from SABER 11 below]

\vspace{5mm} 

\noindent A person living in Colombia has investments in dollars in the United States and knows that the exchange rate of the dollar against the Colombian peso will remain constant this month, with 1 dollar equivalent to 2,000 Colombian pesos. Their investment, in dollars, will yield profits of 3\% in the same period. A friend assures them that their profits in pesos will also be 3\%. Their friend's statement is:

\begin{itemize}
    \item Correct. The proportion in which the investment increases in dollars is the same as in pesos.
    \item Incorrect. The exact value of the investment should be known.
    \item Correct. 3\% is a fixed proportion in either currency.
    \item Incorrect. 3\% is a larger increase in Colombian pesos.
\end{itemize}

\vspace{5mm}

\noindent\rule{\textwidth}{1pt}


\subsection*{English}
This section uses the English test from SABER 11 as a reference, which evaluates that the person demonstrates their communicative abilities in reading and language use in this language.

\vspace{5mm} 

\noindent [Clicking shows the example question from SABER 11 below]

\vspace{5mm} 

\noindent Complete the conversations by marking the correct option.
    \begin{itemize}
        \item Conversation 1: I can't eat a cold sandwich. It is horrible!
        \begin{itemize}
            \item I hope so.
            \item I agree.
            \item I am not.
        \end{itemize}
        \item Conversation 2: It rained a lot last night!
        \begin{itemize}
            \item Did you accept?
            \item Did you understand?
            \item Did you sleep?
        \end{itemize}
    \end{itemize}

\noindent\rule{\textwidth}{1pt}

\vspace{5mm}

\noindent [Following parts are identical for all Subject Areas and are not repeated here for brevity]

\subsection*{Your Score}

Compared to university students, in which percentile do you think your [\textbf{Subject Area}] test score falls (1 is the lowest percentile and 100 the highest)?

\vspace{5mm} 

\noindent [Clicking shows the explanations below]

\vspace{5mm} 

\noindent How is a percentile calculated?

\vspace{5mm} 

\noindent A percentile is a position measurement. To calculate it, we take the test scores for all students currently enrolled in the university and order them from lowest to highest. The percentile value you choose refers to the percentage of students whose score is below yours. For example, if you choose the 20th percentile, you're indicating that 20\% of students have a score lower than yours and the remaining 80\% have a score higher than yours.

\vspace{5mm} 

\noindent What can I earn for this question?

\vspace{5mm} 

\noindent For your answer, you can earn \textbf{20,000 (twenty thousand) PESOS}, but only if the difference between your response and the correct percentile is less than 7. For example, if the percentile where your score falls is 33 and you respond with 38 (or 28), the difference is 5 and the answer is considered correct. But if you respond with 41 or more (or 25 or less), for example, the difference would be greater than 7 and the answer is incorrect.


\vspace{5mm} 

\noindent Please move the sphere to indicate which percentile you think your score falls in:

\vspace{5mm} 

\noindent[Slider with values from 0 to 100]


\vspace{5mm} 

\noindent\rule{\textwidth}{1pt}

\subsection*{Recommendation}

Among the people with whom you have taken any class at the university, who is your recommendation for the [\textbf{Subject Area}] test? Please write that person's name in the box below:

\vspace{5mm} 

\noindent \textbf{\textcolor{red}{Important:}} \textbf{You will not be considered for payment unless the recommended person is someone with whom you have taken at least one class during your studies.}


\vspace{5mm} 

\noindent Your response is only a recommendation for the purposes of this study and we will \textbf{not} contact your recommended person at any time.

\vspace{5mm} 

\noindent [Clicking shows the explanations below]

\vspace{5mm} 

\noindent Who can I recommend?

\vspace{5mm} 

\noindent Your recommendation \textbf{must} be someone with whom you have taken (or are taking) a class. If not, your answer will not be considered for payment. The person you recommend will not be contacted or receive any benefit from your recommendation.

\vspace{5mm} 

\noindent As you write, you will see up to 7 suggested student names containing the letters you have entered. The more you write, the more accurate the suggestions will be. Please write \textbf{without} accents and use the letter `n' instead of `ñ'. If the name of the person you're writing doesn't appear, it could be because you made an error while writing the name.

\vspace{5mm} 

\noindent If the name is correct and still doesn't appear, it could be because the student is not enrolled this semester or didn't take the SABER 11 test. In that case, you must recommend someone else.


\vspace{5mm} 

\noindent My earnings for this question?

\vspace{5mm} 

\noindent For your recommendation, you could receive earnings of up to 250,000 (two hundred and fifty thousand) PESOS. The earnings are calculated based on your recommendation's score and the percentile of that score compared to other UNAB students, as follows:

\begin{itemize}
  \item We will multiply your recommendation's score by \$100 (one hundred) pesos if it's between the 1st and 50th percentiles
  \item We will multiply your recommendation's score by \$500 (five hundred) pesos if it's between the 51st and 65th percentiles
  \item We will multiply your recommendation's score by \$1000 (one thousand) pesos if it's between the 66th and 80th percentiles
  \item We will multiply your recommendation's score by \$1500 (one thousand five hundred) pesos if it's between the 81st and 90th percentiles
  \item We will multiply your recommendation's score by \$2500 (two thousand five hundred) pesos if it's between the 91st and 100th percentiles
\end{itemize}


\vspace{5mm} 

\noindent This is illustrated in the image below:

\begin{figure}[H]
    \centering
    \includegraphics[width=0.65\textwidth]{/Users/reha.tuncer/Documents/GitHub/icfes-referrals/figures/bonus_structure.png}
    \caption{Earnings for recommendation questions}
    \label{fig:earnings}
\end{figure}


\vspace{5mm} 

\noindent For example, if your recommendation got 54 points and the score is in the 48th percentile, you could earn 54x100 = 5400 PESOS. But, if the same score of 54 points were in the 98th percentile, you could earn 54x2500 = 135,000 PESOS.

\vspace{5mm} 

\noindent [Text field with student name suggestions popping up as participant types]

\vspace{5mm} 

\noindent\rule{\textwidth}{1pt}

\subsection*{Relationship with your recommendation}
How close is your relationship with your recommendedation: ``[Name of the student selected from earlier]"? (0 indicates you are barely acquaintances and 10 means you are very close)

\vspace{5mm} 

\noindent [Slider with values from 0 to 10]

\vspace{5mm} 

\noindent\rule{\textwidth}{1pt}

\subsection*{Your recommendation's score}
Compared to university students, in which percentile do you think [Name of the student selected from earlier]'s score falls in the [\textbf{Subject Area}] test (1 is the lowest percentile and 100 the highest)?

\vspace{5mm} 

\noindent [Clicking shows the explanations below]

\vspace{5mm} 

\noindent How is a percentile calculated?

\vspace{5mm} 

\noindent A percentile is a position measurement. To calculate it, we take the test scores for all students currently enrolled in the university and order them from lowest to highest. The percentile value you choose refers to the percentage of students whose score is below yours. For example, if you choose the 20th percentile, you're indicating that 20\% of students have a score lower than yours and the remaining 80\% have a score higher than yours.

\vspace{5mm} 

\noindent What can I earn for this question?

\vspace{5mm} 

\noindent For your answer, you can earn \textbf{20,000 (twenty thousand) PESOS}, but only if the difference between your response and the correct percentile is less than 7. For example, if the percentile where your recommended person's score falls is 33 and you respond with 38 (or 28), the difference is 5 and the answer is considered correct. But if you respond with 41 or more (or 25 or less), for example, the difference would be greater than 7 and the answer is incorrect.

\vspace{5mm} 

\noindent  Please move the sphere to indicate which percentile you think your recommended person's score falls in:

\vspace{5mm} 

\noindent[Slider with values from 0 to 100]

\vspace{5mm} 

\noindent\rule{\textwidth}{1pt}

\section*{Demographic Information}

What is the highest level of education achieved by your father?

\vspace{5mm} 

\noindent [Primary, High School, University, Graduate Studies, Not Applicable]

\vspace{5mm} 

\noindent What is the highest level of education achieved by your mother?

\vspace{5mm} 

\noindent [Primary, High School, University, Graduate Studies, Not Applicable]

\vspace{5mm} 

\noindent Please indicate the socio-economic group to which your family belongs:

\vspace{5mm}

\noindent [Group A (Strata 1 or 2), Group B (Strata 3 or 4), Group C (Strata 5 or 6)]

\vspace{5mm}

\noindent\rule{\textwidth}{1pt}

\section*{UNAB Students Distribution}
Thinking about UNAB students, in your opinion, what percentage belongs to each socio-economic group? The total must sum to 100\%:

\vspace{5mm}

\noindent [Group A (Strata 1 or 2) percentage input area] 

\noindent [Group B (Strata 3 or 4) percentage input area] 

\noindent [Group C (Strata 5 or 6) percentage input area]

\noindent [Shows sum of above percentages]

\vspace{5mm} 

\noindent\rule{\textwidth}{1pt}

\section*{End of the Experiment}
Thank you for participating in this study.

\vspace{5mm}

\noindent If you are chosen to receive payment for your participation, you will receive a confirmation to your UNAB email and a link to fill out a form with your information. The process of processing payments is done through Nequi and takes approximately 15 business days, counted from the day of your participation.

\vspace{5mm}

\noindent [Clicking shows the explanations below]

\vspace{5mm}

\noindent Who gets paid and how is it decided?

\vspace{5mm}

\noindent The computer will randomly select one out of every ten participants in this study to be paid for their
decisions.

\vspace{5mm}

\noindent For selected individuals, the computer will randomly select one area: mathematics, reading, or English, and from that area will select one of the belief questions. If the answer to that question is correct, the participant will receive 20,000 pesos.

\vspace{5mm}

\noindent The computer will randomly select an area (mathematics, critical reading, or English) to pay for one of the recommendation questions. The area chosen for the recommendation question is independent of the area chosen for the belief question. The computer will take one of the two recommendations you have made for the chosen area. Depending on your recommendation's score, you could win up to 250,000 pesos.

\vspace{5mm}

\noindent Additionally, people selected to receive payment for their participation will have a fixed earnings of 70,000 pesos for completing the study.

\vspace{5mm}

\noindent\rule{\textwidth}{1pt}

\section*{Participation}
In the future, we will conduct studies similar to this one where people can earn money for their participation. The participation in these studies is by invitation only. Please indicate if you are interested in being invited to other studies similar to this one:

\vspace{5mm}

\noindent [Yes, No]


\end{document}
