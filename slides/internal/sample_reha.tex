%----------------------------------------------------------------------------------------
%	PACKAGES AND THEMES
%----------------------------------------------------------------------------------------
\documentclass[aspectratio=169,xcolor=dvipsnames]{beamer}
\usetheme{SimplePlus}
\usecolortheme{default}
\definecolor{grayedout}{RGB}{150,150,150}

\usepackage{threeparttable}
\usepackage{hyperref}
\usepackage{subfig}
\usepackage{bbm}
\usepackage{graphicx} % Allows including images
\usepackage{booktabs} % Allows the use of \toprule, \midrule and \bottomrule in tables

\setbeamertemplate{itemize item}{\color{structure!100}\scriptsize\raise1.25pt\hbox{\donotcoloroutermaths$\bullet$}}
\setbeamertemplate{itemize subitem}{\color{structure!100}\scriptsize\raise1.25pt\hbox{\donotcoloroutermaths$\bullet$}}

%----------------------------------------------------------------------------------------
%	TITLE PAGE
%----------------------------------------------------------------------------------------

\title[short title]{Essays on Productivity Beliefs} % The short title appears at the bottom of every slide, the full title is only on the title page
\subtitle{3-minute presentation: PhD Thesis Overview}

\author[Pin-Yen] {Reha Tuncer - University of Luxembourg}

\date{13 March 2025} % Date, can be changed to a custom date


%----------------------------------------------------------------------------------------
%	PRESENTATION SLIDES
%----------------------------------------------------------------------------------------

\begin{document}

\begin{frame}
    % Print the title page as the first slide
    \titlepage
\end{frame}
\begin{frame}{Part 1: Productivity of the Self}
    \begin{itemize}
        \item<1->{\only<1>{Firms monetize user attention, which may hamper individual productivity}\only<2->{{\color{grayedout}Firms monetize user attention, which may hamper individual productivity}}}
        \item<2->{\only<2>{Can the design of a fun activity affect the productivity goals of individuals?}\only<3->{{\color{grayedout}Can the design of a fun activity affect the productivity goals of individuals?}}}
        \item<3->{\only<3>{Platform where participants can earn money by completing tasks or watch videos}\only<4->{{\color{grayedout}Platform where participants can earn money by completing tasks or watch videos}}}
        \item<4->{\only<4>{Allocate time between tasks and videos, next day execute the plan}\only<5->{{\color{grayedout}Allocate time between tasks and videos, next day execute the plan}}}
        \item<5->{\only<5>{Randomly assign participants to different video conditions where one is made for watching more, and remind them their plan}\only<6->{{\color{grayedout}Randomly assign participants to different video conditions where one is made for watching more, and remind them their plan}}}
        \item<6->{\only<6>{Participants stick to their plan and remain unaffected by the video conditions}\only<7->{{\color{grayedout}Participants stick to their plan and remain unaffected by the video conditions}}}
        \item<7->{\only<7>{Subtle intervention in a controlled environment, importance of self-imposed objectives in maintaining productivity}\only<8->{{\color{grayedout}Subtle intervention in a controlled environment, importance of self-imposed objectives in maintaining productivity}}}
    \end{itemize}
\end{frame}

\begin{frame}{Part 2: Productivity of Others}
    \begin{itemize}
        \item<1->{\only<1>{Beliefs about others' productivity matters when composing teams, evaluating coworkers, hiring employees}\only<2->{{\color{grayedout}Beliefs about others' productivity matters when composing teams, evaluating coworkers, hiring employees}}}
        \item<2->{\only<2>{How well do peers identify the productivity of their classmates across hard and soft skills?}\only<3->{{\color{grayedout}How well do peers identify the productivity of their classmates across hard and soft skills?}}}
        \item<3->{\only<3>{Incentivize university students to guess most productive peers without access to performance data}\only<4->{{\color{grayedout}Incentivize university students to guess most productive peers without access to performance data}}}
        \item<4->{\only<4>{Evaluate peers across IQ, EQ, and SAT Math and Verbal areas and guess the top}\only<5->{{\color{grayedout}Evaluate peers across IQ, EQ, and SAT Math and Verbal areas and guess the top}}}
        \item<5->{\only<5>{Peers identify highly productive others across exam results and IQ, but not EQ}\only<6->{{\color{grayedout}Peers identify highly productive others across exam results and IQ, but not EQ}}}
        \item<6->{\only<6>{Social skills are harder to observe and require diverse evaluation methods}\only<7->{{\color{grayedout}Social skills are harder to observe and require diverse evaluation methods}}}
    \end{itemize}
\end{frame}
\end{document}

% \begin{frame}{Inequality and social class}
%      \begin{itemize}
%     \item Persistent differences in labor market outcomes based on social class that go beyond race and gender
%     \item Social capital stands out as a key factor controlling for productivity and demographics (Stansbury and Rodriguez, 2024)
%     \item Better connected individuals have better employment opportunities (Montgomery, 1991; Calvó-Armengol \& Jackson, 2004; Chetty et al., 2022a)
%     \item Biases like homophily may prevent social classes from interacting with one another (McPherson, Smith-Lovin, and Cook; 2001)  
%     \end{itemize}
% \end{frame}

% \begin{frame}{This paper}
%     \begin{itemize}
%     \item Field experiment in a highly segregated society with very high income equality 
%     \item Study the existence of social class bias in a university setting where we abstract away from spatial segregation between social classes (Chetty et al., 2022b)
%    \item Inspiration from job referral experiments where participants identify and nominate skilled individuals under perfomance incentives (Beaman \& Magruder, 2012; Beaman, Keleher, Magruder; 2018)
%    \item Show how well peers identify different skills of their classmates, and whether there is a bias against lower social classes
%    \item Introduce quota-like incentives as a treatment to mitigate social class bias
% \end{itemize}

% \end{frame}

% \begin{frame}{Setting in Colombia}\label{setting}
%     \begin{itemize}
%         \item Among the most unequal countries in the world, where the top decile earn 50 times more than the poorest (World Bank, 2024) 
%         \item Nationwide classification system which categorizes the population into 6 strata, aligning with existing social class divisions (Guevara \& Shields, 2019; Uribe-Mallarino, 2008)
%         \item Sample from end-of-first-year students in a mid-sized university with access to academic transcripts and demographics
%         \item Mixed student body with 57\% Low-SES in sample \hyperlink{dist-strata}{\beamerbutton{distribution}}
%     \end{itemize}
% \end{frame}

% \begin{frame}{Design}
%     \begin{figure}
%         \includegraphics[width=\linewidth]{/Users/reha.tuncer/Documents/GitHub/Inequality-Skills-and-Referrals/figures/figure-1.jpg} 
%         \caption{Timeline} 
%         \label{fig:1}
%       \end{figure}   
%       \begin{itemize}
%         \item Three sequential parts with sessions organized at the classroom level
%         \item Experimental variation in Part 2 with incentives to mitigate social class bias
%     \end{itemize} 
% \end{frame}

% \begin{frame}{Part 1: Skill Assessment}
%     \setcounter{subfigure}{0} 
%     \begin{figure}
%         \centering
%         \subfloat[Cognitive Skill \label{fig:c}]{\includegraphics[width=0.25\linewidth]{/Users/reha.tuncer/Documents/GitHub/Inequality-Skills-and-Referrals/figures/Ravens Survey.jpg}}\qquad
%         \subfloat[Social Skill \label{fig:d}]{\includegraphics[width=0.35\linewidth]{/Users/reha.tuncer/Documents/GitHub/Inequality-Skills-and-Referrals/figures/social-skill-example.png}}
%       \label{fig:2}
%       \end{figure}
%     \begin{itemize}
% \item Participants perform two incentivized skill tests inside the classroom
% \item No performance feedback, only a brief explanation of the skill 
% \item Uncorrelated and job market relevant (Kuhn \& Weinberger, 2005; Heckman, 2006; Heckman \& Kautz, 2012; Weinberger, 2014; Deming, 2017) 
%     \end{itemize}
% \end{frame}

% \begin{frame}{Part 2: Skill Referrals}
%     \setcounter{subfigure}{0} 
%     \begin{figure}
%         \centering
%         \subfloat[Baseline \label{fig:c2}]{\includegraphics[width=0.35\linewidth]{/Users/reha.tuncer/Documents/GitHub/Inequality-Skills-and-Referrals/figures/baseline.png}}\qquad
%         \subfloat[Treatment \label{fig:d2}]{\includegraphics[width=0.35\linewidth]{/Users/reha.tuncer/Documents/GitHub/Inequality-Skills-and-Referrals/figures/quota.png}}
%       \label{fig:3}
%       \end{figure}
%     \begin{itemize}
% \item Participants privately nominate three peers from their classroom for each skill
% \item Nominations among the top 3 highest performing classmates to earn a bonus
% \item \textbf{Treatment} changes the top 3 composition for half of the classroom
% \item Testable assumption: Can peers identify different skills?
%     \end{itemize}
% \end{frame}

% \begin{frame}{Part 3: SES Guessing Ability}\label{guessingabi}
%     \begin{figure}
%         \includegraphics[width=0.35\linewidth]{/Users/reha.tuncer/Documents/GitHub/Inequality-Skills-and-Referrals/figures/guessing.png} 
%         \caption{SES Guessing} 
%         \label{fig:4}
%       \end{figure}  
%     \begin{itemize}
% \item Participants privately nominate three low-SES peers from their classroom
% \item Nominations among low-SES to earn a bonus
% \item Testable assumption: Can peers observe social class? \hyperlink{dist-guess}{\beamerbutton{distribution}}
%     \end{itemize}
% \end{frame}

% \begin{frame}{Procedure}
%     \begin{figure}
%         \includegraphics[width=0.45\linewidth]{/Users/reha.tuncer/Documents/GitHub/Inequality-Skills-and-Referrals/figures/Picture1.png} 
%         \caption{Participants in a classroom} 
%         \label{fig:5}
%       \end{figure}  
%     \begin{itemize}
% \item 863 students across 35 classrooms invited, 713 completed (83\%)
% \item \underline{\textbf{\href{https://doi.org/10.17605/OSF.IO/V9T3W}{Link}}} to pre-registered design and hypotheses 
% \item Median length 20 minutes
% \item \$26 for 117 lottery winners
%     \end{itemize}
% \end{frame}

% \begin{frame}{Descriptive Statistics}
%     \begin{table}[htbp]
%         \centering
%         \begin{threeparttable}
%             \caption{Treatment Balance}
%             \begin{tabular}{lccc}
%                 \toprule
%                 Variable & Baseline & Treatment & \textit{p}-value \\
%                 \midrule
%                 Low-SES (\%) & 59.0 & 55.0 & 0.297 \\
%                 Cognitive skill\tnote{a} & 10.04 & 10.27 & 0.322 \\
%                 Social skill\tnote{b} & 18.45 & 18.50 & 0.886 \\
%                 GPA & 3.95 & 3.95 & 0.828 \\
%                 \# semesters at UNAB & 3.17 & 3.18 & 0.916 \\
%                 \bottomrule
%             \end{tabular}
%             \begin{tablenotes}
%                 \small
%                 \item[a] Max. score: 18 points
%                 \item[b] Max. score: 36 points
%                 \item[] \textit{Note:} p-values from two-sided t-tests comparing group means.
%             \end{tablenotes}
%             \label{tab:descriptive_stats}
%         \end{threeparttable}
%     \end{table}
% \end{frame}

% \begin{frame}{Dependent Variable}
%     \begin{align*}
%         y_{ic}^s = \frac{\sum_{j \neq i} r_{ijc}^s}{n_c - \mathbbm{1}(i \in B_c)} \times 100
%     \end{align*}
%     \begin{itemize}
%         \item Percentage share of nominations individual $i$ gets for skill $s$
%         \item From classmates in class $c$ and in Baseline ($B_c$)
%         \item Sum of nominations ($\sum_{j \neq i} r_{ijc}^s$) divided by the max. possible nominations ($n_c$)
%         \item Accounting for no-self-referral rule
%         \item Allows comparing characteristics of individuals who get nominated more often across classrooms of different sizes
% \end{itemize}        
% \end{frame}

% \begin{frame}{Dependent Variable}
%     \begin{figure}
%         \includegraphics[width=0.5\linewidth]{/Users/reha.tuncer/Documents/GitHub/Inequality-Skills-and-Referrals/figures/ref_skill_baseline.png} 
%         \caption{Distribution of nominations in Baseline} 
%         \label{fig:6}
%       \end{figure}        
% \end{frame}

% \begin{frame}{Result 1: Can peers identify skills at baseline?}
%     \begin{table}[H]
%         \centering
%         \begin{threeparttable}
%         \begin{tabular*}{0.5\textwidth}{@{\extracolsep{\fill}}l c c@{}}
%         \toprule
%         & (1) & (2) \\
%         & Cognitive & Social \\
%         \midrule
%         Skill & 1.197** & 0.037 \\
%         & (0.479) & (0.510) \\
%         Constant & 12.986*** & 13.049*** \\
%         & (0.653) & (0.521) \\
%         \midrule
%         Observations & 665 & 665 \\
%         \bottomrule
%         \end{tabular*}
%         \end{threeparttable}
%         \caption{Share of nominations conditional on skill}\label{tab:regression-results1}
%       \end{table}
%     \begin{itemize}
%         \item Cognitive skill can be identifed by peers, but not social skill
%         \item Magnitude of the effect very modest
% \end{itemize}        
% \end{frame}

% \begin{frame}{Result 1: Can peers identify skills at baseline?}
%     \begin{table}[H]
%         \centering
%         \begin{threeparttable}
%         \begin{tabular*}{0.5\textwidth}{@{\extracolsep{\fill}}l c c@{}}
%         \toprule
%         & (1) & (2) \\
%         & Cognitive & Social \\
%         \midrule
%         Skill & 0.873* & -0.278 \\
%         & (0.467) & (0.460) \\
%         GPA & 3.949*** & 3.429*** \\
%         & (0.664) & (0.581) \\
%         Constant & 12.806*** & 12.891*** \\
%         & (0.662) & (0.636) \\
%         \midrule
%         Observations & 665 & 665 \\
%         \bottomrule
%         \end{tabular*}
%         \caption{Share of nominations conditional on skill and GPA}\label{tab:regression-results2}
%         \end{threeparttable}
%       \end{table}

%     \begin{itemize}
% \item Participants use academic performance as a proxy for both skills
%     \end{itemize}
% \end{frame}

% \begin{frame}{Result 2: Nomination strategies}
%     \begin{figure}
%         \includegraphics[width=0.45\linewidth]{/Users/reha.tuncer/Documents/GitHub/Inequality-Skills-and-Referrals/figures/ses_dist3.png} 
%         \caption{Distribution of identical nominations} 
%         \label{fig:7}
%       \end{figure}  
%     \begin{itemize}
% \item Up to 6 unique nominations can be made but the observed diversity is much less
% \item More than half of the sample nominate the same classmates 2 or 3 times
%     \end{itemize}
% \end{frame}

% \begin{frame}{Result 2: Nomination strategies}
%     \begin{align*}
%         y_{ic}^{s} = y_{ic}^{s,single} + y_{ic}^{twice}
%     \end{align*}
%     \begin{itemize}
% \item Percentage share of nominations received can be decomposed into nominations for skill $s$ only versus for both
% \item Help us identify whether nominations for one skill are different than for both
%     \end{itemize}
% \end{frame}

% \begin{frame}{Result 2: Nomination strategies}
%     \begin{table}[H]
%         \centering
%         \begin{threeparttable}
%         \begin{tabular*}{0.75\textwidth}{@{\extracolsep{\fill}}l c c c@{}}
%         \toprule
%         & (1) & (2) & (3) \\
%         & Twice & Cognitive Only & Social Only \\
%         \midrule
%         GPA & 3.172*** & 0.801** & 0.260 \\
%         & (0.464) & (0.391) & (0.334) \\
%         Cognitive Skill & -0.042 & 1.006*** & \\
%         & (0.416) & (0.270) & \\
%         Social Skill & -0.353 & & 0.086 \\
%         & (0.304) & & (0.381) \\
%         Constant & 7.407*** & 5.400*** & 5.485*** \\
%         & (0.524) & (0.386) & (0.376) \\
%         \midrule
%         Observations & 665 & 665 & 665 \\
%         \bottomrule
%         \end{tabular*}
%         \end{threeparttable}
%       \end{table}
%     \begin{itemize}
% \item 3 types: GPA, GPA and cognitive skill, neither social skill or GPA
%     \end{itemize}
% \end{frame}

% \begin{frame}{Result 3: Treatment and social class}
%     \begin{table}[H]
%         \centering
%         \begin{threeparttable}
%         \begin{tabular*}{0.5\textwidth}{@{\extracolsep{\fill}}l c c@{}}
%         \toprule
%         & (1) & (2) \\
%         & Cognitive & Social \\
%         \midrule
%         Treat & -0.073 & 0.299 \\
%         & (0.755) & (0.716) \\
%         Low-SES & -1.230 & 0.364 \\
%         & (1.079) & (1.282) \\
%         Treat × Low-SES & -0.167 & -0.835 \\
%         & (1.117) & (1.181) \\
%         Constant & 13.551*** & 12.706*** \\
%         & (0.986) & (0.952) \\
%         \midrule
%         Performance controls & YES & YES \\
%         Observations & 1,330 & 1,330 \\
%         \bottomrule
%         \end{tabular*}
%         \end{threeparttable}
%       \end{table}
%     \begin{itemize}
% \item No low-SES bias when controlling for skills and GPA, no treatment effect 
%     \end{itemize}
% \end{frame}

% \begin{frame}{Result 3: Treatment and social class}
%     \begin{table}[H]
%         \centering
%         \begin{threeparttable}
%         \begin{tabular*}{0.75\textwidth}{@{\extracolsep{\fill}}l c c c@{}}
%         \toprule
%         & (1) & (2) & (3) \\
%         & Twice & Cognitive Only & Social Only \\
%         \midrule
%         Treat & 0.436 & -0.509 & -0.136 \\
%         & (0.817) & (0.598) & (0.523) \\
%         Low-SES & 0.857 & -2.074*** & -0.510 \\
%         & (0.920) & (0.722) & (0.613) \\
%         Treat × Low-SES & -1.584 & 1.417** & 0.750 \\
%         & (1.159) & (0.656) & (0.717) \\
%         Constant & 6.952*** & 6.591*** & 5.765*** \\
%         & (0.727) & (0.654) & (0.562) \\
%         \midrule
%         Performance controls & YES & YES & YES \\
%         Observations & 1,330 & 1,330 & 1,330 \\
%         \bottomrule
%         \end{tabular*}
%         \label{tab:twice_treat}
%         \end{threeparttable}
%       \end{table}
%     \begin{itemize}
% \item Low-SES bias only for cognitive skill nominations, treatment mitigates
%     \end{itemize}
% \end{frame}

% \begin{frame}{Result 3: Treatment and social class}
%     \begin{table}[H]
%         \centering
%         \begin{threeparttable}
%         \begin{tabular*}{0.5\textwidth}{@{\extracolsep{\fill}}l c c@{}}
%         \toprule
%         & (1) & (2) \\
%         & Low-SES & High-SES \\
%         \midrule
%         Treat & 0.541 & -0.872 \\
%         & (0.754) & (0.946) \\
%         Low-SES & -0.724 & -1.681 \\
%         & (0.717) & (0.993) \\
%         Treat × Low-SES & -0.552 & 2.711** \\
%         & (0.927) & (1.289) \\
%         Constant & 4.359*** & 4.766*** \\
%         & (0.472) & (0.663) \\
%         \midrule
%         Performance controls & YES & YES \\
%         Observations & 1,328 & 1,310 \\
%         \bottomrule
%         \end{tabular*}
%         \end{threeparttable}
%       \end{table}
%     \begin{itemize}
% \item High-SES drive the results when adjusting for low-SES share across classrooms 
%     \end{itemize}
% \end{frame}

% \begin{frame}{Conclusion}
%     \begin{itemize}
%         \item Limited peer capacity in identifying cognitive and social skills: First-year students may not know each other's skills very well, or the choice/design of the skill assessment was not optimal
%         \item For half of the sample, nomination decisions are characterized by grades instead 
%         \item No general social class bias in nominations and no treatment effect when controlling for performance
%         \item Less than 25\% of nominations affected by low-SES bias, driven by high-SES peers
%         \item For this subsample treatment is effective as well
%             \end{itemize}
% \end{frame}

% \begin{frame}{Distribution of SES guessing ability}\label{dist-guess}
%     \begin{figure}
%     \includegraphics[width=0.5\linewidth]{/Users/reha.tuncer/Documents/GitHub/Inequality-Skills-and-Referrals/figures/guess_ratio.png}
%     \caption{SES guessing ability \hyperlink{guessingabi}{\beamerbutton{back}}}
%     \end{figure}
% \end{frame}

% \begin{frame}{Distribution of strata}\label{dist-strata}
%     \begin{figure}
%     \includegraphics[width=0.5\linewidth]{/Users/reha.tuncer/Documents/GitHub/Inequality-Skills-and-Referrals/figures/strato.png}
%     \caption{Sample strata distribution at UNAB \hyperlink{setting}{\beamerbutton{back}}}
%     \end{figure}
% \end{frame}

